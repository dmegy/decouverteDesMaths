
Attention, la présentation qui suit diffère sans doute beaucoup de celle vue en terminale : il faut faire l'effort de l'étudier en détail même si l'ordre dans lequel les notions sont introduites semble \og mauvais\fg : en fait, c'est le \og bon\fg{}  ordre.

Le cours d'arithmétique des polynômes suivra le même canevas (définitions semblables, mêmes lemmes aux mêmes endroits, mêmes preuves), de même que le cours d'algèbre générale sur les anneaux par la suite.

\section{Préliminaires}

\subsection{Division euclidienne}
\begin{proposition}[Division euclidienne]
Soit $a\in \N$ et $b\in \N*$. Il existe un unique couple $(b,r) \in \N^2$ vérifiant les deux propriétés suivantes:
\begin{enumerate}
\item $a=bq+r$;
\item $r < b$.
\end{enumerate}
L'entier $b$ est le \emph{quotient} de la division euclidienne de $a$ par $b$, et $r$ est le \emph{reste}.
Effectuer la division euclidienne de $a$ par $b$, c'est écrire $a = bq+r$ avec $b$ et $q$ comme plus haut.
\end{proposition}

Exemple : $17=5\times 3 + 2$ est la division euclidienne de $17$ par $5$. Le quotient est $3$ et il reste $2$. Par contre, l'écriture $17=5\times 2+7$ bien que correcte  n'est pas une division euclidienne, car le reste \emph{doit} être strictement inférieur à $5$, dans une division euclidienne.

\subsection{Idéaux de $\Z$}

Soit $I \subseteq \Z$ une partie de $\Z$. On dit que $I$ est un \emph{idéal} de $\Z$ si
\begin{enumerate}
\item C'est un \emph{sous-groupe} de $\Z$, c'est-à-dire $I$ contient $0$ et est stable par addition et opposé : 
\[ \forall x, y\in I, \: x+y \in I \text{et} -x \in I.\]
\item Il est \emph{absorbant pour la multiplication} c'est-à-dire:
\[  \text{Si } x \in I \text{ et }n\in \Z, \text{alors } n\cdot x \in I.\]
\end{enumerate}

\begin{proposition}
Tout idéal de $\Z$ est de la forme $\{k\alpha\:\mid \: k \in \Z\}$, avec $\alpha /in \N$. (Un tel ensemble est noté $\alpha\Z$.)
\end{proposition}

\begin{proof}
On va montrer que les sous-groupes de $\Z$ sont de cette forme, et que ce sont des idéaux.

Soit $G \subseteq \Z$ un sous-groupe. Soit $G^*_+ = G \cap \N^*$. Il y a deux cas:
\begin{enumerate}
\item Si $G^*_+$ est vide, cela signifie que $G$ ne possède aucun élément strictement positif. Comme $G$ est stable par opposé, il ne peut pas non plus contenir d'éléments strictement négatifs. Cela signifie que $G=\{0\}$.
\item Sinon, c'est une partie non vide de $\N$, qui possède donc un plus petit élément, notons-le $\alpha$.
Par définition, $G$ est stable par somme et opposé, donc $2\alpha\in G$ et $-\alpha \in G$ et plus généralement, pour tout $k\in \Z$, on a $k\alpha \in G$.
Donc $\alpha \subseteq G$. Montrons l'inclusion inverse.
Soit $x\in G$, positif. \'Ecrivons la division euclidienne de $x$ par $\alpha$. On a $x = \alpha q + r$, avec $r<\alpha$. Comme $G$ est stable par somme et différence et que $\alpha q \in G$, on en déduit que $r = x-\alpha q$ est également dans $G$. Or, $r<\alpha$, donc par minimalité de $\alpha$, $r=0$, ce qui montre que $x = \alpha q$, donc que $x\in \alpha\Z$.
Si $x$ est négatif, ce qui précède montre que $-x\in \alpha\Z$, donc que $x\in \alpha\Z$.
\end{enumerate}

On vérifie ensuite sans peine que tous les sous-groupes de $\Z$ sont en fait des idéaux de $\Z$.
\end{proof}


L'entier $\alpha$ dans la définition est appelé le générateur principal de $G$.


\section{Pgcd}

\begin{proposition}
Soient $a$ et $b$ deux entiers. L'ensemble $\{ak+bl\:\mid\: k, l\in \Z\}$ noté par définition $a\Z+b\Z$ ou $\langle a,b\rangle$, est un idéal de $\Z$. 
\end{proposition}
\begin{proof}
Appliquer la définition de sous-groupe, ce qui prouve que c'est un idéal par la section précédente.
\end{proof}

\begin{definition}
Soient $a$ et $b$ des entiers. Le générateur principal de $a\Z+b\Z$ est appelé le pgcd de $a$ et $b$, il est noté $\pgcd(a,b)$ ou bien $a\wedge b$.
\end{definition}

\begin{proposition}
Soient $a$ et $b$ des entiers, et $d=\pgcd(a,b)$.
On a les propriétés suivantes
\begin{enumerate}
\item L'entier $a$ est dans $a\Z+b\Z$, donc $d$ divise $a$. De même, $d$ divise $b$.
\item L'entier $df$ est dans $d\Z$, donc il existe $k$ et $l$ dans $\Z$, tels que $d = ak+bl$. On dit que $(k,l)$ est un couple (ou paire, par abus de langage) de Bézout pour $a$ et $b$. L'égalité $d=ak+bl$ est appelée \emph{relation de Bézout}.
\item Au sens de la divisibilité, $d$ est le plus grand diviseur commun  de $a$ et $b$. Ceci explique le nom (\emph{plus grand commun diviseur} de $d$. Cette propriété est précisée dans la proposition suivante.
\item Si $d=0$, alors $a=b=0$.
\item On a $\pgcd(a,b)=\pgcd(a,b) = \pgcd(a,-b)$, car $a\Z+b\Z = b\Z+a\Z=a\Z+(-b)\Z$.
\item $\pgcd(a,0)=|a|$.
\item $\pgcd(a,1)=1$.
\end{enumerate}
\end{proposition}

\begin{proposition}
Si $x>0$, $x|a$ et $x|b$, et $\forall m, m|a \text{ et } m|b \implies m|x$, alors $x=d$.
\end{proposition}
\begin{proof}
Si $x|a$ et $x|b$, alors $x|d$. D'autre part, $d|a$ et $d|a$, donc $d|x$. Donc finalement, $d=x$.
Attention, la condition $x>0$ est indispensable pour ce raisonnement. Deux entiers relatifs peuvent se diviser l'un l'autre, comme $1$ et $-1$, sans être égaux.
\end{proof}

\begin{proposition}
Soit $k>0$. On a $\pgcd(ka,kb)=k\pgcd(a,b)$.
\end{proposition}
\begin{proof}
Notons provisoirement $d_1 = \pgcd(a,b)$ et $d_2 = \pgcd(ka,kb)$.

Comme $d_1|a$ et $d_1|b$, on a $kd_1|ka$ et $kd_1|kb$ donc finalement $kd_1|d_2$. En  particulier, $k|d_2$ donc $\frac{d_2}{k}$ est un entier.

D'autre part, $d_2|ka$ et $d_2|kb$, donc  en divisant par $k$ et en utilisant la remarque précédente, on a $\frac{d_2}{k} | a$ et $\frac{d_2}{k} | b$ donc $\frac{d_2}{k} | d_1$, d'où $d_2 | kd_1$. 

Comme $kd_1$ et $d_2$ sont positifs, on en déduit $d_2=kd_1$.
\end{proof}

\subsection{Algorithme d'Euclide}

\begin{lemme}[d'Euclide]
Soient $a$, $b$ et $k$ des entiers relatifs. Alors:
\[ \pgcd(a,b) = \pgcd(a+kb,b).\]
\end{lemme}

\begin{proof}

\end{proof}

\section{Nombres premiers}

\subsection{Définition}

\subsection{Factorisation en produit d'irréductibles}

\section{Ppcm}

\section{Compléments}

\subsection{Petit théorème de Fermat}
\subsection{Théorème chinois}
\subsection{Indicatrice d'Euler}
