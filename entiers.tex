\chapter{Entiers, ensembles finis et combinatoire}
\minitoc
\hyperlink{toc}{\retourTOC}

\section{L'ensemble $\N$ et la récurrence}
\index{$\N$}

L'ensemble $\N$ est supposé connu. Il vérifie les propriétés suivantes (les ensembles ordonnés sont traités dans un chapitre ultérieur mais le vocabulaire devrait être connu):
\begin{enumerate}
\item tout partie non vide majorée admet un plus grand élément;
\item toute partie non vide admet un plus petit élément.
\end{enumerate}

C'est la deuxième propriété qui permet de distinguer $\N$ de $\Z$ et qui fonde le principe de récurrence.


\begin{definition}[Propriété héréditaire]
Soit $A(n)$ une assertion dépendant d'un paramètre $n\in \N$. 
On dit qu'elle est \emph{héréditaire} si:
\[
\forall n\in \N,\: A(n)\implies A(n+1).
\]
(On définit de manière similaire l'hérédité à partir d'un certain rang $n_0$, au lieu du rang $0$.)
\end{definition}

\begin{theoreme}[Principe de récurrence]
\index{principe de récurrence}
Soit $A(n)$ une propriété dépendant d'un paramètre $n\in \N$. 

Si $A(0)$ est vraie et que la propriété est héréditaire, alors la propriété est vraie pour tout $n\in\N$, autrement dit:
\[
\big( A(0) \text{ et }(\forall n\in \N,\: A(n)\implies A(n+1))\big) \implies (\forall n\in \N, \: A(n)).
\]
\end{theoreme}
\begin{proof}Supposons la propriété vraie au rang $0$ et héréditaire.

Montrons que la partie $A = \{n\in \N\:\mid\: A(n)\text{ est fausse}\} \subseteq \N$ est vide.

Si $A$ est non-vide, elle possède un plus petit élément $m$ avec $m\geq 1$ puisque $A(0)$ est vraie. Donc $m-1 \in \N$ et $A(m-1)$ est vraie par définition de $m$. Par hérédité, $A(m)=A((m-1)+1)$ est vraie, contradiction. Donc $A=\varnothing$ ce qui signifie exactement: $\forall \in\N,\:A(n)$.
\end{proof}

\begin{theoreme}[Récurrence forte]\index{principe de récurrence!forte}
Soit $A(n)$ une propriété dépendant d'un paramètre $n\in \N$. 

Si $A(0)$ est vraie et que $\forall n\in \N, (\forall k\leq n, A(k))\implies A(n+1)$, alors $A(n)$ est vraie pour tout $n\in N$.
\end{theoreme}
\begin{proof}
Exercice. Appliquer le principe de  récurrence simple à la propriété $B(n)$ : \og $\big(\forall k\leq n,\: A(k)\big)$\fg.
\end{proof}
\section{Ensembles finis}

\subsection{Ensembles $\llbracket a,b\rrbracket$}
\index{$\llbracket a,b\rrbracket$}

Si $a, b\in \Z$, on note $\llbracket a,b\rrbracket$ l'ensemble $\{n\in \Z\:\mid\: a\leq n \leq b\}$. Si $b<a$, cet ensemble est vide.

\begin{lemme} Soient $m\geq 1$ et $a\in \llbracket 1,m\rrbracket$. L'application 
\[
\phi : \llbracket 1,m\rrbracket \setminus \{a\} \to \llbracket 1,m-1\rrbracket, 
x\mapsto \begin{cases}x & \text{ si }x<a\\x-1 & \text{ si }x>a\end{cases}
\]
est une bijection
\end{lemme}
\begin{proof} Exercice. Remarquer que si $m=1$, on obtient juste une bijection entre l'ensemble vide et lui-même.
\end{proof}


Les deux lemmes suivants établissent des résultats qui semblent \og évident\fg{} mais qui doivent être démontrés rigoureusement afin d'asseoir la définition de cardinal sur des bases solides.

\begin{lemme}
Soient $m,n\in \N$.
S'il existe une injection de $\llbracket 1,m\rrbracket$ dans $\llbracket 1,n\rrbracket$, alors $m\leq n$.
\end{lemme}
\begin{proof}
Pour $m$ entier, notons $A(m)$ l'assertion 
\begin{center}
$\forall n\in \N, \quad  (\text{il existe une injection } \llbracket 1,m\rrbracket \to \llbracket 1,n\rrbracket) \implies m\leq n$.
\end{center}
Montrons $\forall m, A(m)$ par récurrence, ce qui prouve la proposition.\\
\textbf{Initialisation.} $A(0)$ est vraie car pour tout $n$, $0\leq n$ est vraie donc l'implication dans $A(0)$ est vraie.\\
\textbf{Hérédité.} Soit $m\in\N$ et supposons $A(m)$. Montrons $A(m+1)$.

Soit $n\in \N$ et soit $f : \llbracket 1,m+1\rrbracket \to \llbracket 1,n\rrbracket$ une injection. Alors la restriction $f|_{\llbracket 1,m\rrbracket} : \llbracket 1,m\rrbracket \to \llbracket 1,n\rrbracket\setminus \{f(m+1)\}$ est également injective.

En composant avec une bijection $\phi : \llbracket 1,n\rrbracket\setminus \{f(m+1)\} \to \llbracket 1,n-1\rrbracket$ (par exemple celle fournie par le lemme), on obtient une injection $\phi\circ f|_{\llbracket 1,m\rrbracket} : \llbracket 1,m\rrbracket \to \llbracket 1,n-1\rrbracket$.

Par hypothèse de récurrence, on a donc $m\leq n-1$, et donc $m+1\leq n$, donc $A(m+1)$ est vraie.
\end{proof}

\begin{lemme}
Soient $m, n\in \N$. S'il existe une bijection entre $\llbracket 1,m\rrbracket$ et $\llbracket 1,n\rrbracket$, on a  $m=n$
\end{lemme}
\begin{proof}
Soit $f$ une telle bijection. Comme elle est injective, on a $m\leq n$.

Considérons alors la bijection réciproque $f^{-1} : \llbracket 1,n\rrbracket \to \llbracket 1,m\rrbracket$. Comme elle est également  injective, on a $n\leq m$. D'où $m=n$.
\end{proof}

\subsection{Ensembles finis et cardinal}

\begin{definition}
\index{ensemble fini}
Un ensemble $E$ est \emph{fini} s'il existe $n\in \N$ et une injection de $E$ dans $\llbracket 1,n\rrbracket$. Un ensemble qui n'est pas fini est \emph{infini}.
\end{definition}

\begin{remarque}
\begin{enumerate}[label=\alph*)]
\item On en déduit immédiatement qu'un ensemble qui s'injecte dans un ensemble fini est lui-même fini (la composée de deux injections est une injection).
\item À priori, l'entier $n$ de la définition n'est pas unique, car si $n$ convient, alors $n+1$ aussi.
\item (Zérologie) L'ensemble vide\index{ensemble vide} est fini : il existe une (unique) application de l'ensemble dans tout ensemble $F$, c'est celle dont le graphe est la partie vide de $\varnothing\times F$ (cette partie vérifie bien les conditions pour être un graphe de fonction). On l'appelle \og l'application vide\fg.
On vérifie ensuite que cette application est injective (en appliquant la définition).
\end{enumerate}
\end{remarque}


\begin{propdef}
\index{cardinal!d'un ensemble fini}
Soit $E$ un ensemble fini. Alors, il existe un unique $n\in \N$ tel que $E$ soit en bijection avec $\llbracket 1,n\rrbracket$.

Cet entier $n$ est appelé le \emph{cardinal} de $E$. Il est noté $\Card(A)$, ou $|A|$ ou $\sharp A$.
\end{propdef}
\begin{proof}
L'unicité découle des lemmes précédents.

Pour l'existence, Soit $n\in \N$ et soit $f : E\to \llbracket 1,n\rrbracket$ une injection.
Si $f(E) = \llbracket 1,n\rrbracket$, l'application est surjective donc bijective.
Sinon, il existe $a\in \llbracket 1,n\rrbracket \setminus f(E)$, donc en composant $f$ avec une injection $\phi \llbracket 1,n\rrbracket \setminus \{a\} \to \llbracket 1,n-1\rrbracket$, on obtient une injection $\phi\circ f : E \to \llbracket 1,n-1\rrbracket$.
On itère ce processus tant que l'application n'est pas bijective, ce qui finit par se produire puisque l'ensemble d'arrivée des injections diminue strictement à chaque étape.
\end{proof}




\begin{proposition}
\begin{enumerate}
\item Un ensemble en bijection avec un ensemble fini de cardinal $n$ est également fini de cardinal $n$.
\item L'ensemble vide est fini de cardinal zéro. Réciproquement, un ensemble fini de cardinal zéro est vide.
\end{enumerate}
\end{proposition}
\begin{proof}
\begin{enumerate}
\item Soit $f:A\to B$ une bijection. Si $B$ est fini de cardinal $n$, alors il existe une bijection $\phi : B\to \llbracket 1,n\rrbracket$, et donc  $\phi\circ f : A\to \llbracket 1,n\rrbracket$ est une bijection.
\item On a déjà vu qu'il existe une (unique) application entre $\varnothing$ et $\llbracket 1,0\rrbracket=\varnothing$ et qu'elle est injective. On peut vérifier qu'elle est surjective, toujours en appliquant la définition. Réciproquement, un ensemble de cardinal zéro est par définition en bijection avec $\llbracket 1,0\rrbracket = \varnothing$, donc est vide.
\end{enumerate}
\end{proof}

\begin{proposition}\index{cardinal!d'une union disjointe}
\begin{enumerate}
\item Si $A$ et $B$ sont disjoints et finis, alors $A\cup B$ est fini et $|A\cup B|=|A|+|B|$.
\item  Si $A$ est fini et $B \subseteq A$, alors $B$ est fini et $|B| \leq |A|$.
\item Si de plus $|B| = |A|$, alors $B=A$.
\item Si $A$ et $B$ sont finis, alors $|A\cup B| = |A| + |B| - |A\cap B|$.\index{cardinal!d'une union}
\end{enumerate}
\end{proposition}

\begin{proof}
\begin{enumerate}
\item Soient $n$ et $m$ des entiers et $f : A\to \llbracket 1,m\rrbracket$, $g : B\to \llbracket 1,n\rrbracket$ des bijections. L'application 
\[
\phi : A\cup B \to \llbracket 1,m+n\rrbracket, \quad x\mapsto
\begin{cases}
f(x) & \text{ si }x\in A\\
m+g(x) & \text{ si }x\in B
\end{cases}
\]
est bien définie, et c'est une bijection de $A\cup B$ dans $\llbracket 1,m+n\rrbracket$.
\item Si $A$ est fini, alors $B$ s'injecte dans un ensemble fini donc est fini. De même, la partie $A\setminus B$ de $A$ est également finie. On peut alors écrire $A$ comme l'union disjointe d'ensembles finis $A=B \cup (A\setminus B)$ et par ce qui précède, on a $|A|=|B|+|A\setminus B|$. On en déduit que $|B|\leq A$ et que s'il y a égalité, $A\setminus B$ est de cardinal $0$, donc vide, d'où $A=B$.
\item On a l'union disjointe $A = A\cup (B\setminus A)$  donc $|A\cup B| = |A|+|B\setminus A|$.
D'autre part, on a l'union disjointe $B = (B\cap A) \cup (B\setminus A)$, donc $|B| = |B\cap A|+|B\setminus A|$.
En remplaçant $|B\setminus A|$ par $|B|-|B\cap A|$ dans la première égalité, on obtient le résultat.
\end{enumerate}
\end{proof}

\subsection{Applications et ensembles finis}

\begin{proposition}
Soit $f : A\to B$ une application.
\begin{enumerate}
\item Si $B$ est fini, alors $|f(A)|\leq |B|$ et si de plus $|f(A)|= |B|$ alors $f$ est surjective.
\item Si $A$ est fini et $f$ est injective, alors $|f(A)|=|A|$.
\end{enumerate}
\end{proposition}
\begin{proof}
\begin{enumerate}
\item On a $f(A)\subseteq B$ donc $f(A)$ est fini et $|f(A)|\leq |B|$. S'il y a égalité des cardinaux, alors on a $f(A)=B$ ce qui signifie que $f$ est surjective.
\item Soit $g : A\to f(A)$ l'application déduite de $f$ en remplaçant le codomaine $B$ par $f(A)$. L'application $g$ est surjective par construction, que $A$ soit fini ou pas.

Si $f$ est injective, $g$ l'est également. On en déduit que $A$ et $f(A)$ sont en bijection. Si de plus $A$ est fini, ils ont donc le même cardinal.
\end{enumerate}
\end{proof}

\begin{proposition}
Soit $f : A\to B$ une application.
\begin{enumerate}
\item Si $B$ est fini et $f$ est injective, alors $A$ est fini et $|A|\leq |B|$.
\item Si $A$ est fini et $f$ est surjective, alors $B$ est fini et $|A|\geq |B|$.
\end{enumerate}
\end{proposition}
\begin{proof}
\begin{enumerate}
\item Si $B$ est fini, alors $A$ s'injecte dans un ensemble fini donc est fini. De plus, on a $|A|=|f(A)|\leq |B|$.
\item Soit $g:B\to A$ une \emph{section}\index{section} de $f$, c'est-à-dire une application  qui à $y\in B$ associe un antécédent quelconque de $y$. Par construction, on a $f\circ g=\Id_{B}$ donc $g$ est injective, et par le premier point $B$ est fini et $|B| \leq |A|$.
\end{enumerate}
\end{proof}

\begin{theoreme}[IMPORTANT]
Soient $A$ et $B$ finis \textbf{de même cardinal}, et soit $f : A\to B$. Alors, on a les équivalences suivantes:
\begin{center}
$f$ est injective $\iff$ $f$ est surjective $\iff$ $f$ est  bijective.
\end{center}
\end{theoreme}
\begin{proof}
Il suffit de prouver la première équivalence.

Sens $\implies$ : Si $f$ est injective, on a $|f(A)|=|A|=|B|$, et comme $f(A)\subseteq B$, l'égalité des cardinaux force $f(A)=B$ c'est-à-dire que $f$ est surjective.

Sens $\impliedby$, par contraposée : Si $f$ n'est pas injective, soient $x$ et $y$ distincts tels que $f(x)=f(y)$. Alors $f(A) = f(A\setminus \{y\})$, donc 
\[
|f(A)| \leq |A\setminus \{y\}| = |A|-1 = |B|-1,
\]
donc $f(A) \neq B$ et donc $f$ n'est pas surjective.
\end{proof}

Ce théorème est à retenir, il est indispensable dans tous les domaines des mathématiques. En particulier, il est crucial pour la théorie de la dimension des espaces vectoriels, au prochain semestre.

\begin{corollaire}
Soit $f : A\to B$ entre ensembles finis. Alors $f$ est injective si et seulement si $|f(A)|=|A|$.
\end{corollaire}
\begin{proof}On a déjà prouvé le sens \og seulement si\fg.

Si $|f(A)|=|A|$, alors la corestriction\index{corestriction} $g : A\to f(A), x\mapsto f(x)$ qui est par définition surjective, est également injective par le précédent théorème. Donc $f$ est injective.
\end{proof}



\subsection{Remarque sur les définitions équivalentes}

Il existe d'autres définitions (équivalentes) d'ensemble fini et de cardinal.
Par exemple, on aurait pu donner comme définition  : un ensemble $E$ est fini s'il existe $n\in \N$ et une surjection de $\llbracket 1,n\rrbracket$ dans $E$.

Dans ce cas, on aurait commencé par prouver le lemme suivant : \og s'il existe une surjection de $\llbracket 1,m\rrbracket$ dans $\llbracket 1,n\rrbracket$, alors $m\geq n$\fg, et l'ordre des résultats établis, ainsi que les preuves, auraient été différents.

\begin{exercice} \'Etablir tous les résultats du cours en prenant  cette définition pour base, au lieu de celle avec les injections.
\end{exercice}

On peut aussi définir les ensembles finis en utilisant $\N$.

\begin{exercice}
Soit $E$ un ensemble.
Prouver que $E$ est fini si et seulement si aucune application de $\N$ dans $E$ n'est injective.
\'Etablir une formulation équivalente avec des surjections.
\end{exercice}

L'essentiel est d'avoir une définition équivalente, mais surtout une définition maniable et efficace pour prouver les résultats du cours.

\section{Sommes et produits}

\begin{definition}[Notation $\sum$ et $\prod$]
\index{$\sum$}\index{$\prod$}
Soit $E$ un ensemble fini, et $f : E\to \C$ (ou autre codomaine que $\C$, l'essentiel étant de pouvoir sommer et multiplier).

On note $\sum_{x\in E} f(x)$ le nombre $0$ auquel on ajoute la somme des valeurs prises par $f(x)$ lorsque $x$ parcourt $E$.

On note $\prod_{x\in E} f(x)$ le nombre $1$ que l'on multiplie par le produit des valeurs prises par $f(x)$ lorsque $x$ parcourt $E$.
\end{definition}

\begin{remarque}[\og Un produit vide vaut $1$, une somme vide vaut $0$\fg]
Faire intervenir $0$ et $1$ sert à avoir les propriétés: $\sum_{x\in \varnothing} (...)=0$ et $\prod_{x\in \varnothing} (...) = 1$.
\end{remarque}

\begin{remarque} Dans la définition, si $E=A\cup B$ et $A\cap B = \varnothing$, alors $\sum_{x\in E}f(x) = \sum_{x\in A}f(x)+\sum_{x\in B}f(x)$.
\end{remarque}

Si $I$ est un ensemble fini et $(u_i)_{i\in I}$ est une famille de complexes, on note $\sum_{i\in I} u_i$ la somme de (zéro plus) tous ces complexes. 

\begin{definition}
La notation $\sum_{i=0}^n u_i$ signifie par définition $\sum_{i\in \llbracket 0,n\rrbracket} u_i$, et plus généralement,  $\sum_{i=a}^b u_i$ signifie par définition $\sum_{i\in \llbracket a,b\rrbracket} u_i$. 
\end{definition}

\begin{proposition}[Linéarité de la somme]
\index{linéarité de $\sum$}
Soit $A$ un ensemble fini, $f$, $g$ deux fonctions de $A$ dans $\C$ et $\lambda$, $\mu$ deux complexes. Alors :
\[
\sum_{x\in A}(\lambda f(x)+\mu g(x)) 
= \lambda\sum_{x\in A}f(x)+ \mu\sum_{x\in A} g(x).
\]
\end{proposition}

\begin{exemple}
Soit $n\in N$. On peut écrire 
\[
\sum_{k=0}^n (2k+3) 
= 2\sum_{k=0}^n k + 3\sum_{k=0}^n1 
= 2\frac{n(n+1)}{2}+3(n+1) 
= (n+1)(n+3).
\]
\end{exemple}

\begin{theoreme}[Théorème de Fubini pour les sommes finies]
\index{Fubini, théorème}
Soient $A$ et $B$ des ensembles finis et $f : A\times B \to \C$. On a les égalités:
\[
\sum_{x\in A}\left(\sum_{y\in B} f(x,y)\right)
=
\sum_{(x,y)\in A\times B} f(x,y)
=
\sum_{y\in B}\left(\sum_{x\in A} f(x,y)\right)
\]
\end{theoreme}
\begin{proof}
Par récurrence sur $|B|$. 

\noindent\textbf{Initialisation.} Si $|B|=0$, alors $B$ est vide, et les sommes indexées par $B$ sont nulles. D'autre part, $A\times B=\varnothing$, donc au final les trois sommes sont nulles.

\noindent\textbf{Hérédité.} Soit $n\in \N$, et supposons que pour toute partie $B$ de cardinal $n$ et tout ensemble fini $A$, les égalités soient vraies. Soit $B$ un ensemble de cardinal $n+1$. Écrivons $B = B'\cup \{b\}$, avec $b\not\in B'$, et $B'$ de cardinal $n$. On a alors une union disjointe $A\times B = A\times B' \cup A\times\{b\}$ et donc
\[ 
\sum_{(x,y)\in A\times B}f(x,y)
=
\sum_{(x,y)\in A\times B'}f(x,y) + \sum_{(x,y)\in A\times \{b\}}f(x,y)
=
\sum_{(x,y)\in A\times B'}f(x,y) + \sum_{x\in A} f(x,b).
\]
D'autre part, on a 
\begin{align*}
\sum_{x\in A}\left(\sum_{y\in B}f(x,y)\right)
&=
\sum_{x\in A}\left(\sum_{y\in B'}f(x,y)+\sum_{y=b}f(x,y)\right)\\
&=
\sum_{x\in A}\sum_{y\in B'}f(x,y) + \sum_{x\in A}f(x,b)\\
&=
\sum_{(x,y)\in A\times B'}f(x,y) + \sum_{x\in A}f(x,b)\text{ par hypothèse de récurrence}\\
&=
\sum_{(x,y)\in A\times B} f(x,y) \text{ par la rémarque précédente}.
\end{align*}
La deuxième égalité se prouve de la même façon en échangeant les rôles de $A$ et $B$.
\end{proof}

\begin{definition}[Factorielle]
\index{factorielle}
Soit $n\in \N$. La factorielle de $n$, notée $n!$, est le produit de tous les entiers strictement positifs et inférieurs à $n$. Autrement dit, $n! = \prod_{k \in \llbracket 1,n\rrbracket} k$. En particulier, si $n=0$, le produit est vide et donc $0!=1$ par définition d'un produit vide. 
\end{definition}


\section{Combinatoire}
\subsection{Principes élémentaires de combinatoire}

\begin{proposition}
\index{cardinal!d'un produit}
Soient $E$ et $F$ finis de cardinal $n$ et $p$. Alors $E\times F$ est de cardinal $np$.
\end{proposition}
\begin{proof}
L'application 
\[
\phi : \llbracket 1,n\rrbracket \times \llbracket 1,p\rrbracket\to \llbracket 1,np\rrbracket,\quad
(x,y)\mapsto (x-1)p+y
\]
est bijective.
\end{proof}
\begin{corollaire}
\index{cardinal!d'une puissance}
Soit $E$ un ensemble fini. Alors pour tout $k\in \N^*$, $|E^k|=|E|^k$.
\end{corollaire}
\begin{proof}
Par récurrence sur $k$. 
\textbf{Initialisation. }Lorsque $k=1$, on a bien  $|E^1|=|E|=|E|^1$.\\
\textbf{Hérédité.} Soit $k\in \N^*$ et supposons que $|E^k|=|E|^k$.
On a $|E^{k+1}| = |E^k\times E|=|E^k|\cdot |E|$ par la proposition précédente, et d'autre part par hypothèse de récurrence, on a $|E^k|=|E|^k$. Finalement,  $|E^{k+1}|=|E|^{k+1}$. 
\end{proof}

\begin{corollaire}\index{cardinal!d'un ensemble de fonctions}
Soient $E\neq \varnothing$ et $F$ des ensembles finis de cardinal  $n$ et $p$. L'ensemble $\mathcal F(E,F)$ des fonctions de $E$ dans $F$ est de cardinal $p^n$. 
\end{corollaire}
\begin{proof}
L'ensemble $\mathcal F(E,F)$ est en bijection avec $F^n$. (Une fonction correspond au choix d'un élément de $F$ pour chacun des $n$ éléments de $E$.) 
\end{proof}

\begin{remarque}
\begin{enumerate}
\item Si $E$ est vide, il existe une unique application de $E$ dans n'importe quel ensemble, fût-il vide : l'application vide\index{vide!application}\index{application!vide}. Donc $|\mathcal F(E,F)|=1$, ce qui permet d'étendre la formule lorsque $E$ est vide. Lorsque $E$ et $F$ sont tous deux vides, on pose $0^0=1$ (ou plutôt on démontre, si on a la \og bonne\fg{} définition de $a^b$) et la formule reste valable.
\item L'ensemble $\mathcal F(E,F)$ est également noté $F^E$. Avec cette notation, on a la formule $\left| F^E\right| = |F|^{|E|}$.
\end{enumerate}
\end{remarque}

\begin{proposition}
Soient $n, p\in \N$ avec $p\leq n$.
\begin{enumerate}
\item Il y a $n(n-1)(n-2)...(n-p+1)=\frac{n!}{(n-p)!}$ injections de $\llbracket 1,p\rrbracket$ dans $\llbracket 1,n\rrbracket$.
\item En particulier, il y a $n!$ bijections de $\llbracket 1,n\rrbracket$ dans lui-même.
\end{enumerate}
\end{proposition}

Une bijection de $\llbracket 1,n\rrbracket$ dans lui-même s'appelle une \emph{permutation} $\llbracket 1,n\rrbracket$.

\begin{proposition}
\index{cardinal!de $\mathcal P(E)$}
Soit $E$ un ensemble fini. L'ensemble $\mathcal P(E)$ des parties de $E$ est fini, de cardinal $2^{|E|}$.
\end{proposition}
\begin{proof}
Il y a une bijection entre $\mathcal P(E)$ et $\mathcal F(E,\{0,1\})$, donnée par $A\mapsto \mathbf{1}_A$, l'application qui à une partie $A$ associe sa fonction caractéristique. Or, on a $|\mathcal F(E,\{0,1\})| = |\{0,1\}|^{|E|} =2^{|E|} $.
\end{proof}

\subsection{Coefficients binomiaux}

\begin{definition}
\index{coefficients binomiaux}
Soit $n\in\N$, et $k\in \Z$. 

On note $\mathcal P_k(\llbracket 1,n\rrbracket)$ ou même $\mathcal P_k(n)$ l'ensemble des parties de $\llbracket 1,n\rrbracket$ qui sont de cardinal $k$.

On note $\binom{n}{k}$ le nombre de parties de $\llbracket 1,n\rrbracket$ de cardinal $k$, c'est-à-dire $\binom{n}{k}=|\mathcal P_k(n)|$.
\end{definition}

Remarque : si $k<0$ ou si $k>n$, $\binom{n}{k}=0$ car il n'y a aucune partie de $\llbracket 1,n\rrbracket$ de cardinal $k$.

\begin{proposition}
On a $\binom{n}{k} = \binom{n}{n-k}$.
\end{proposition}
\begin{proof}
L'application $\phi : \mathcal P_k(n) \to \mathcal P_{n-k}(n), \: A\mapsto  A^c$, est une bijection.
\end{proof}

\begin{proposition}
Soit $n\in\N$. On a $\sum_{k=0}^n \binom{n}{k} =2^n$.
\end{proposition}
\begin{proof}
On classe les parties de $P(\llbracket 1,n\rrbracket)$ suivant leur cardinal $k$, c'est-à-dire qu'on écrit  $\mathcal P(\llbracket 1,n\rrbracket) = \bigcup_{k=0}^n \mathcal P_k(n)$, l'union étant disjointe. En prenant le cardinal des deux membres on obtient $2^n=|P(\llbracket 1,n\rrbracket)| = \sum_{k=0}^n |\mathcal P_k(n)| = \sum_{k=0}^n \binom{n}{k}$.
\end{proof}

\begin{proposition}[Formule!de Pascal]
\index{formule de Pascal}
Soient $k,n\in \N$. Alors $\binom{n}{k}+\binom{n}{k+1} = \binom{n+1}{k+1}$.
\end{proposition}
\begin{proof}
On compte les parties à $k+1$ éléments de $\llbracket 1,n+1\rrbracket$ selon qu'elles contiennent ou non $n+1$. Celles qui ne contiennent pas $n+1$ sont en bijection avec $\mathcal P_{k+1}(n)$, et celles qui contiennent $n+1$ contiennent $k$ autres éléments de $\llbracket 1,n\rrbracket$ et sont donc en bijection avec $\mathcal P_k(n)$.
\end{proof}

\begin{proposition}[Formule du binôme de Newton]
\index{formule!du binôme de Newton}
Soient $a, b\in \C$ et $n\in \N$. Alors $(a+b)^n = \sum_{k=0}^n \binom{n}{k} a^k b^{n-k}$.
\end{proposition}
\begin{proof}
On développe le produit $(a+b)^n = (a+b)(a+b)...(a+b)$, ce qui donne $2^n$ termes tous de la forme $a^kb^{n-k}$, pour certains $0\leq k\leq n$.
À chaque façon de choisir un terme ($a$ ou $b$) dans chaque parenthèse, on associe une partie $X \in \llbracket 1,n\rrbracket$ qui correspond aux parenthèses où on choisit $a$ au lieu de $b$. On peut alors écrire:
\begin{align*}
(a+b)^n
&= \sum_{X \in \mathcal P(\llbracket 1,n\rrbracket)} a^{|X|}b^{n-|X|}\\
&= \sum_{k=0}^n \left(\sum_{X\in\mathcal P_k(n)} a^{|X|}b^{n-|X|}\right)\\
&= \sum_{k=0}^n \left( \sum_{X\in\mathcal P_k(n)} a^k b^{n-k}\right)\\
&= \sum_{k=0}^n a^kb^{n-k} \left|\mathcal P_k(n)\right|\\
&= \sum_{k=0}^n \binom{n}{k}a^kb^{n-k}\\
\end{align*}
\end{proof}

Remarque : il existe aussi une preuve par récurrence sur $n$ qui utilise la formule de Pascal, qui est moins parlante du point de vue combinatoire.

\begin{proposition}Soient $n, k\in \N$. Alors $\binom{n}{k} = \frac{n!}{k!(n-k)!}$.
\end{proposition}

La preuve qui suit est la version rigoureuse de la phrase \og pour compter le nombre de parties de cardinal $k$ de $\llbracket 1,n\rrbracket$, on compte le nombre de  listes ordonnées de cardinal $k$, c'est-à-dire $\frac{n!}{(n-k)!}$, puis on divise par le nombre de façons de désordonner ces listes, c'est-à-dire $k!$, puisqu'on ne s'occupe pas de l'ordre\fg. (La fin de la phrase est floue et non justifiée : pourquoi est-il correct de \og diviser\fg{} lorsqu'on ne \og s'occupe pas\fg{} de quelque chose ?)

\begin{proof} 

Soit $I$ l'ensemble des injections de $\llbracket 1,k\rrbracket$ dans $\llbracket 1,n\rrbracket$ (en bijection avec les listes ordonnées de $k$ éléments de $\llbracket 1,k\rrbracket$). Il est de cardinal $\frac{n!}{(n-k)!}$. Montrer le résultat revient à montrer que $|I| = k! |\mathcal P_k(n)|$.

Or, on peut écrire $|I| = \sum_{X\in \mathcal P_k(n)} \left| \{f\in I\:,\: f(\llbracket 1,k\rrbracket)=X\}\right|$. Mais si $X$ est de cardinal $k$, une injection $f\in I$ telle que  $f(\llbracket 1,k\rrbracket)=X$ est forcément une bijection, et on sait qu'il y a $k!$ telles bijections.

Donc, $|I| = \sum_{X\in \mathcal P_k(n)} k! = k! |\mathcal P_k(n)|$, ce qu'il fallait démontrer.
\end{proof}

\begin{remarque}
Le principe combinatoire général derrière cette preuve est le suivant : Si $\phi : A\to B$ entre ensembles finis, alors $|A| = \sum_{b\in B} \left|f^{-1}(\{b\})\right|$. Cela revient à compter le nombre d'éléments de $a$ en les classant d'abord selon leur image dans $B$, puis en sommant, pour chaque $b$, le nombre d'antécédents de $b$. Ici, ce principe serait appliqué avec $A=I$, $B=\mathcal P_k(n)$, et $\phi$ serait l'application qui à $f\in I$ associe son image, qui est un élément de $\mathcal P_k(n)$. Dans ce cas particulier, toutes les images réciproques ont le même cardinal $k!$.
\end{remarque}

\begin{remarque}
On peut trouver d'autres preuves des résultats présentés ici : des preuves par récurrence, ou bien des preuves calculatoires utilisant la formule $\binom{n}{k} = \frac{n!}{k!(n-k)!}$ qui doit alors être démontrée le plus tôt possible. Les preuves combinatoires sont souvent plus riches de sens.
\end{remarque}