\chapter{Ensembles}
\minitoc
\hyperlink{toc}{\retourTOC}

\section{Définitions (ou pas)}

En mathématiques, le sens du mot \emph{ensemble} est plus précis que celui donné par la langue française. Définir rigoureusement ce qu'est un ensemble (au sens mathématique du terme) est assez complexe et dans ce cours, on utilisera la définition intuitive suivante.

\begin{definition}(Ensemble, définition intuitive)

\begin{enumerate}
\item Un \emph{ensemble} $E$ est une collection d'objets.
\item Les objets dont est constitué la collection définissant $E$  sont les \emph{éléments} de $E$.
\item On dit que $x$ appartient à $E$ et on note $x\in E$ si $x$ est un élément de $E$. On note $x\not\in E$ dans le cas contraire.
\item Deux ensembles $E$ et $F$ sont dits \emph{égaux} s'ils ont les mêmes éléments. Dans ce cas on note $E=F$ (et $E\neq F$ dans le cas contraire).
\end{enumerate}
\end{definition}

(La définition donnée est insuffisante car en réalité, toutes les collections ne sont pas autorisées (pour éviter certains paradoxes). Mais la plupart de celles auxquelles on peut penser forment bien des ensembles au sens mathématique du terme). 

\begin{definition}(Manières de définir un ensemble)

\index{définition!par énumération}\index{définition!par compréhension}
\begin{enumerate}
\item Une définition \emph{par énumération (ou extension)} d'un ensemble $E$ est la donnée explicite de tous les éléments de l'ensemble, sous forme de liste entre accolades. Par exemple : $E = \left\{1,3,\pi,5,\sqrt2\right\}$.
\item Une définition \emph{par compréhension} d'un ensemble $E$ est la donnée d'une propriété qui caractérise les éléments de $E$ parmi un ensemble plus gros $F$. Par exemple : $E = \{x\in \R\:\mid\: x^2+x\leq 2\}$, qui se lit \og$E$ est l'ensemble des réels $x$ tels que $x^2+x\leq 2$\fg.
\item Un \emph{paramétrage} d'un ensemble $E$ est la donnée de $E$ comme l'ensemble des éléments d'une certaine forme, dépendant d'un paramètre, lorsque le paramètre varie dans un autre ensemble. Autrement dit, c'est une écriture de la forme $E = \{f(a)\:\mid\: a\in A\}$, ce qui se lit \og $E$ est l'ensemble des $f(a)$, lorsque le paramètre $a$ décrit l'ensemble $A$\fg. Par exemple, l'ensemble $\{2k+1\:\mid\: k\in \Z\}$ est l'ensemble des entiers de la forme $2k+1$, lorsque le paramètre $k$ décrit $\Z$. (C'est l'ensemble des nombres impairs.)
\end{enumerate}
\end{definition}



Attention, dans une définition par énumération, il n'y a pas de notion d'ordre, ni de multiplicité (un élément ne peut pas appartenir \og plusieurs fois\fg{} à un ensemble). Donc  $\left\{1,3,\pi,5,\sqrt2\right\} = \left\{\sqrt 2, 3,5,1,\pi\right\}=\left\{\sqrt 2,2, 2, 3,  3,5,1,1,\pi\right\}$.

\begin{definition}(Ensemble vide, singleton, paire)
\index{ensemble vide}\index{singleton}
\begin{enumerate}
\item L'\emph{ensemble vide} est l'unique ensemble ne contenant aucun élément. On le note $\varnothing$ (la notation $\{\:\}$ est également correcte mais n'est pas utilisée). Une assertion du type \og $\forall x\in \varnothing, \: A(x)$\fg{} est toujours vraie par définition. L'ensemble vide est inclus dans tout ensemble, puisque l'assertion \og $\forall x\in \varnothing, \: x\in E$\fg{} est toujours vraie.
\item Un \emph{singleton} est un ensemble contenant un unique élément. C'est donc un ensemble de la forme $\{x\}$.
\item Une \emph{paire} est un ensemble de la forme $\{a,b\}$. Si $a=b$, alors il s'agit d'un singleton, mais la plupart des cas, les éléments sont différents et $\{a,b\}$ est donc un ensemble contenant deux éléments distincts.
\end{enumerate}
\end{definition}

\begin{exemple} (Paramétrage d'un segment) 
Soient $a$ et $b$ deux réels. L'intervalle fermé de $\R$ délimité par $a$ et $b$ peut être paramétré par:
\[ \{a+t(b-a)\:\mid\: t\in [0,1]\}\]
C'est paramétrage standard de cet intervalle. Lorsque $t=0$, le réel $a+t(b-a)$ vaut $a$, lorsque $t=1$, le réel $a+t(b-a)$ vaut $b$, et lorsque $t$ varie entre $0$ et $1$, le réel $a+t(b-a)v$, varie entre $a$ et $b$.

On voit souvent $(1-t)a+tb$ au lieu de $a+t(b-a)$ : les deux écritures doivent être reconnues instantanément. Noter qu'on pourrait paramétrer l'intervalle de nombreuses autres manières, par exemple en \og partant de $b$\fg{} c'est-à-dire en échangeant $a$ et $b$ dans le paramétrage.
\end{exemple}

\begin{exemple} (Nombres complexes de module un) \index{cercle!unité de $\C$}
 On appelle \emph{ensemble des nombres complexes de module un} (ou encore \emph{cercle unité de $\C$}) et on note $\U$ l'ensemble 
\[ \U = \ensemble{ z\in \C}{|z|=1}.\]
 On montrera dans la suite que cet ensemble peut s'écrire sous forme paramétrée de la façon suivante :
\[ \U = \ensemble{ e^{it}}{t \in \R}.\]
\end{exemple}

% mettre après les inclusions pour pouvoir justifier les égalités par double inclusion ?
\begin{exemple}
Il est crucial de savoir jongler entre définitions paramétrées et définitions en compréhension. Ainsi, on a par exemple les égalités suivantes :
% en anticipant sur les ensembles produits
\begin{align*}
\left\{2k+1 \:\mid\: k\in \N\right\}  &= \left\{ n\in \N\:\mid\: 2\not|\: n\right\} \\
\left\{2k \:\mid\: k\in \N\right\} &=  \left\{ n\in \N\:\mid\: 2| n\right\}\\
\R_+ = \{u^2\:\mid\: u\in \R\} & = \{x\in \R\:\mid\: x\geq 0\}
\end{align*}
Noter que plusieurs valeurs du paramètre peuvent correspondre à un même élément de l'ensemble. L'important est que le paramétrage \og couvre\fg{} l'ensemble souhaité, il peut y avoir de la redondance.
\end{exemple}



\begin{remarque}
En anticipant sur le chapitre suivant, paramétrer un ensemble revient à l'écrire comme \emph{image d'une application} (voir la section\ref{subsec-retour-parametrage-familles}).
\end{remarque}




\section{Parties d'un ensemble}

\begin{definition}[Sous-ensemble / partie]
\index{ensemble!sous-ensemble}\index{ensemble!partie d'un ensemble}\index{partie!d'un ensemble}
Soient $E$ et $F$ des ensembles. On dit que $E$ est inclus dans $F$, ou que $E$ est un sous-ensemble, ou une partie de $F$ si tous les éléments de $E$ sont des éléments de $F$, autrement dit si 
\[
\forall x\in E, \: x\in F.
\]
Dans ce cas on note $E\subset F$ (notation la plus répandue) ou $E\subseteq F$ (dans ce cours, les deux notations sont synonymes et on privilégie la seconde). On note $E\not\subset F$ ou $E\nsubseteq F$ si $E$ n'est pas un sous-ensemble de $F$, et $E\subsetneq F$  si c'est un sous-ensemble \emph{strict} de $F$, c'est-à-dire $E\subseteq F$ et $E\neq F$.
\end{definition}

\begin{remarque}[Principe de double-inclusion] Si $E$ et $F$ sont des ensembles, alors 
$
E=F \iff \left(E\subseteq F \text{ et } F \subseteq E\right).
$
\end{remarque}


\begin{remarque}
Attention, les objets $x$ et $\{x\}$ sont différents !  Par exemple, $\varnothing$ et $\{\varnothing\}$ sont deux ensembles différents:  le premier est l'ensemble vide, alors que $\{\varnothing\}$ est un ensemble non vide : c'est un ensemble contenant un élément (l'ensemble vide).
\end{remarque}

\begin{axiomedef}[Ensemble des parties]
Soit $E$ un ensemble. La collection de toutes les parties de $E$ est un ensemble (au sens mathématique). On le note $\mathcal P(E)$. 

Ainsi, si $F$ est un ensemble, alors on a $F\in \mathcal P(E) \iff F\subseteq E$.
\end{axiomedef}

Remarque : cet ensemble n'est jamais vide car il contient toujours au moins $\varnothing$, qui est une partie de  tout ensemble $E$.

\begin{exemple}
Un singleton $\{a\}$ contient deux parties : la partie vide $\varnothing$ et la partie $\{a\}$. 

L'ensemble $\{1,2\}$ a pour ensemble de parties :  $\mathcal P(\{1,2\}) = \{\varnothing,\{1\},\{2\},\{1,2\}\}$.
\end{exemple}


\section{Union, intersection, complémentaire, produit}

\begin{definition}(Union, intersection, complémentaire)

Soient $A$ et $B$ des ensembles. Leur \emph{union}, notée $A\cup B$, est la collection formée par les éléments de $A$ et de $B$. Leur \emph{intersection}, notée $A\cap B$, est l'ensemble des éléments de $A$ qui sont également des éléments de $B$ (ou encore : l'ensemble des éléments de $B$ qui sont aussi des éléments de $A$). 

L'ensemble $A\setminus B$ est l'ensemble des éléments de $A$ qui n'appartiennent pas à $B$.

Si $A$ est un sous-ensemble d'un ensemble $E$, le complémentaire de $A$ dans $E$ est $E\setminus A$, on le note aussi $\complement A$ s'il n'y a pas d'ambiguïté sur l'ensemble $E$ dans lequel on prend le complémentaire de $A$.
\end{definition}


\begin{proposition}
Si $A$ et $B$ sont deux parties d'un ensemble $E$, on a :
\[
\complement(A\cup B) = \complement A \cap \complement B; \quad
\complement(A\cap B) = \complement A \cup \complement B.
\]
\end{proposition}
\begin{proof}
Soit $x\in E$. Alors:
\begin{align*}
x\in \complement(A\cup B) 
&\iff \text{non}(x \in A\cup B)\\
&\iff \text{non}(x \in A \text{ ou } x\in B)\\
&\iff \text{non}(x \in A) \text{ et } \text{non}(x\in B)\\
&\iff \left(x\in\complement A\right) \text{ et } \left(x\in\complement B\right)\\
&\iff x\in \complement A \cap \complement B.
\end{align*}
\end{proof}

\begin{definition}[Ensembles disjoints, unions disjointes]
Deux ensembles $A$ et $B$ sont \emph{disjoints} si leur intersection est vide : $A\cap B = \varnothing$. 
\end{definition}

\begin{attention}
Ne pas confondre des ensembles \emph{disjoints} et des ensembles \emph{distincts}. Les ensembles $\{1,2\}$ et $\{1,3\}$ sont distincts mais non disjoints.
\end{attention}


\begin{definition}[Produit cartésien]
\index{produit cartésien d'ensembles}
Soient $E$ et $F$ deux ensembles. Le \emph{produit cartésien}, ou simplement \emph{produit}, noté $E\times F$, est la collection de tous les couples  de la forme $(x,y)$, avec $x\in E$ et $y\in F$. Si $E=F$, on note $E^2$ au lieu de $E\times E$. 

On peut définir de même les produits finis du type $E_1\times E_2 \times ... E_n$ : leurs éléments sont les $n$-uplets de la forme $(x_1, x_2, ... x_n)$, avec $x_1\in E_1$, $x_2\in E_2$ etc. 
\end{definition}

\begin{remarque}
Le lecteur est sans doute déjà familier avec l'ensemble $\R^2$ (l'ensemble des couples de réels), qui est par exemple utilisé en géométrie plane lorsque l'on travaille en coordonnées (relatives à un repère fixé).
\end{remarque}

\begin{remarque}
$E\times F = \varnothing \iff \left( E=\varnothing\text{ ou }F=\varnothing\right)$.
\end{remarque}

\begin{definition}[Diagonale]
\index{diagonale!d'un produit de deux ensembles}
Soit $E$ un ensemble. La \emph{diagonale} de $E\times E$ est l'ensemble des couples d'éléments identiques, c'est-à-dire l'ensemble
\[
\Delta_E = \{(x,x)\:|\: x\in E\}
\]
\end{definition}



Pour finir, on donne quelques exemples de sous-ensembles de $\R^2$ et $\R^3$, sous forme paramétrée ou définis par des équations.

\begin{exemple}
\begin{enumerate}
\item L'ensemble $\mathcal D = \{(x,y)\in \R^2\:\mid\: x+6y-20=0\}$ est la droite de $\R^2$ d'équation cartésienne $x+6y-20=0$.

On peut aussi l'écrire sous forme paramétrée $\mathcal D = \{ (2,3)+t(6,-1)\:\mid\: t\in \R\}$ (droite passant par le point $(2,3)$ et dirigée par le vecteur $(6,-1)$).
\item \index{cercle!unité de $\R^2$} On appelle \emph{cercle unité de $\R^2$} et on note $\S^1$ le cercle centré sur l'origine et de rayon un, c'est-à-dire :
\[ \S^1=\ensemble{(x,y)\in \R^2}{x^2+y^2=1}.\]
Une de ses écritures paramétrées les plus courantes est  $\S^1=\ensemble{(\cos t,\sin t)}{t\in \R}$, mais on peut également écrire $\S^1=\ensemble{(\cos t,\sin t)}{t\in [-\pi,\pi]}$ ou même $\S^1=\ensemble{(\cos t,\sin t)}{t\in ]-\pi,\pi]}$.
\item (Un cylindre) \index{cylindre} L'ensemble $\ensemble{(x,y,z)\in \R^3}{x^2+y^2=1}$ peut être visualisé comme un  cylindre (\og creux, vertical et de longueur infinie\fg). On peut aussi l'écrire sous forme de produit $\S^1 \times \R$ (cercle par droite), qui est bien un sous-ensemble de $\R^2\times \R = \R^3$. Enfin, on peut aussi l'écrire sous la forme paramétrée $\ensemble{(\cos t,\sin t,z)}{(t,z)\in \R^2}$.
\item \index{sphère!unité de $\R^3$} On appelle \emph{sphère unité de $\R^3$} et on note\footnote{Le \og $2$\fg{} dans la notation $\S^2$ désigne la \og dimension\fg{}: la sphère est une surface, donc de \og dimension\fg{} deux, même si elle est plongée dans un espace ambiant de dimension trois.} $\S^2$ l'ensemble
\[ \S^2 = \ensemble{(x,y,z)\in \R^3}{x^2+y^2+z^2=1}\]
C'est l'ensemble des points de l'espace à distance $1$ de l'origine. Cet ensemble est un peu plus difficile à paramétrer. On utilise en général le paramétrage donné par les \emph{coordonnées sphériques}\index{coordonnées!sphériques}:
\[ \S^2 = \ensemble{(\cos\theta\cos\phi, \sin\theta\cos\phi, \sin\phi)}{\theta \in [-\pi,\pi], \: \phi\in[-\frac{\pi}{2},\frac{\pi}{2}]}\]
On démontrera dans l'exercice \ref{exo-coord-spheriques-sphere} que ceci est bien un paramétrage de la sphère. (Les notations sont ici celles des mathématiciens et géographes, et non des physiciens : $\theta$ est la longitude, et $\phi$ est la latitude usuelle. Les physiciens utilisent la colatitude et permutent les symboles $\theta$ et $\phi$.) On peut également paramétrer l'espace $\R^3$ par des \og coordonnées\fg{} sphériques, voir l'exercice \ref{exo-coord-spheriques-espace} qui explique également les guillemets.
\end{enumerate}
\end{exemple}



\section{Familles indexées}

\begin{definition}
Soient $E$ et $I$ des ensembles. Une famille d'éléments de $E$ indexée ou paramétrée par $I$ est un objet de la forme $(x_i)_{i\in I}$, c'est-à-dire la donnée, pour tout élément $i\in I$, d'un élément de $E$ noté $x_i$.
\end{definition}

\begin{exemple}
Une suite réelle $(u_n)_{n\in \N}$ est une famille de réels indexée par $\N$.
\end{exemple}

\begin{attention}
Ne pas confondre \emph{famille paramétrée} et \emph{ensemble paramétré}. Par exemple la famille $\left( (-1)^n \right)_{n\in \N}$ est bien différente de l'ensemble $\left\{(-1)^n\:\mid\:n\in \N\right\}$ : en effet, la famille est infinie, c'est la suite $(1,-1, 1,-1, \dots)$, alors que l'ensemble, lui est fini, il ne contient que deux éléments : $1$ et $-1$.
\end{attention}

L'ensemble $I$ qui sert à indexer la famille peut être fini ou infini, et s'il est infini, il peut être plus gros que $\N$ : il n'est pas nécessaire de pouvoir numéroter les éléments de la famille par des nombres : une famille peut être indexée par $\R$. Par exemple, si $a\in \R$, on peut définir la fonction $f_a : \R\to \R; x\mapsto e^{ax}$. Les fonctions $f_a$ forment la famille $(f_a)_{a\in \R}$.


\begin{definition}[Unions et intersections indexées par un ensemble]
Soit $E$ un ensemble, et $(E_i)_{i\in I}$ une famille de parties de $E$ indexée par un ensemble $I$.

Leur \emph{union}, notée $\bigcup_{i\in I} E_i$, est l'ensemble $\{x\in E\:\mid\: \exists i\in I, x\in E_i\}$.

Leur \emph{intersection}, notée $\bigcap_{i\in I} E_i$, est l'ensemble $\{x\in E\:\mid\: \forall i\in I, x\in E_i\}$.
\end{definition}


%\begin{definition}[Cylindre sur un ensemble]\index{cercle}\index{cylindre}
%Soit $E$ un ensemble. Le \emph{cylindre sur $E$} est l'ensemble 
%\[ \operatorname{Cyl}(E) = E\times[0,1].\]
%Dans le cas particulier où l'ensemble $E$ est un cercle, son cylindre peut effectivement être vu comme un cylindre (de hauteur un). Le cas général est une abstraction de cet exemple. Dans le cas général, le cylindre peut être vu comme \og l'épaississement de $E$ dans une direction\fg.
%\end{definition}

\section{Exercices d'approfondissement}

\begin{exercice}[Topologies]
Soit $E$ un ensemble et  $\mathcal O$ une partie de $\mathcal P(E)$. On dit que $\mathcal O$ est une \emph{topologie sur $E$} si les conditions suivantes sont vérifiées
\begin{itemize}
\item $\mathcal O$ est stable par intersection finie, autrement dit : pour tout $n\in \N^*$ et toute famille $U_1, \cdots U_n$ d'éléments de $\mathcal O$, on a $\bigcap_{i=1}^n U_i\in \mathcal O$.
\item $\mathcal O$ est stable par union quelconque, autrement dit : pour tout ensemble $I$ et toute famille $(U_i)_{i\in I}$ d'éléments de $\mathcal O$, $\bigcup_{i\in I}U_i \in \mathcal O$.
\item Les parties $\emptyset$ et $E$ sont des éléments de $\mathcal O$.
\end{itemize}

\begin{enumerate}
\item Montrer que $\mathcal O_1=\{\emptyset, E\}$ et $\mathcal O_2=\mathcal P(E)$ sont des topologies sur $E$.
\item Montrer que 
\[ \mathcal O_3 = \ensemble{U\in \mathcal P(E)}{U=\emptyset \text{ ou }{}^cU\text{ est fini}}
\]
est une topologie sur $E$.
\item Combien de topologies différentes y a-t-il si $E$ est l'ensemble vide ? S'il n'a qu'un seul élément ? Deux éléments ? Trois éléments ?
% les ensembles finis n'ont pas encore été vus en détail mais c'est faisable
\end{enumerate}
\end{exercice}

\begin{exercice}[Bornologies]
Dans l'ensemble $\R$, il existe une notion de \emph{partie bornée} : c'est une partie qui est incluse dans un segment du type $[-M,M]$, pour un certain $M$. Cet exercice montre comment généraliser cette notion de \emph{partie bornée} à un ensemble quelconque.

Soit $E$ un ensemble et  $\mathcal B$ une partie de $\mathcal P(E)$. On dit que $\mathcal B$ est une \emph{bornologie sur $E$} si les conditions suivantes sont vérifiées
\begin{itemize}
\item Si $A\in \mathcal B$ et $B\subseteq A$, alors $B\in \mathcal B$.
\item Si $A\in \mathcal B$ et $B \in \mathcal B$, alors $A\cup B\in \mathcal B$.
\item Pour tout $x\in E$, on a  $\{x\} \in \mathcal B$.
\end{itemize}
Les éléments de $\mathcal B$ sont dits \emph{$\mathcal B$-bornés}, ou simplement \emph{bornés} s'il n'y a pas d'ambiguïté sur la bornologie utilisée.

Dans la suite, on fixe un ensemble $E$.
\begin{enumerate}
\item Montrer que $\mathcal B_1=\{\emptyset, E\}$ est une bornologie de $E$. On l'appelle la \emph{bornologie triviale (ou : grossière)}.
\item Montrer que l'ensemble $\mathcal B_2$ des parties finies de $E$ est une bornologie de $E$. On l'appelle la \emph{bornologie discrète}.
\item Combien de bornologies différentes y a-t-il si $E$ est vide ? S'il contient (exactement) un élément ? Deux ? Trois ?
\item On suppose maintenant que $E=\R$. Soit $\mathcal B_3$ l'ensemble des parties $A\subseteq \R$ bornées au sens classique, autrement dit 
\[ A\in \mathcal B_3 \iff \exists M\in \R, \forall a\in A, |a|\leq M\]
Montrer que $\mathcal B_3$ est une bornologie. On l'appelle la \emph{bornologie usuelle sur $\R$}, et lorsqu'on parle de bornés de $\R$, il est implicite qu'on se réfère à cette bornologie (et non aux deux premières par exemple).
\item 
\end{enumerate}
\end{exercice}

\begin{exercice}[Algèbre de parties]
Soit $E$ un ensemble et  $\mathcal A$ une partie de $\mathcal P(E)$. On dit que $\mathcal A$ est une \emph{algèbre de parties $E$} si les conditions suivantes sont vérifiées:
\begin{itemize}
\item $\mathcal A$ n'est pas vide.
\item Si $X\in \mathcal A$, alors $E\setminus X$ aussi.
\item $\mathcal A$ est stable par union finie, autrement dit : pour tout $n\in \N^*$ et toute famille $U_1, \cdots U_n$ d'éléments de $\mathcal A$, on a $\bigcup_{i=1}^n U_i\in \mathcal A$.
\end{itemize}
\begin{enumerate}
\item Montrer que $\mathcal P(E)$ est une algèbre de parties de $E$.
\item Montrer  qu'une algèbre de parties de $E$ est stable par intersection finie.
\item Si $E$ a un, deux, ou trois éléments, combien d'algèbres de parties y a-t-il ?
\end{enumerate}
\end{exercice}
