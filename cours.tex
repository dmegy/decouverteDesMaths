\documentclass[10pt,a4paper]{article}
\usepackage{geometry}

\usepackage[utf8]{inputenc}
\usepackage[T1]{fontenc}
\usepackage[francais]{babel}

\usepackage{amsfonts,amsmath,amssymb,amsthm,fancybox,graphicx,tikz}
\usepackage{multicol,comment,variations}

\pagestyle{empty} % ne pas numéroter les pages
\setlength{\parindent}{0cm}

\newcommand{\N}{\mathbb{N}}
\newcommand{\Z}{\mathbb{Z}}
\newcommand{\Q}{\mathbb{Q}}
\newcommand{\R}{\mathbb{R}}
\newcommand{\C}{\mathbb{C}}
\newcommand{\K}{\mathbb{K}}

\theoremstyle{definition}
\newtheorem{exo}{Exercice}


\begin{document}

\title{UE ?? Découverte des maths\\
Résumé de cours}
\maketitle
\tableofcontents


\section{Logique et raisonnement}
\subsection{Introduction}


\section{Ensembles et applications}

\section{Nombres naturels et ensembles finis}

\section{Arithmétique}

\end{document}