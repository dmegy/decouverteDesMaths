\documentclass[11pt,a4paper]{book}

\usepackage{geometry}% gestion des marges etc
\usepackage[utf8]{inputenc} % caractères utf8 dans le fichier source
\usepackage[T1]{fontenc} % encodage en sortie
\usepackage[francais]{babel} % paramètres de langue : guillemets etc
\usepackage{amssymb,mathtools,amsthm}
\usepackage{stmaryrd,mathrsfs} % polices et symboles supplémentaires
\usepackage{fancybox,graphicx}
\usepackage{multicol,comment,variations,enumitem,datetime}

\usepackage{hyperref}
\hypersetup{
    colorlinks=true,       % false: boxed links; true: colored links
    linkcolor=[rgb]{0,0.2,0.6},          % color of internal links
    citecolor=[rgb]{0,0.2,0.6},        % color of links to bibliography
    filecolor=[rgb]{0,0.2,0.6},      % color of file links
    urlcolor=[rgb]{0.7,0.2,0.2}           % color of external links
}

\usepackage{pgf,pgfmath,tikz}
\usetikzlibrary{arrows}
\usetikzlibrary[patterns]
\tikzset{every picture/.style={execute at begin picture={
   \shorthandoff{:;!?};}
}}

\usepackage{makeidx}% imakeidx ne met pas les liens?


% - - - - - - -
% Spécifique à ce document :

\usepackage{fourier} % police de caractères : Adobe Utopia + Fourier math
\everymath{\displaystyle} % plus lisible mais casse l'homogénéité de la mise en page

\newcommand{\N}{\mathbb{N}}
\newcommand{\Z}{\mathbb{Z}}
\newcommand{\Q}{\mathbb{Q}}
\newcommand{\R}{\mathbb{R}}
\newcommand{\C}{\mathbb{C}}
\newcommand{\K}{\mathbb{K}}
\renewcommand{\P}{\mathbb{P}}
\newcommand{\U}{\mathbb{U}}
\DeclareMathOperator{\pgcd}{pgcd}
\DeclareMathOperator{\ppcm}{ppcm}
\DeclareMathOperator{\Id}{Id}
\DeclareMathOperator{\Card}{Card} % cardinal

% Environnements : 

\theoremstyle{definition}
\newtheorem{theoreme}{Th\'eor\`eme}[section]
\newtheorem{proposition}[theoreme]{Proposition}
\newtheorem{corollaire}[theoreme]{Corollaire}
\newtheorem{lemme}[theoreme]{Lemme}
\newenvironment{preuve}{{\bf Preuve. }}{$\Box$}
\newenvironment{red}{\begin{quote}\emph{Exemple de rédaction:}\\}{\end{quote}}

\newtheorem{propdef}[theoreme]{Proposition--Définition}
\newtheorem{definition}[theoreme]{D\'efinition}
\newtheorem{remarque}[theoreme]{Remarque}
\newtheorem{exercice}[theoreme]{Exercice}
\newtheorem{exemple}[theoreme]{Exemple}
\newtheorem{exemples}[theoreme]{Exemples}

\makeindex
% et faire un "makeindex"
\begin{document}

\title{Découverte des mathématiques\\
Résumé de cours\\
L1, année 2017-2018\\
{\small 
\'Etat d'avancement : chapitres 4.2, 4.4, 5 et 6 finis, en cours de relecture.\\ Le reste ne sera peut-être pas rédigé, se reporter au cours fait en classe}}
\date{Version du \today{} à \currenttime}
\maketitle


\tableofcontents





\chapter{Logique et raisonnement}
\minitoc
\hyperlink{toc}{\retourTOC}

\section{Préambule : vocabulaire et ensembles classiques}
Afin de pouvoir illustrer les notions de ce chapitre dans le contexte des mathématiques, on part du principe qu'un certain nombre de choses sont connues:

\begin{enumerate}
\item Les ensembles classiques :  $\N$, $\Z$, $\Q$, $\R$, $\C$, les mêmes privés de zéro : $\N^*$, ..., $\C^*$. Les lois de composition classiques sur ces ensembles : addition, multiplication, avec leurs règles de calcul.
\item L'égalité dans ces ensembles, la relation d'ordre dans $\R$ : $x< y$ se lit \og$x$ est strictement inférieur à $y$\fg, $x\leq y$ se lit \og $x$ est inférieur à $y$\fg{} (on précise parfois \og inférieur ou égal\fg{} même si sans précision, une inégalité est toujours prise au sens large).
\item La relation de divisibilité dans $\Z$ : la suite de symboles $a|b$ se lit \og $a$ divise $b$\fg.
\item Les notations d'appartenance d'un élément à un ensemble:on écrit $x \in E$ pour dire que $x$ est un élément de l'ensemble $E$ et $x\not\in E$ sinon. Par exemple, $\frac23 \in \Q$, mais $\sqrt{2} \not\in\Q$ (cela sera prouvé dans la suite du chapitre). 
\item Les notations $\R_+$, $\Q_-$, etc pour des contraintes de signe (au sens large : $0$ appartient à $\Q_+$ par exemple). On peut combiner : l'ensemble $\R_+^*$ est l'ensemble des réels strictement positifs. 
\item Les fonctions classiques, comme la racine carrée et la valeur absolue.
\end{enumerate}

Tout ceci sera revu en détail de toutes façons.


\section{Propositions / assertions logiques}

\begin{definition}
\index{assertion logique}
Une proposition (logique), ou assertion (logique) est une phrase à laquelle on peut attribuer le statut \og VRAI\fg{} ou \og FAUX \fg{}. La phrase peut en outre comporter des symboles qui désignent des objets mathématiques (comme des chiffres) et d'autres symboles qui désignent des relations mathématiques entre objets (par exemple l'égalité, inégalité, divisibilité, appartenance à un ensemble...)
\end{definition}

Par exemple \og$2+2=3$\fg{} et \og $2+3=5$\fg{} sont des propositions (la première est fausse, la seconde vraie). La phrase \og le nombre complexe $i$ est positif\fg{} (ou encore \og quelle heure est-il?\fg) ne sont pas des propositions, on ne peut pas leur affecter de statut : la première n'a pas de sens (un nombre complexe n'a pas de signe), la seconde a un sens mais on ne peut pas lui affecter de statut VRAI ou FAUX.\\
 
 
\paragraph{Variables, paramètres, assertions ouvertes et fermées}
\index{paramètre}\index{variable!muette/liée/quantifiée}\index{variable!libre}

Une proposition peut dépendre d'un ou plusieurs paramètres, ou variables. Un paramètre est un symbole qui désigne un élément (explicité ou pas) d'un ensemble.

Les symboles nouveau doivent \textbf{toujours} être définis (ou déclarés) avec leur type (l'ensemble auquel appartient l'objet), de sorte à pouvoir être sûr du fait que la phrase est bien une assertion, c'est-à-dire possède un statut VRAI ou FAUX.

Par exemple : Dans \og$x\geq 0$\fg, le symbole $x$ n'est pas défini, on ne peut pas être sûr que la phrase ait un sens. Si $x$ était un nombre complexe par exemple, la phrase n'aurait aucun sens. Le symbole $x$ pourrait désigner beaucoup d'autres objets mathématiques, par exemple ... un cercle, les coordonnées d'un point du plan, auquel cas la phrase n'a pas non plus de sens.

 D'autre part si $x$ est un nombre naturel, la phrase a un sens mais elle est trivialement vraie car tous les naturels sont positifs. Tout ceci montre qu'il est crucial de déclarer clairement les variables et leur type \emph{avant} de commencer à les utiliser.

\index{assertion!fermée}\index{assertion!ouverte}
Une assertion dont le statut ne dépend pas de la valeur d'un paramètre est dite \emph{fermée}. Dans le cas contraire, elle est dite \emph{ouverte}. Par exemple, \og $2+2=5$\fg{} est une assertion fermée, et $x^2+x+1\leq 5$, qui dépend du paramètre $x\in \R$, est une assertion ouverte, son statut dépend de la valeur de $x$ (par exemple, elle est fausse pour $x=2$ et vraie pour $x=1$).

\index{déclaration d'un objet}
On déclare des objets à l'aide de la locution \og Soit\fg. La phrase \og Soit $x$ un réel.\fg{} déclare un réel, que l'on note $x$. La phrase \og Soit $k\in \Z$.\fg{} déclare un entier relatif (l'usage de $\in$ comme abréviation pour \og appartenant à\fg est toléré dans ce cas-là, même si en général on interdit d'utiliser les symboles mathématiques comme des abréviations).

La phrase \og Soit $x$.\fg{} n'est pas une déclaration correcte  d'objet mathématique : on doit préciser le type.

Si on précise que $x$ est un nombre réel, \og$x\geq 0$\fg devient une assertion mathématique bien formée. Le statut de cette proposition dépend de la valeur de $x$ : elle est vraie si $x\in \R_+$, elle est fausse si $x\in \R_-^*$. Le fait ne pas pouvoir connaître explicitement le statut n'est pas un problème. De fait que lorsqu'on déclare un réel $x$, on ne sait pas a priori lequel c'est.



\section{Construction de propositions}

Considérons deux propositions $A$ et $B$. Dans les exemples qui suivent, sauf précision, $x$ est un nombre réel.

\paragraph{Conjonction : \og A et B\fg} \index{assertion!conjonction}

La proposition \og A et B \fg{} est vraie si A et B sont vraies. Elle est fausse dès que l'une au moins des deux est fausse.

Exemple : \og$x>2$ et $x<5$\fg{} est vraie si $x\in]2,5[$. Elle est fausse sinon.

\paragraph{Disjonction : \og A ou B\fg} \index{assertion!disjonction}

La proposition  \og A ou B\fg{} est vraie dès que l'une des deux est vraie, elle est fausse si les deux sont fausses. Lorsqu'on affirme que \og A ou B\fg est vraie, l'un n'exclut pas l'autre.

Exemple : \og $x>2$ ou $x<5$\fg{ est vraie pour tout nombre réel $x$.

\paragraph{Négation : \og non A\fg}\index{assertion!négation}

 La proposition \og non A\fg{} est vraie si A est fausse et inversement.

\paragraph{Implication logique : \og$A \Rightarrow B$\fg}\index{implication}

La proposition \og $A \Rightarrow B$\fg signifie par définition \og B ou non-A\fg. Elle est vraie si $A$ est fausse ou si $B$ est vraie.\\
Exemples :  $2+2=4 \Rightarrow 2\times2 = 4$ est vraie. $2+2=5 \Rightarrow 2\times2 = 4$ est vraie. $2+2=5 \Rightarrow 2\times2 = 5$ est vraie. $2+2=4 \Rightarrow 2\times2 = 5$ est fausse. Autre exemple:  si $x$ est un nombre réel, la proposition $x>3 \Rightarrow x>4$  est vraie pour $x\leq 3$ ou pour $x>4$. Elle est fausse si $3<x\leq 4$.

Attention: le symbole $\Rightarrow$ n'est en aucun cas une abréviation pour \og donc\fg. La proposition $A \Rightarrow B$ ne veut pas dire \og  A est vraie donc B est vraie\fg !

\paragraph{\'Equivalence logique : \og $A \Leftrightarrow B$\fg}
\index{équivalence (logique)}

La proposition \og$A \Leftrightarrow B$\fg{} signifie par définition \og $A \Rightarrow B$ et $B \Rightarrow A$\fg{}. Elle est  vraie si $A$ et $B$ ont même statut, que ce soit vrai ou faux. Elle est fausse si A et B ont des statuts différents.\\
Exemples : $2+2=5 \Leftrightarrow 2\times 3 = 7$ est vraie. $1>0 \Leftrightarrow 2+2=4$ est vraie. Si $x$ est un nombre réel, la proposition $x>3 \Leftrightarrow x<4$ est vraie pour $x\in]3,4[$. Elle est fausse sinon.

\section{Quantificateurs}\index{quantificateur}
Soit $A(x)$ une proposition dépendant d'un paramètre $x$ appartenant à un ensemble $E$ (exemple :  \og $x>3$\fg, où $x \in \Z$).

\paragraph{Quantificateur universel : $\forall$ (quelque soit/pour tout)}\index{quantificateur!universel}

La proposition \og$\forall x\in E,\:A(x)$\fg{} se lit \og pour tout $x$ dans $E$, $A(x)$\fg. Elle est vraie si $A(x)$ est vraie pour toutes les valeurs que peut prendre $x$ dans l'ensemble $E$. Elle est fausse dès qu'il existe une valeur spéciale de $x$ pour laquelle $A(x)$ est fausse.
Attention, contrairement à la proposition $A(x)$, la proposition $\forall x\in E,\:A(x)$ est une proposition qui ne dépend d'aucun paramètre : elle est soit vraie soit fausse : on dit que $x$ est une variable muette, ou interne.
Exemples :  $\forall x\in R,\: x^2>1$ est fausse. La proposition $\forall x\in \Z^*,\: x^2\geq 1$ est vraie.

\paragraph{Quantificateur existentiel : $\exists$ (il existe)}\index{quantificateur!existentiel}


La proposition \og$\exists x\in E\slash A(x)$\fg{} se lit \og il existe $x$ dans $E$ tel que $A(x)$\fg. Elle est vraie s'il y a une valeur de $x$ dans l'ensemble $E$ telle que $A(x)$ soit vraie. Elle est fausse si $A(x)$ est fausse pour toutes les valeurs de $x$.



\begin{theoreme} On a les équivalences suivantes:\\
non (non A) $\Leftrightarrow$ A.\\
non ( A ou B ) $\Leftrightarrow$ (non A) et (non B).\\
non ( A et B ) $\Leftrightarrow$ (non A) ou (non B).\\
$(\forall x\in E,\; A(x))\Leftrightarrow (\forall y\in E,\; A(y))$.\\
$\text{non}(\forall x\in E,\: A(x)) \Leftrightarrow \exists x\in E,\: \text{non}(A(x)))$.\\
$\text{non}(\exists x\in E,\: A(x)) \Leftrightarrow \forall x\in E,\: \text{non}(A(x)))$.
\end{theoreme}

Démonstration : voir TD.

\section{Méthodes de démonstration}


\paragraph{Démonstration directe}$ $\\
\\
Exemple : soit $n \in Z$; montrer que \og $n$ pair $\Rightarrow n^2$ pair\fg.\\
\begin{red} Si $n$ est pair, il existe $k \in Z$ tel que $n = 2k$. Alors, $n^2 = 4k^2 = 2(2k^2)$ est pair. (et si $n$ est impair, l'implication est vraie par définition, il n'y a rien à prouver).\end{red}

\paragraph{Démonstration par contraposée}
\index{démonstration!par contraposée}

Principe : $(A\Rightarrow B)$ est équivalente à $(\text{non-}B \Rightarrow \text{non-}A)$.\\
Preuve du principe: $(\text{non-}B \Rightarrow \text{non-}A)\Leftrightarrow (\text{non-}A \text{ ou } non-non-B)\Leftrightarrow (B\text{ ou }non-A)\Box$.\\
Exemple d'application : soit $n\in \Z$; montrer que $n^2$ pair $\Rightarrow n$ pair.
\begin{red}
On va montrer la contraposée, autrement dit on va montrer 
\og $n$ impair $\Rightarrow n^2$ impair\fg, qui est équivalente, mais plus facile à montrer. Supposons donc $n$ impair. Alors il existe $k \in\Z$ tel que $n = 2k+1$. Mais alors $n^2 = 4k^2+4k+1 = 2(2k^2+2k)+1$ est impair.\\
En combinant avec le résultat précédent, on a donc prouvé : \og$n^2$ pair $\Leftrightarrow n$ pair\fg
\end{red}

\paragraph{Démonstration par l'absurde}
\index{démonstration!par l'absurde}

Principe : Si $F$ désigne n'importe quelle proposition fausse, on a $A \Leftrightarrow (\text{ non-}A \Rightarrow F)$.\\
Preuve du principe: $(\text{ non-}A \Rightarrow F) \Leftrightarrow (F\text{ ou non-non-}A)\Leftrightarrow A$.\\
 Donc pour montrer $A$, il suffit de supposer $A$ faux et d'en déduire une contradiction (c'est-à-dire n'importe quelle proposition fausse).\\
Exemple d'application : Montrer que $\sqrt{2}$ n'est pas rationnel.
\begin{red}
Par l'absurde, supposons $\sqrt{2}\in\Q$. Alors il existe deux entiers $p$ et $q$ premiers entre eux tels que $\sqrt{2} = p/q$. Donc $p = q\sqrt{2}$ et donc $p^2 = 2 q^2$, donc $p^2$ est pair, donc par l'exemple précédent $p$ est pair. Donc il existe $k\in Z$ tel que $p = 2k$, d'où en remplacant $4k^2 = 2 q^2$, donc en simplifiant $q^2$ est pair donc $q$ est pair. Donc $p$ et $q$ sont tous les deux pairs, contradiction car ils sont premiers entre eux. Finalement cette contradiction prouve que $\sqrt{2} \not\in \Q$.
\end{red}

\paragraph{Démonstrations de propositions avec quantificateur universel}$ $\\

Pour démontrer $\forall x\in E,\:A(x)$, on écrit:\\
\og Soit $x\in E$ un élément quelconque\fg.\\
Puis, on démontre $A(x)$.\\
Puis, pour conclure, on écrit : \og $x$ étant pris quelconque dans $E$, la propriété est bien démontrée\fg.\\


\begin{exemple}
Montrer que $\forall x\in \R, x^2+x+1> 0$\fg.
\begin{red}
\begin{tabular}{lr}
Soit $x\in \R$. & (déclaration de $x$)\\
On a $x^2+x+1 = (x+\frac12)^2+\frac34$. & (Début preuve de $A(x)$)\\
Comme un carré est toujours positif, on a $(x+\frac12)^2 \geq 0$ & \\
 et donc  $x^2+x+1>0$. & (fin preuve de $A(x)$)\\
Ceci montre donc bien $\forall x\in \R, x^2+x+1>0$ & (Conclusion)\\
\end{tabular}
\end{red}
\end{exemple}


\begin{exemple}\index{démonstration!par disjonction de cas}
Démontrer que $\forall x\in\R,\; x^2+\cos(x)>0$.
\end{exemple}
\begin{red}
Soit $x\in\R$.\\
On distingue deux cas possibles suivant la valeur de $x$.\\
Si $0\leq |x|< \pi/2$, alors $x^2\geq 0$ et $\cos(x)> 0$ donc $x^2+\cos(x)>0$.\\
Si  $\pi/2\leq|x|$, alors $x^2+\cos(x)\geq \pi^2/4-1>0$.\\
Comme $x$ est quelconque, on a bien montré la propriété pour tout $x\in\R$.\end{red}

\paragraph{Cas particulier : démonstrations par récurrence}
\index{démonstration!par récurrence}

Dans le cas particulier où le quantificateur universel porte sur l'ensemble $\N$, on peut utiliser une méthode de preuve spécifique, la récurrence. Cette méthode de démonstration s'appuie sur le fait que toute partie non vide de $\N$ admet un plus petit élément (ce qui est faux pour la plupart des autres ensembles classiques). Il suffit alors de montrer d'une part que $A(0)$ est vraie, ce qui est généralement facile, puis de montrer que pour tout $n \in\N$, on a $A(n) \Rightarrow A(n+1)$. La première étape est cruciale et le raisonnement est faux si on l'omet.


\paragraph{Démonstrations de propositions avec quantificateur existentiel}\index{démonstration!de l'existence d'un objet}

Pour démontrer \og$\exists x\in E\slash A(x)$, il faut soit construire un élément $x$ tel que $A(x)$ soit vrai, soit utiliser un théorème qui affirme dans sa conclusion l'existence d'un tel objet (ou qui affirme l'existence d'un objet à partir duquel on peut obtenir l'existence de $x$).

\begin{exemple}
Soit $f$ une fonction croissante de $[0,1]$ dans $\R$. Montrer que $f$ est majorée, autrement dit montrer que $(\exists M\in\R\slash (\forall x\in[0,1],\;f(x)\leq M))$.
\begin{red}
Posons $M = f(1)$. On a bien $\forall x\in[0,1],\;f(x)\leq f(1)= M$, car $f$ est croissante.
\end{red}
\end{exemple}

\begin{exemple}\index{démonstration!par principe du tiers-exclus}
Montrer qu'il existe deux irrationnels $a$ et $b$ tels que $a^b$ soit rationnel.
\begin{red}
Considérons le nombre réel $\sqrt{2}^{\sqrt{2}}$. Il est soit rationnel, soit irrationnel. Dans le premier cas, il suffit de poser $a=b=\sqrt{2}$ (irrationnels, voir exemple plus haut) et la preuve est terminée. Dans le second cas, il suffit de poser $a=\sqrt{2}^{\sqrt{2}}$ (qui est supposé irrationnel) et $b=\sqrt{2}$ On a alors $a^b = \left(\sqrt{2}^{\sqrt{2}}\right)^{\sqrt{2}} = \sqrt{2}^2 = 2 \in \Q$.
\end{red}
\end{exemple}

Ce deuxième exemple montre que parfois, on n'a pas besoin de construire explicitement l'objet, seulement de montrer que ça existe, soit par l'analyse de cas de figure complémentaires, soit en utilisant un théorème qui affirme l'existence d'un certain objet sans forcément l'expliciter. Cela dit, la plupart du temps, il faut construire l'objet.


\section{Résolution des équations}
\index{équation}

Soit $A(x)$ une proposition portant sur $x\in E$. Résoudre $A(x)$, c'est déterminer exactement l'ensemble des $x$ tels que $A(x)$ soit vrai. Cet ensemble est un sous-ensemble de $E$, on l'appelle l'ensemble des solutions. Il peut parfois être vide (aucune solution) ou égal à $E$ (équation triviale).


\paragraph{Méthode par équivalence}$ $\\
$A(x) \Leftrightarrow B(x) \Leftrightarrow ... \Leftrightarrow C(x)$ et on sait facilement résoudre $C(x)$. Cette méthode ne  s'applique que rarement, essentiellement qu'aux (systèmes d') équations linéaires.

\begin{exemple}
Résoudre $2x+3=5$, d'inconnue $x\in \R$.
\begin{red}
Soit $x\in \R$. On a 
\begin{align*}
2x+3=8
&\iff 2x=5\\
&\iff x=5/2.
\end{align*}
\end{red}
\end{exemple}

\paragraph{Méthode par conditions nécessaires et suffisantes}
\index{démonstration!par analyse-synthèse / par conditions nécessaires et suffisantes}

Lorsque $A(x) \Rightarrow B(x)$, on dit que $B(x)$ est une condition nécessaire à $A(x)$, et $A(x)$ est une condition suffisante pour $B(x)$.\\
Dans la pratique, on écrit $A(x)\Rightarrow B(x) \Rightarrow ... x\in\Omega$. Ensuite, parmi les éléments de $\Omega$, on détermine ceux qui sont solution.\\
\\
Exemple : résoudre $|x-1|=2x+3$, d'inconnue $x\in \R$.\\
\begin{red}
Soit $x\in\R$. On a la chaîne d'implications $|x-1|=2x+3 \Rightarrow |x-1|^2=(2x+3)^2 \Leftrightarrow x^2-2x+1=4x^2+12x+9 \Leftrightarrow 3x^2+14x+8=0 \Leftrightarrow (x\in \{-4;-2/3\})$. Réciproquement, on vérifie que $-4$ n'est pas solution mais que $-2/3$ est solution. Finalement, l'équation a une unique solution, $-2/3$.
\end{red}



\chapter{Ensembles}
\minitoc
\hyperlink{toc}{\retourTOC}

\section{Définitions (ou pas)}

En mathématiques, le sens du mot \emph{ensemble} est plus précis que celui donné par la langue française. Définir rigoureusement ce qu'est un ensemble (au sens mathématique du terme) est assez complexe et dans ce cours, on utilisera la définition intuitive suivante.

\begin{definition}(Ensemble, définition intuitive)

\begin{enumerate}
\item Un \emph{ensemble} $E$ est une collection d'objets.
\item Les objets dont est constitué la collection définissant $E$  sont les \emph{éléments} de $E$.
\item On dit que $x$ appartient à $E$ et on note $x\in E$ si $x$ est un élément de $E$. On note $x\not\in E$ dans le cas contraire.
\item Deux ensembles $E$ et $F$ sont dits \emph{égaux} s'ils ont les mêmes éléments. Dans ce cas on note $E=F$ (et $E\neq F$ dans le cas contraire).
\end{enumerate}
\end{definition}

(La définition donnée est insuffisante car en réalité, toutes les collections ne sont pas autorisées (pour éviter certains paradoxes). Mais la plupart de celles auxquelles on peut penser forment bien des ensembles au sens mathématique du terme). 

\begin{definition}(Manières de définir un ensemble)

\index{définition!par énumération}\index{définition!par compréhension}
\begin{enumerate}
\item Une définition \emph{par énumération} d'un ensemble $E$ est la donnée explicite de tous les éléments de l'ensemble, sous forme de liste entre accolades. Par exemple : $E = \left\{1,3,\pi,5,\sqrt2\right\}$.
\item Une définition \emph{par compréhension} d'un ensemble $E$ est la donnée d'une propriété qui caractérise les éléments de $E$ parmi un ensemble plus gros $F$. Par exemple : $E = \{x\in \R\:\mid\: x^2+x\leq 2\}$, qui se lit \og$E$ est l'ensemble des réels $x$ tels que $x^2+x\leq 2$\fg.
\end{enumerate}
\end{definition}

Attention, dans une définition par énumération, il n'y a pas de notion d'ordre, ni de multiplicité (un élément ne peut pas appartenir \og plusieurs fois\fg{} à un ensemble). Donc  $\left\{1,3,\pi,5,\sqrt2\right\} = \left\{\sqrt 2, 3,5,1,\pi\right\}=\left\{\sqrt 2,2, 2, 3,  3,5,1,1,\pi\right\}$.


\begin{definition}(Ensemble vide, singleton, paire)
\index{ensemble vide}\index{singleton}
\begin{enumerate}
\item L'\emph{ensemble vide} est l'unique ensemble ne contenant aucun élément. On le note $\varnothing$ (la notation $\{\:\}$ est également correcte mais n'est pas utilisée). Une assertion du type \og $\forall x\in \varnothing, \: A(x)$\fg{} est toujours vraie par définition. L'ensemble vide est inclus dans tout ensemble, puisque l'assertion \og $\forall x\in \varnothing, \: x\in E$\fg{} est toujours vraie.
\item Un \emph{singleton} est un ensemble contenant un unique élément. C'est donc un ensemble de la forme $\{x\}$.
\item Une \emph{paire} est un ensemble de la forme $\{a,b\}$. Si $a=b$, alors il s'agit d'un singleton, mais la plupart des cas, les éléments sont différents et $\{a,b\}$ est donc un ensemble contenant deux éléments distincts.
\end{enumerate}
\end{definition}

\section{Parties d'un ensemble}

\begin{definition}[Sous-ensemble / partie]
\index{ensemble!sous-ensemble}\index{ensemble!partie d'un ensemble}
Soient $E$ et $F$ des ensembles. On dit que $E$ est inclus dans $F$, ou que $E$ est un sous-ensemble, ou une partie de $F$ si tous les éléments de $E$ sont des éléments de $F$, autrement dit si 
\[
\forall x\in E, \: x\in F.
\]
Dans ce cas on note $E\subset F$ (notation la plus répandue) ou $E\subseteq F$ (dans ce cours, les deux notations sont synonymes et on privilégie la seconde). On note $E\not\subset F$ ou $E\nsubseteq F$ si $E$ n'est pas un sous-ensemble de $F$, et $E\subsetneq F$  si c'est un sous-ensemble \emph{strict} de $F$, c'est-à-dire $E\subseteq F$ et $E\neq F$.
\end{definition}

\begin{remarque}[Principe de double-inclusion] Si $E$ et $F$ sont des ensembles, alors 
$
E=F \iff \left(E\subseteq F \text{ et } F \subseteq E\right).
$
\end{remarque}


\begin{remarque}
Attention, les objets $x$ et $\{x\}$ sont différents ! L'un est l'objet $x$, l'autre est un ensemble contenant un unique élément : $x$. Par exemple, $\varnothing$ et $\{\varnothing\}$ sont deux choses différentes le premier est l'ensemble vide, alors que $\{\varnothing\}$ est un ensemble non vide : c'est un ensemble contenant un élément (l'ensemble vide).
\end{remarque}

\begin{axiomedef}[Ensemble des parties]
Soit $E$ un ensemble. La collection de toutes les parties de $E$ est un ensemble (au sens mathématique). On le note $\mathcal P(E)$. 

Ainsi, si $F$ est un ensemble, alors on a $F\in \mathcal P(E) \iff F\subseteq E$.
\end{axiomedef}

Remarque : cet ensemble n'est jamais vide car il contient toujours au moins $\varnothing$, qui est une partie de  tout ensemble $E$.

\begin{exemple}Un singleton $\{a\}$ contient deux parties : la partie vide $\varnothing$ et la partie $\{a\}$. 

L'ensemble $\{1,2\}$ a pour ensemble de parties :  $\mathcal P(\{1,2\}) = \{\varnothing,\{1\},\{2\},\{1,2\}\}$.
\end{exemple}


\section{Union, intersection, complémentaire, produit}

\begin{definition}(Union, intersection, complémentaire)

Soient $A$ et $B$ des ensembles. Leur \emph{union}, notée $A\cup B$, est la collection formée par les éléments de $A$ et de $B$. Leur \emph{intersection}, notée $A\cap B$, est l'ensemble des éléments de $A$ qui sont également des éléments de $B$ (ou encore : l'ensemble des éléments de $B$ qui sont aussi des éléments de $A$). 

L'ensemble $A\setminus B$ est l'ensemble des éléments de $A$ qui n'appartiennent pas à $B$.

Si $A$ est un sous-ensemble d'un ensemble $E$, le complémentaire de $A$ dans $E$ est $E\setminus A$, on le note aussi $\complement A$ s'il n'y a pas d'ambiguïté sur l'ensemble $E$ dans lequel on prend le complémentaire de $A$.
\end{definition}


\begin{proposition}
Si $A$ et $B$ sont deux parties d'un ensemble $E$, on a :
\[
\complement(A\cup B) = \complement A \cap \complement B; \quad
\complement(A\cap B) = \complement A \cup \complement B.
\]
\end{proposition}
\begin{proof}
Soit $x\in E$. Alors:
\begin{align*}
x\in \complement(A\cup B) 
&\iff \text{non}(x \in A\cup B)\\
&\iff \text{non}(x \in A \text{ ou } x\in B)\\
&\iff \text{non}(x \in A) \text{ et } \text{non}(x\in B)\\
&\iff \left(x\in\complement A\right) \text{ et } \left(x\in\complement B\right)\\
&\iff x\in \complement A \cap \complement B.
\end{align*}
\end{proof}

\begin{definition}[Ensembles disjoints, unions disjointes]
Deux ensembles $A$ et $B$ sont \emph{disjoints} si leur intersection est vide : $A\cap B = \varnothing$. 
\end{definition}


\begin{definition}[Produit cartésien]
Soient $E$ et $F$ deux ensembles. Le \emph{produit cartésien}, ou simplement \emph{produit}, noté $E\times F$, est la collection de tous les couples  de la forme $(x,y)$, avec $x\in E$ et $y\in F$. Si $E=F$, on note $E^2$ au lieu de $E\times E$. 

On peut définir de même les produits finis du type $E_1\times E_2 \times ... E_n$ : leurs éléments sont les $n$-uplets de la forme $(x_1, x_2, ... x_n)$, avec $x_1\in E_1$, $x_2\in E_2$ etc. 
\end{definition}

\begin{remarque}
$E\times F = \varnothing \iff \left( E=\varnothing\text{ ou }F=\varnothing\right)$.
\end{remarque}

\section{Familles indexées}

\begin{definition}
Soient $E$ et $I$ des ensembles. Une famille d'éléments de $E$  indexée par $I$  est un objet de la forme $(x_i)_{i\in I}$, c'est-à-dire la donnée, pour tout élément $i\in I$, d'un élément de $E$ noté $x_i$.
\end{definition}

\begin{exemple}
Une suite réelle $(u_n)_{n\in \N}$ est une famille de réels indexée par $\N$.
\end{exemple}

L'ensemble $I$ qui sert à indexer la famille peut être fini ou infini, et s'il est infini, il peut être plus gros que $\N$ : il n'est pas nécessaire de pouvoir numéroter les éléments de la famille par des nombres : une famille peut être indexée par $\R$. Par exemple, si $a\in \R$, on peut définir la fonction $f_a : \R\to \R; x\mapsto e^{ax}$. Les fonctions $f_a$ forment la famille $(f_a)_{a\in \R}$.


\begin{definition}[Unions et intersections indexées par un ensemble]
Soit $E$ un ensemble, et $(E_i)_{i\in I}$ une famille de parties de $E$ indexée par un ensemble $I$.

Leur \emph{union}, notée $\bigcup_{i\in I} E_i$, est l'ensemble $\{x\in E\:\mid\: \exists i\in I, x\in E_i\}$.

Leur \emph{intersection}, notée $\bigcap_{i\in I} E_i$, est l'ensemble $\{x\in E\:\mid\: \forall i\in I, x\in E_i\}$.
\end{definition}



\chapter{Applications}
\minitoc
\hyperlink{toc}{\retourTOC}

% TODO : mettre exemple dans le plan, par exemple x^2+y^2, antécédents, applications linéaires (montrer surjectivité, injectivité)
% mettre des courbes paramétrées
% images réciproques et fibres <-> courbes de niveau ?
% changer notation image directe et réciproque de parties : f_* et f^* ?
% mettre un exercice sur les B-espaces ?



\section{Applications}

\subsection{Définitions, graphes}

\emph{Résumé : applications, graphes, source, but, images, antécédents, ensemble des fonctions de $E$ dans $F$, fonction caractéristique, point fixe d'une application}

\begin{definition}[Graphe d'application]\label{def-graphe}
Soient $E$ et $F$ deux ensembles, et $\Gamma \subseteq E\times F$. On dit que $\Gamma$ est un \emph{graphe d'application de $E$ dans $F$} si la condition suivante est vérifiée:
\[\forall x\in E, \: \exists! y\in F, \: (x,y) \in \Gamma.\]
\end{definition}

\begin{exemple}[Application directe de la définition]
\begin{enumerate}
\item Si $E = \{1,2,3\}$ et $F = \{1,4\}$, alors l'ensemble $\Gamma = \{(1,4),(2,1),(3,1)\} \subseteq E\times F$ est un graphe d'application.

L'ensemble $\Gamma' = \{(1,1),(2,4)\} \subseteq E\times F$ n'est pas un graphe d'application.

L'ensemble $\Gamma'' = \{(1,1),(2,1),(2,4),(3,4)\} \subseteq E\times F$ non plus.
\item Si $E = F = \R$, l'ensemble $\Gamma = \{x,y)\in \R^2 \:\mid\: y=x^2\}$ est un graphe d'application, mais pas l'ensemble $\Gamma' = \{x,y)\in \R^2 :\mid\: x=y^2\}$. Par contre, si $E=F=(\R_+)^2$, l'ensemble  $\Gamma'' = \{x,y)\in (\R_+)^2 :\mid\: x=y^2\}$ est un graphe d'application de $E$ dans $F$.
\item Pour un sous-ensemble $\Gamma\subseteq E\times F$, être un graphe d'application de $E$ dans $F$ ne dépend pas que de l'ensemble $\Gamma$ lui-même mais aussi de $E$ et de $F$. Par exemple, si $E=\R_+$ et $F=\R_+$, alors $\Gamma = \{(x,y)\in \R_+\times \R_+ \:\mid\: x=y^2\}$ est un graphe d'application de $E$ dans $F$. Par contre, si $E=\R$ et $F=\R_+$, l'ensemble $\{(x,y)\in \R\times \R_+ \:\mid\: x=y^2\}$ (c'est le même que le précédent : les éléments sont les mêmes) n'est \emph{pas} un graphe d'application de $E$ dans $F$.
\end{enumerate}
\end{exemple}

\begin{definition}[Applications/fonction entre ensembles]
\index{application}\index{fonction}\index{domaine}\index{source}\index{codomaine}\index{but}\index{graphe}\index{image!d'un élément}
Une \emph{application} (ou \emph{fonction}: dans ce cours, les deux mots sont synonymes) $f$ est la donnée de trois objets:
\begin{enumerate}
\item un ensemble $E$, appelé le \emph{domaine} ou la \emph{source}, ou encore l'\emph{ensemble de départ} de $f$;
\item un ensemble $F$, appelé le \emph{codomaine} ou le \emph{but} ou encore l'\emph{ensemble d'arrivée} de $f$;
\item une partie $\Gamma_f \subseteq E\times F$, appelée le \emph{graphe de $f$} qui est un \emph{graphe d'application} au sens de la définition précédente. 
\end{enumerate}
Ceci revient à donner $E$, $F$, et pour tout élément $x \in E$, un élément (unique) $y\in F$, appelé l'\emph{image de $x$ par $f$}. Cet élément est noté $f(x)$.
\end{definition}

\begin{attention}
Il existe des ouvrages où les mots \og fonction\fg{} et \og application\fg{} ont des sens différents. Ce n'est \textbf{pas} la convention adoptée ici. Ici, et comme dans, par exemple, les programmes officiels des classes préparatoires, les deux mots désignent rigoureusement le même concept jusqu'à la fin du cours. Autrement dit les mots \og fonction\fg{} et \og application\fg{} sont synonymes.\\
 En particulier, le fameux \og domaine de définition d'une fonction\fg{} (terminologie du lycée, que l'on ne croisera bientôt plus) est ici toujours ...  son domaine tout entier : une fonction est par définition bien définie sur son domaine. La question de savoir si une expression est bien définie peut demeurer, mais doit être résolue \emph{avant} de définir une fonction à l'aide de cette expression.
\end{attention}

Deux fonctions sont égales si elles ont même source et but, et si les images des éléments sont les mêmes. (Et il n'est pas suffisant de demander que les images soient les mêmes.)



\begin{remarque}
\begin{enumerate}
\item Si $f$ est une application de $\R$ dans $\R$, ce que l'on appelle souvent une \og représentation graphique de $f$\fg{} est en fait une représentation graphique de son graphe. La représentation graphique n'est pas unique (l'échelle peut varier, on ne représente en général pas le domaine ni le codomaine en entier mais seulement une partie, etc) mais le graphe, lui, est un objet mathématique abstrait et unique.
\item Une fonction ne peut pas être uniquement définie par son graphe : la donnée du domaine et du codomaine sont nécessaires.
\item \index{application!vide}\index{vide!application}\index{zérologie} (Zérologie : application vide) Soit $E = \varnothing$ et $F$ un ensemble. Il existe une (unique) application de $E$ dans $F$, appelée \emph{application vide}, celle dont le graphe $\Gamma$ est la partie vide de $E\times F = \varnothing$. (Si $E=\varnothing$,  l'assertion \og $\forall x\in E, \exists! y\in F,\: (x,y)\in \Gamma$\fg{} est effectivement vraie même si $\Gamma$ est vide et donc $\Gamma$ est bien un graphe d'application.)
\end{enumerate}
\end{remarque}

Pour définir une fonction de $E$ dans $F$, on écrit \og Soit $f : E\to F$ une fonction\fg. Pour définir une fonction particulière, plutôt que donner son graphe comme le demanderait la définition, on utilise le symbole \og$\mapsto$\fg{} qui se lit \og est envoyé sur / s'envoie sur / est associé à \fg{} comme dans l'exemple suivant:
\[
\text{ Soit } f :\Z \to \R,\: n\mapsto \sqrt{n^2+n+1}.
\]
Ceci se lit par exemple \og Soit $f$ l'application de $\Z$ dans $\R$ qui à (un entier relatif) $n$ associe (le réel) $\sqrt{n^2+n+1}$\fg.


(Dans cet exemple, on devrait auparavant justifier que l'expression sous le radical désigne bien un réel positif, c'est bien le cas : exercice.)

On rencontre également la mise en forme du type suivant:
\[
\text{ Soit } f :\begin{cases}\Z \to \R,\\ n\mapsto \sqrt{n^2+n+1}.\end{cases}
\]
\begin{definition}
Soient $E$ et $F$ des ensembles. L'ensemble des fonctions de $E$ dans $F$ est noté $\mathcal F(E,F)$ ou bien $F^E$ (attention à l'ordre dans la seconde notation).
\end{definition}

\begin{remarque}
Un graphe de fonction n'est pas forcément défini par une formule simple du type $y=\sin(x)$, ou $y=x^2+e^x$. Par exemple, on peut utiliser plusieurs formules suivant l'endroit du domaine où se trouve la variable :
\[ f: 
\R \to \R, 
x\mapsto \begin{cases}\sqrt{x}\text{ si }x\geq 0\\ x^2+x+e^x\text{ sinon.}\end{cases}\]
%Il existe des fonctions pouvant paraître encore plus inhabituelles, par exemple:
%\[ f: 
%\R \to \R, 
%x\mapsto \begin{cases}e^x\text{ si } x\not\in\Q}\\ \text{si $x \in \Q$, le dénominateur (positif) $q$ de la fraction irréductible $\frac{p}{q}$ représentant $x$}\end{cases}\]
%Une fonction n'a pas de raison d'être continue, dérivable etc.
\end{remarque}

\subsection{Images, antécédents, fibres}

On a vu que si $f : E\to F$ est une application et $x\in E$, alors l'élément $f(x)\in F$ est appelé \emph{image de $x$ par $f$}. On peut alors considérer l'ensemble de toutes les images possibles.

\begin{definition}[Image d'une application]\index{image!d'une application}
Soit $f : E\to F$. L'\emph{image} de $f$, notée $\Im(f)$, est l'ensemble de toutes les images des éléments de $E$ par $f$:
\[
\Im(f) = \{f(x)\:\mid\: x\in E\}
\]
\end{definition}

L'image d'une application est donc une partie (stricte ou pas) de son codomaine : $\Im(f)\subseteq F$.

\begin{definition}[Antécédents d'un élément]
Soit $f : E\to F$ une fonction, et $y\in F$. On dit qu'un élément $x\in E$ est un \emph{antécédent} de $y$ si $f(x)=y$.
\end{definition}

\begin{attention}
Un élément du codomaine peut n'avoir \textbf{aucun} antécédent, ou en avoir \textbf{un, ou plusieurs}. Par exemple, si $f : \R\to \R, x\mapsto x^2$, alors l'élément $-1$ n'a aucun antécédent (il n'existe pas de $x\in \R$ tel que $f(x)=x^2=-1$). L'élément $0$ a exactement un antécédent ($0$), et l'élément $4$ a deux antécédents : $2$ et $-2$. Selon les fonctions, un élément peut même avoir une infinité d'antécédents.
\end{attention}

\begin{exercice}
Soit $f : E\to F$. Montrer qu'un élément de $F$ appartient à $\Im f$ si et seulement s'il admet au moins un antécédent.
\end{exercice}

\begin{exemple}
Soit $f : \R^2\to \R,\: (x,y)\mapsto \sqrt{x^2+y^2}$. C'est l'application qui à un point du plan lui associe sa distance à l'origine. L'élément $-3$ n'a aucun antécédent (une distance est positive). L'élément $0$ a un seul antécédent, l'origine $(0,0)$. Le réel $2$ a une infinité d'antécédents : tous les points du plan à distance $2$ de l'origine (donc un cercle).
\end{exemple}

\begin{exercice}
Soit $f : \R\setminus \{1\} \to \R, x\mapsto \frac{2x}{x-1}$. Montrer que l'élément $2 \in \R$ n'a pas d'antécédent. (Autrement dit, montrer que $2$ n'appartient pas à l'image de $f$.)
\end{exercice}


\begin{definition}[Fibres d'une application]
\index{fibre}
Soit $f : E\to F$ et $y\in F$. La \emph{fibre de $f$ au-dessus de $y$} est par définition l'ensemble des antécédents de $y$, c'est-à-dire $\{x\in E \:\mid\: f(x)=y\}$.

En reformulant, la fibre de $f$ au-dessus de $y$ est donc l'ensemble des solutions de l'équation $f(x)=y$, d'inconnue $x$ et de paramètre $y$. (Une fibre peut donc éventuellement être vide.)
\end{definition}

\begin{exemple}
Soit $f : \R\to \R, x\mapsto x^2$. Soit $y\in \R$. La fibre de $f$ au-dessus de $y$ est l'ensemble 
\[
\{x\in \R\:|\: x^2=y\} = 
\begin{cases}
\varnothing &\quad\text{si}\quad y<0\\
\{0\} & \quad\text{si}\quad y=0\\
\{-\sqrt y, \sqrt y\}& \quad\text{si}\quad y>0.
\end{cases}
\]
(La distinction entre les deux derniers cas est superflue car $\{-0,0\}=\{0\}$ mais elle permet de souligner le changement de situation.)
\end{exemple}

\begin{exemple}
Soit $f : \R^2\to \R,\: (x,y)\mapsto 2x-3y$. La fibre de $f$ au-dessus de $5$ est l'ensemble des points $(x,y) \in R^2$ tels que $2x-3y=5$. Autrement dit, la fibre au-dessus de $5$ est la droite de $\R^2$ d'équation cartésienne $2x-3y-5=0$. (Exercice : les autres fibres sont également des droites, toutes parallèles.)
\end{exemple}

\begin{definition}[Fonction caractéristique]
Soit $E$ un ensemble et $A\in \mathcal P(E)$ une partie de $E$. La \emph{fonction caractéristique} de $A$ (sous-entendu, dans $E$) est la fonction 
\[
\operatorname{1}_A :\begin{cases}E \to \{0,1\},\\ x\mapsto \begin{cases}1&\text{ si } x\in A\\0&\text{ si } x\not\in A\end{cases}\end{cases}
\]
\end{definition}




\subsection{Retour sur les familles et les paramétrages}
\label{subsec-retour-parametrage-familles}

On peut maintenant donner une nouvelle définition de ce qu'est une \emph{famille} :

\begin{definition}
Soient $I$ et $E$ deux ensembles. Une famille $(a_i)_{i\in I}$ d'éléments de $E$ indexée ou paramétrée par $I$ est une application
\begin{align*}
I&\to E\\ i&\mapsto a_i.
\end{align*}
\end{definition}

Par exemple, une suite de réels, notée $(u_n)_{n\in \N}$, est  simplement une application de $\N$ dans $\R$, que l'on pourrait aussi noter  $u : \N\to \R$, $n\mapsto u(n)$. Pour diverses raisons, on utilise pour les suites la notation en famille plutôt que la notation fonctionnelle, mais les deux objets sont identiques. L'ensemble des suites réelles est donc $\R^\N$, celui des suites complexes est $\C^\N$ etc.

En ce qui concerne les \textbf{écritures paramétrées} d'ensembles, il s'agit en fait de les écrire comme des \textbf{images par une application}. Par exemple, la droite du plan $\mathcal D$ d'équation $2x+3y=4$ admet l'écriture paramétrée $\{(2,0)+t(3,-2)\:\mid\: t\in \R\}$. Ceci revient à considérer l'application
\[ \phi:\begin{cases} \R &\to \R^2\\ t &\mapsto (2+3t,-2t)\end{cases}
,\]
et à écrire que $\mathcal D = \Im \phi  = \{\phi(t)\:\mid\: t\in \R\}$.

Ceci permet  de comprendre d'une autre façon que $\left(u_i\right)_{i\in I}$ est un objet d'une nature bien différente de $\left\{u_i\:\mid\: i\in I\right\}$ : le premier est une application, le second est son image. Une famille prend en compte les éventuelles \og répétitions\fg{} des éléments. La suite $((-1)^n)_{n\in \N}$ oscille indéfiniment entre $1$ et $-1$, alors que l'ensemble $\left\{(-1)^n\:\mid\: n\in \N\right\}$ est simplement l'ensemble image $\{1,-1\}$ qui ne contient que deux éléments.

\subsection{Diagrammes}

Pour terminer cette section, on introduit un outil indispensable pour visualiser efficacement plusieurs ensembles et applications d'un seul coup : les diagrammes.

\begin{definition}[Diagramme]
\index{diagramme}
Lorsque l'on est en présence de plusieurs ensembles et de plusieurs applications entre ces ensembles, il est classique de visualiser la situation à l'aide d'un diagramme. Un diagramme (d'applications entre ensembles) est un graphe orienté dont chaque sommet représente un ensemble et chaque arête (orientée) représente une application.
\end{definition}

Attention, un tel graphe n'est pas simple, autrement dit il peut avoir des boucles (s'il y a des applications d'un ensemble dans lui-même) ou des arêtes multiples (s'il y a plusieurs applications distinctes entre deux ensembles donnés).

\begin{exemple}
\begin{enumerate}
\item \'Etant donnés des ensembles $A$, $B$, $C$ et $D$ et des applications $f : A\to B$, $g : A\to C$, $h : D\to A$, $\phi : D\to C$, $\psi : B\to A$, on peut représenter la situation en donnant simplement diagramme suivant:
\[
\xymatrix{
A \ar@/^1pc/[r]^{f} \ar[d]_{g} & B \ar@/^/[l]_{\psi}\\
C & D \ar[ul]^{h} \ar[l]^{\phi}
}
\]
\item \'Etant donnés des ensembles $E$ et $F$ et des applications $f : E\to E$, $g : E\to F$ et $h : E\to F$, on peut représenter la situation par le diagramme:
\[
\xymatrix{
 E \ar@(dl,ul)^{f} \ar@/^1pc/[rrr]^{g} \ar@/_1pc/[rrr]^{h}
& & &  F 
}
\]
\end{enumerate}
\end{exemple}



%----------------------------------
\section{Composition}
%----------------------------------

\emph{Résumé : composition, factorisation, la composition est associative, fonction identité, diagramme (commutatif ou pas), }




\begin{definition}[Composition]
\index{composition de deux applications}
Soit $f : X\to Y$ et $g : Y\to Z$ deux fonctions. La composée de $g$ et de $f$ est la fonction $g\circ f$ (se lit \og $g$ rond $f$\fg) de $X$ dans $Z$ qui à $x\in X$ associe $g(f(x)) \in Z$. Autrement dit, par définition, $(g\circ f)(x) = g(f(x))$.

Une composition d'applications se visualise à l'aide du diagramme\index{diagramme} suivant (attention à l'ordre : appliquer la fonction $g\circ f$ consiste à appliquer $f$ \emph{suivie} de $g$):
\[
\xymatrix{
X \ar[r]_{f} \ar@/^1pc/[rr]^{g\circ f}& Y \ar[r]_{g}& Z
}
\]
\end{definition}

Plus généralement, pour pouvoir composer deux fonctions il est suffisant que le codomaine de la première fonction (dans l'ordre de la composition) soit inclus dans le domaine de la seconde. (Cette condition est bien sûr également nécessaire, autrement l'écriture $g(f(x))$ n'a pas de sens. Deux fonctions quelconques ne sont donc en général pas composables.)

\begin{definition}[Factorisation d'une application comme composée d'autres applications]
\index{factorisation!d'une application comme composée}
\'Ecrire une application $f$ sous la forme d'une composition $f = g\circ h$, c'est la \emph{factoriser}. Plusieurs factorisations sont possibles. Par exemple, si $f : \R\to \R, x\mapsto x^2+1$, on peut écrire $f : g\circ h$, avec
\[
h : \R\to \R, x\mapsto x^2 
\text{ et }
g : \R\to \R, x\mapsto x+1.
\]
\end{definition}

\noindent\fbox{
\begin{minipage}{.98\linewidth}
La notion de composition d'applications est centrale dans ce cours (et en mathématiques). Une compétence attendue en fin de semestre est d'interpréter rapidement des fonctions comme des composées de fonctions plus simples, c'est-à-dire de factoriser de tête certaines applications. Par exemple, la fonction $f : \R\to \R, x\mapsto \ln\sqrt{1+e^x}$ peut naturellement s'écrire comme la composée de quatre applications simples:
\[ \R \xrightarrow{x\mapsto e^x} \R_+^* \xrightarrow{x\mapsto x+1} \R_+^* \xrightarrow{x\mapsto \sqrt x} \R_+^* \xrightarrow{x\mapsto \ln x} \R.\]
Une telle factorisation se voit instantanément avec un peu de pratique et  permet par exemple de montrer sans aucun calcul que $f$ est croissante.
\end{minipage}
}


\begin{attention}[La composition n'est pas commutative]
Attention, même si les domaines et codomaines permettent de composer deux fonctions dans les deux sens, les fonctions $g\circ f$ et $g\circ f$ obtenues sont en général distinctes. Par exemple, avec $f : \R\to R, x\mapsto x^2
$ et $g : \R\to \R, x\mapsto x+1$,  on peut composer dans les deux sens mais on a :
\begin{align*}
g\circ f : \R\to \R, & x\mapsto g(f(x)) = g(x^2) = x^2+1,\\
f\circ g : \R\to \R, & x\mapsto f(g(x)) = f(x+1) = (x+1)^2 = x^2+2x+1.
\end{align*}
\end{attention}


\begin{proposition}[\og La composition est associative\fg]
\index{associative (composition)}
Soient $f : X\to Y$, $g : Y\to Z$, $h  : Z\to T$ des fonctions. Alors $h\circ (g\circ f) = (h\circ g)\circ f$. Cette fonction est notée $h\circ g\circ f$.

Ce résultat se visualise à l'aide du diagramme\index{diagramme}:
\[
\xymatrix{
X \ar[r]_{f} \ar@/^1pc/[rr]^{g\circ f} \ar[r]_{f} \ar@/^3pc/[rrr]^{h\circ (g\circ f)} \ar@/_3pc/[rrr]_{(h\circ g) \circ f}& Y \ar[r]_{g} \ar@/_1.2pc/[rr]_{h\circ g} & Z \ar[r]^{h} & T
}
\]

\end{proposition}
\begin{proof}
Les domaines et codomaines sont les mêmes ($X$ et $T$), et si $x\in X$, on a 
\[ \left(h\circ (g\circ f)\right) (x) = h((g\circ f)(x)) = h(g(f(x)) \text{ et } \]
\[ \left( (h\circ g)\circ f \right) (x) = (h \circ g)(f(x)) = h(g(f(x))\quad \]
d'où l'égalité des deux fonctions.
\end{proof}

Cette proposition permet de ne pas avoir à noter les parenthèses lors de compositions successives, puisque toutes les possibilités de parenthésage donnent la même fonction.

\begin{definition}[Fonction identité]
Soit $E$ un ensemble. La fonction identité sur $E$ est la fonction $\Id_E : E\to E, x\mapsto x$. (En d'autres termes, la fonction identité de $E$ est la fonction de $E$ dans $E$ dont tous les points sont fixes.)
\end{definition}

\begin{attention}
\begin{enumerate}
\item Ne pas confondre la fonction identité avec une fonction constante (ou avec fonction nulle si le but est $\R$).
\item Si $\phi = E\to F$, alors $\phi = \phi\circ \Id_E = \Id_F\circ \phi$.
\end{enumerate}
\end{attention}


\begin{definition}[Diagramme commutatif]
\index{diagramme!commutatif}
Considérons un diagramme d'applications entre ensembles. En général, il y a plusieurs chemins entre deux sommets donnés, et ces chemins correspondent à différentes fonctions composées entre les deux ensembles.

On dit que le diagramme \emph{est commutatif} (ou \emph{qu'il commute}) lorsque pour tout couple de sommets, les différentes applications composées reliant ces ensembles sont égales.
\end{definition}

\begin{exemple}
\begin{enumerate}
\item  Par exemple, dire que le diagramme
$
\xymatrix{
X \ar[d]_{f} \ar[r]^{h}& Z \\
Y \ar[ur]_{g}& 
}
$
est commutatif revient à dire que $h = g\circ f$.
\item Dire que le diagramme 
$\xymatrix{
A \ar[r]^{\phi} \ar[d]_{f} & B \ar[d]^{g}\\
C \ar[r]^{\psi}& D 
}$
est commutatif revient à dire que $\psi\circ f = g \circ \phi$. Un tel diagramme est appelé \emph{carré commutatif}.
\end{enumerate}
\end{exemple}

\begin{attention}
En général, un diagramme d'applications n'a aucune raison d'être commutatif.
Par exemple, le diagramme
\[\xymatrix{
\R \ar[rr]^{x\mapsto x^2} \ar@/_1pc/[rrrr]_{x\mapsto e^x}  & & \R \ar[rr]^{x\mapsto \cos(x)} & & \R
}\]
n'est pas commutatif, puisque l'assertion \og $\forall x\in \R, \cos(x^2)=e^{x}$\fg{} est fausse. Encore plus simplement, le diagramme suivant n'est pas commutatif car les deux fonctions sont différentes :
\[\xymatrix{
\R \ar@/^/[rr]^{x\mapsto 2x} \ar@/_/[rr]_{x\mapsto x+1} & & \R
}\]
\end{attention}

\begin{exercice} Montrer que le diagramme suivant est commutatif :
\[\xymatrix{
\R  \ar[r]^{x\mapsto x^2} \ar[d]_{x\mapsto \cos(x^2)} & \R  \ar[d]^{x\mapsto e^{\cos(x)}}\\
\R  \ar[r]^{x\mapsto e^x}& \R 
}\]
\end{exercice}




%---------------------------------
\section{Applications réciproques, sections et rétractions}
%---------------------------------

\emph{Résumé : fonctions réciproque, unicité de la réciproque, sections et rétractions}


\begin{definition}[Fonction réciproque]
\index{réciproque (fonction)}
Soient $f : E\to F$ et $g = F\to E$ deux fonctions. On dit qu'elles sont réciproques l'une de l'autre (ou que $g$ est une réciproque de $f$, ou que $f$ est une réciproque de $g$) si $g\circ f = \Id_E$ \underline{et} $f\circ g = \Id_F$. 
\[
\xymatrix{
 E \ar@(dl,ul)^{\Id_E} \ar@/^2pc/[rrr]^{f} \ar@(ur,dr)^{g\circ f} 
& & & 
\ar@/^2pc/[lll]^{g} \ar@(dl,ul)^{f\circ g} F \ar@(ur,dr)^{\Id_F}
}
\]
\end{definition}



Attention, une fonction $f$ n'a pas toujours de fonction réciproque!

\begin{proposition} Soit $f : E\to F$ une fonction. Si elle admet une (fonction) réciproque, alors celle-ci est unique. Elle est notée généralement $f^{-1}$.
\end{proposition}
\begin{proof}
Soient en effet $g = F\to E$ et $h : F \to E$ deux réciproques de $f$. Alors
\[
g\circ  f \circ h = (g\circ f) \circ h = \Id_E \circ h = h, \text{ et}
\]
\[
g\circ  f \circ h = g\circ (f \circ h) = g \circ \Id_F = g
\]
d'où $g=h$.
\end{proof}



\begin{exemple}
Les fonctions $f = \R\to \R_+^*, x\mapsto e^x$ et $g : \R_+^*\to R, x\mapsto \ln(x)$ sont réciproques l'une de l'autre.
\end{exemple}

\begin{attention}
\begin{enumerate}
\item On ne doit pas utiliser la notation $f^{-1}$ avant d'avoir démontré que la fonction admet effectivement une réciproque.
\item Ne pas confondre fonction réciproque $f^{-1}$ et fonction \emph{inverse} $\frac{1}{f}$. La notion de fonction inverse  concerne les fonctions à valeurs réelles qui ne s'annulent pas et n'a rien à voir avec la notion de réciproque. Par exemple, la fonction inverse de l'exponentielle est $x\mapsto 1/e^x = e^{-x}$, alors que sa fonction réciproque est $x\mapsto \ln(x)$
\end{enumerate}
\end{attention}

\begin{attention}
Dans la définition de réciproque, les conditions $g\circ f = \Id_E$ et  $f\circ g = \Id_F$ sont toutes les deux nécessaires : il est en effet possible que l'une soit vérifiée et pas l'autre. Par exemple, les fonctions 
\[
f : \R_+\to \R, x\mapsto \sqrt x
\quad \text{ et }\quad
g : \R\to \R_+, x\mapsto x^2
\]
vérifient $g\circ f=\Id_{\R_+}$, mais $f\circ g \neq \Id_{\R}$ : en effet, on a $(f\circ g)(x)=\sqrt{x^2}=|x|\neq x$.

Dans ce type de cas, on n'est pas en présence de fonctions réciproques mais la situation porte tout de même un nom. C'est l'objet de la définition suivante.
\end{attention}

\begin{definition}[Rétraction/inverse à gauche, section/inverse à droite]
Soit $f : E\to F$ une fonction.
\begin{enumerate}
\item Une \emph{rétraction} (ou \emph{inverse à gauche}) de $f$, est une fonction $r:F\to E$ telle que $r\circ f = \Id_E$.
\[
\xymatrix{
 E \ar@(dl,ul)^{\Id_E} \ar@/^1pc/[rrr]^{f}
& & & 
\ar@/^1pc/[lll]^{r}  F 
}
\]
\item Une \emph{section} (ou \emph{inverse à droite}) de $f$, est une fonction $s:F\to E$ telle que $f \circ s = \Id_F$.
\[
\xymatrix{
 E  \ar@/^1pc/[rrr]^{f} 
& & & 
\ar@/^1pc/[lll]^{s}  F \ar@(ur,dr)^{\Id_F}
}
\]
\end{enumerate}
\end{definition}

\begin{exemple}
Soit $f = \R\to \R_+, x\mapsto x^2$. Alors les fonctions $s_1 : \R_+\to \R, x\mapsto \sqrt x$ et $s_2 : \R_+\to \R, x\mapsto -\sqrt x$ sont deux sections (inverses à droite) distinctes de $f$ (on a bien $f\circ s_1 = \Id_{\R^+}$ et $f\circ s_2 = \Id_{\R^+}$). 
Par ailleurs, $f$ est une rétraction de $s_1$ et de $s_2$. (Voir remarque ci-dessous.)
%Soit $f : \{1,2\} \to \{3,4,5\}$ définie par $f(1)=3$ et $f(2)=5$. La fonction $g : \{3,4,5\}\to\{1,2\}$ telle que $g(3)=1$, $g(4)=2$ et $g(5)=2$ est une rétraction de $f$.  Par ailleurs, la fonction $f$ est une section de $g$ (voir remarque plus bas).
\end{exemple}

\begin{remarque}
\begin{enumerate}
\item Une fonction $g$ est une rétraction de $f$ si et seulement si $f$ est une section de $g$ puisque les deux assertions signifient $g\circ f = \Id_E$ : les deux notions sont \og duales\fg.
\item Les sections et rétractions, lorsqu'elles existent, ne sont en général pas uniques (contrairement à la fonction réciproque qui est unique si elle existe).
\item Une fonction réciproque est à la fois une rétraction et une section (ou : à la fois un inverse à gauche et un inverse à droite).
\item De même que toutes les fonctions n'ont pas forcément de réciproque, toutes les fonctions n'admettent pas forcément une section ou une rétraction. Par exemple, $f : \R\to \R, x\mapsto x^2$ n'admet ni section ni rétraction. 
\end{enumerate}
\end{remarque}

% autre exemple avec inverse à gauche / pas à droite ou l'inverse



%---------------------------------
\section{Restriction, prolongement, corestriction}
%---------------------------------

\emph{Résumé : restriction, prolongement, corestriction}

\begin{definition}[Restriction]
\index{restriction d'une application}\label{def-restriction}
Soit $f : E\to F$ et $A \in \mathcal P(E)$ une partie de $E$. La \emph{restriction} de $f$ à $A$, notée $f|_{A}$, est l'application de $A$ dans $F$ suivante:
\[
f|_{A} : A\to F, \: x\mapsto f(x)
\]
Attention, les fonctions $f|_A$ et $f$ doivent être considérées comme distinctes car leurs domaines sont distincts ($A$ au lieu de $E$).
\end{definition}

\begin{remarque}
Dans le contexte de la définition, si on note $i : A\to E, x\mapsto x$ l'inclusion de $A$ dans $E$, alors la restriction $f|_A$ est simplement la composition $f\circ i$.
\[\xymatrix{
A \ar[r]^{i} \ar@/_/[rr]_{f\circ i = f|_A} & E \ar[r]^f & F
}\]
\end{remarque}

\begin{definition}[Prolongement]
Soient $E$ et $F$ des ensembles, $A\in \mathcal P(E)$ une partie de $E$ et $f : A\to F$ une fonction. On dit qu'une application $g : E\to F$ est un \emph{prolongement} de $f$ si $g|_A = f$.
\end{definition}

\begin{attention}
Il existe en général plusieurs prolongements possibles d'une même fonction et même si la fonction $f$ est donnée par une formule, un prolongement n'a aucune raison d'être défini par la même formule hors du domaine originel de $f$. Par exemple, si $f : \R_+^* \to \R, x\mapsto e^x$, alors les fonctions suivantes  sont des prolongements de $f$ (à divers domaines):
\[g : \R^* \to \R, x\mapsto \begin{cases}e^x\text{ si } x>0\\ \sin(x) \text{ si } x<0\end{cases},\]
\[h : \R_+ \to \R, x\mapsto \begin{cases}e^x\text{ si } x>0\\ 10 \text{ si }x=0\end{cases}.\]
(Un prolongement ne doit pas non plus être forcément continu ni dérivable, etc.)
\end{attention}


\begin{definition}[Corestriction]
\index{corestriction}\label{def-corestriction}
Soit $f : E\to F$, et $B$ une partie de $F$ contenant toutes les images des éléments de $E$ (autrement dit, $\forall x\in E, f(x)\in B$).
La \emph{corestriction} de $f$ à $B$ est l'application de domaine $E$, codomaine $B$ et de même graphe que $f$, autrement dit c'est l'application 
\[ g : E\to B, x\mapsto f(x).\]
\end{definition}




\begin{exemple}
Soit $f : \R\to \R, x\mapsto x^2$. On peut la corestreindre à $[-3,+\infty[$ car cette partie de $\R$ contient toutes les images de $f$. La corestriction de $f$ à $[-3,+\infty[$ est $g : \R\to [-3,+\infty[, x\mapsto x^2$.
\end{exemple}


\begin{attention}
La corestriction de $f$ à $B$ se note $f|^{B}$ mais cette notation n'est pas (encore ou partout) aussi standard que celle pour la restriction. Pour cette raison, il est conseillé, plutôt que d'utiliser la notation $f|^{B}$, de redéfinir explicitement la fonction. Par exemple, plutôt que d'écrire \og soit $g=\sin|^{[-1,1]}$\fg, on écrira \og soit $g : \begin{cases} \R&\to [-1,1],\\ x&\mapsto \sin(x)\end{cases}$.\fg{}

Par contre, la notation pour la restriction $f|_A$, elle, est complètement standard et on peut l'utiliser sans plus de précisions.
\end{attention}


%---------------------------------
\section{Fonctions injectives et surjectives}
%---------------------------------

\emph{Résumé : fonctions injectives, surjectives, bijectives, stabilité par composition, réciproques partielles, injectivité ssi rétractions, surjectivité ssi  sections, bijectivité ssi réciproque}



\begin{definition}[Fonction injective]
\index{injection}
Soit $f : A \to B$ une application. On dit que $f$ est \emph{injective} (ou que c'est une \emph{injection}) si les conditions équivalentes suivantes sont vérifiées.
\begin{enumerate}
\item $\forall (x,y) \in A^2,\quad f(x)=f(y) \Rightarrow  x=y$.
\item $\forall (x,y) \in A^2,\quad x\neq y \Rightarrow  f(x)\neq f(y)$. (La contraposée de la précédente.)
\item Deux éléments distincts de $A$ ont des images distinctes. 
\item Un élément de $B$ admet au plus un antécédent par $f$.
\end{enumerate}
Une formulation moins précise mais parlante est qu'une fonction injective \og sépare les points\fg{}.
\end{definition}

\begin{exemple}
$f : \R\to \R, x\mapsto x^2$ n'est \textbf{pas} injective puisque $f(1)=f(-1)$.
\end{exemple}


\begin{exercice}
Montrer que  $\sin : \R\to \R, x\mapsto \sin(x)$ n'est pas injective, mais que $f:\R_+\to \R_+, x\mapsto \sqrt{x+1}$ l'est.
\end{exercice}


% méthode : pour rédiger une preuve, on utilise en général la première formulation, ou la seconde. Les autres formulations servent plutôt à assimiler la notion.
% méthode : pour montrer qu'une fonction n'est pas injective, il suffit de trouver deux éléments distincts qui ont même image.
% mettre un petit exo avec exemple et contre-exemple ?

\begin{definition}[Fonction surjective]
\index{surjection}
Soit $f : A \to B$ une application. On dit que $f$ est \emph{surjective} (ou que c'est une \emph{surjection}) si
\[\forall b \in B,\quad \exists a\in A / f(a)=b,\]
autrement dit tout élément $b\in B$ a (au moins) un antécédent par $f$.
Une formulation moins précise mais parlante est  qu'une fonction surjective \og recouvre son codomaine\fg{}.
\end{definition}

\begin{exemple}
$f : \R\to \R, x\mapsto x^2$ n'est \textbf{pas} surjective puisque $-1$ n'a pas d'antécédent par $f$.
\end{exemple}

\begin{exercice}
Montrer que  $\exp: \R\to \R, x\mapsto e^x$ n'est pas surjective.
\end{exercice}


\begin{exercice}[paramétrage standard du cercle]\index{cercle}\label{exo-param-cercle}
Soit $\S^1 = \ensemble{(x,y)\in \R^2}{x^2+y^2=1}$ le cercle unité de $\R^2$. On considère la fonction $f : \R\to \R^2, \quad t\mapsto (\cos t, \sin t)$. Montrer que son image est le cercle $\S^1$. Autrement dit, montrer que
\[ \S^1=\ensemble{(\cos t, \sin t)}{t\in \R}.\]
D'autre part, $f$ est-elle injective ?
\end{exercice}
\begin{red}
La fonction n'est pas injective puisque $f(0)=(\cos 0, \sin 0) = (\cos2\pi, \sin 2\pi) = f(2\pi)$. Son image est un sous-ensemble de $\S^1$ puisque si $t\in \R$, alors $\cos(t)^2+\sin(t)^2=1$, donc $f(t)=(\cos t,\sin t) \in \S^1$. Montrons maintenant que son image est exactement $\S^1$.

Soit $(x,y)\in \S^1$. Le fait qu'il existe $t\in \R$ tel que $(x,y)=(\cos t,\sin t) $ résulte du cours sur l'exponentielle complexe, ou bien sur les fonctions trigonométriques. Si l'on connaît les fonctions trigonométriques inverses, on peut même expliciter un antécédent (parmi d'autres) : si $y\geq 0$, on peut choisir $t=\arccos(x)$, et si $y\leq 0$ on peut choisir $t=-\arccos(x)$. Ceci fournit un antécédent $t\in [-\pi,\pi]$, mais il y en a une infinité d'autres.
\end{red}


\begin{exercice}\label{exo-decomp-polaire}
Montrer que $g : \R_+\times \S^1 \to \R^2, \quad (r,(x,y))\mapsto (rx,ry)$ est surjective.
\end{exercice}
\begin{red}
Soit $(x',y')\in \R^2$. Montrons qu'il possède un $g$-antécédent en distinguant deux cas de figure:
\begin{itemize}
\item Si $(x',y')=(0,0)$, alors $(x',y')=g(0,(0,1))$, donc $(0,(0,1))$ est un $g$-antécédent de $(x',y')$. (En fait, tout élément de la forme $(0,(x,y))$ est un antécédent de $(0,0)$.)
\item Si $(x',y')\neq (0,0)$, alors la quantité $r=\sqrt{x'^2+y'^2}$ est non nulle et donc on peut considérer l'élément\footnote{Il faut  voir $(x,y)$ comme la projection radiale de $(x',y')$ sur le cercle unité de $\R^2$.} $(x,y)=\left(\frac{x'}{r},\frac{y'}{r}\right)$.

Par construction, on a $(x,y)\in \S^1$, et d'autre part $(x',y')=(rx,ry)=g\big(\:(r,(x,y))\:\big)$. Donc $(r,(x,y))$ est un $g$-antécédent de $(x',y')$. 
\end{itemize}
On en déduit finalement que $g$ est surjective. Remarquer que l'origine du plan a une infinité d'antécédents, mais que les autres points du plan n'ont qu'un seul antécédent.
\end{red}





\begin{definition}[Fonction bijective]
\index{bijection}
On dit que $f : A\to B$ est \emph{bijective} (ou que c'est une \emph{bijection}) si elle est injective et surjective. Autrement dit, $f$ est bijective si tout élément $y\in B$ admet \emph{exactement} un antécédent par $f$.
\end{definition}

\begin{exemple}
\begin{enumerate}
\item La fonction identité (d'un ensemble $E$ dans lui-même) est bijective.
\item Si $A\subseteq B$, la fonction $i : A\to B, x\mapsto x$, appelée \emph{l'inclusion de $A$ dans $B$}, est injective. De même que les fonctions identité, les fonctions d'inclusion jouent souvent un rôle important malgré leur apparence anodine.
\item Si $f : E\to F$ est injective, sa restriction $f|_A$ à toute partie $A\subseteq E$ est injective.
\item On a déjà vu que $f : \R \to \R, x\mapsto x^2$ n'est ni injective, ni surjective. La corestriction de $f$ à $\R_+$ est $g : \R \to \R_+, x\mapsto x^2$. Elle n'est pas injective pour les mêmes raisons que $f$, mais elle est surjective : le codomaine est cette fois $\R_+$, et tout nombre réel positif $y\geq 0$ a au moins un antécédent, par exemple $-\sqrt{y}$.
\item La restriction de $g$ à $\R_+$ est $h : \R_+ \to \R_+, x\mapsto x^2$. Elle est injective et surjective, donc bijective. Elle est injective car si $x$ et $y$ sont des réels positifs ayant même carré, ils sont forcément égaux (ils sont positifs donc il n'y a pas l'ambiguïté de signe). Sa surjectivité ne découle pas directement de celle de $g$ car le domaine a été restreint : elle est surjective car tout nombre réel positif $y\geq 0$ a au moins un antécédent \emph{dans $\R_+$}, à savoir $\sqrt{y}$.
\item (Zérologie)\index{zérologie} L'unique fonction de $\varnothing$ dans $\varnothing$ est bijective. La fonction vide de $\varnothing$ dans n'importe quel ensemble est toujours injective.
\end{enumerate}
\end{exemple}

\begin{exercice}
Montrer que  $f: \R\to \R, x\mapsto x^3$ est bijective.
\end{exercice}

\begin{exercice}
On définit l'application 
\[\phi : \N\to \Z, n\mapsto \begin{cases} n/2&\text{ si $n$ est pair},\\ -\frac{n+1}{2} & \text{sinon.}\end{cases}
\]
Montrer que $\phi$ est (bien définie et) bijective. Déterminer son application réciproque.
\end{exercice}

Cet exercice prouve donc le résultat suivant, fondamental (et potentiellement contre-intuitif) :

\begin{mdframed}[linewidth=2]
Il existe une bijection entre $\N$ et $\Z$.
\end{mdframed}


\begin{remarque}
En général, la surjectivité est plus difficile à montrer que l'injectivité, car il faut résoudre une équation à paramètre : l'équation $f(x)=y$, de paramètre $y$, et d'inconnue $x$, et ce pour tous les paramètres $y$. La non surjectivité est en revanche souvent plus facile à montrer, il suffit de trouver un élément qui n'a pas d'antécédent, en général cela se voit (éventuellement après un petit calcul / majoration / développement d'expression).
\end{remarque}


\begin{proposition}[Stabilité à la composition de l'injectivité et de la surjectivité]\label{prop-composition-inj-surj-bij}
Soient $f : E\to F$ et $g : F\to G$ deux fonctions.
\begin{enumerate}
\item Si $f$ et $g$ sont injectives, alors $g\circ f$ l'est également.
\item Si $f$ et $g$ sont surjectives, alors $g\circ f$ l'est également.
\item Si $f$ et $g$ sont bijectives, alors $g\circ f$ l'est également.
\end{enumerate}
\end{proposition}
\begin{proof}
\begin{enumerate}
\item Soient $x, y \in E$ tels que $(g\circ f)(x) = (g\circ f)(y) $, c'est-à-dire tels que $g(f(x))=g(f(y))$. Comme $g$ est injective, on a $f(x)=f(y)$. Comme $f$ est injective, on a alors $x=y$, ce qu'il fallait démontrer.
\item Soit $z\in G$. Comme $g$ est surjective, $z$ possède un antécédent par $g$ c'est-à-dire qu'il existe $y\in F$ tel que $g(y)=z$. Ensuite, comme $f$ est surjective, $y$ possède un antécédent par $f$, c'est-à-dire qu'il existe $x\in E$ tel que $f(x)=y$. On a alors $g(f(x)) = g(y)=z$, donc $x$ est un antécédent de $z$ par $g\circ f$. Ceci montre que tout élément de $G$ possède un antécédent par $g\circ f$, donc que $g\circ f$ est surjective.
\item Il suffit d'appliquer les deux premiers points.
\end{enumerate}
\end{proof}



\begin{exemple}
Cette proposition permet d'éviter des calculs parfois lourds. Par exemple, la fonction $f : \R\to \R, x\mapsto \left(\ln(e^x+2)\right)^3$ est injective car elle s'écrit comme la composée des fonctions injectives :
\[ \R \xrightarrow{x\mapsto e^x} \R_+^* \xrightarrow{x\mapsto x+2} \R_+^* \xrightarrow{x\mapsto \ln(x)} \R \xrightarrow{x\mapsto x^3}\R\]
La proposition est encore plus utile pour vérifier le caractère surjectif de fonctions.
\end{exemple}

\begin{exercice}[Coordonnées polaires sur le plan]\label{exo-coord-polaires-plan}
Montrer que 
\[\R^2 = \ensemble{(r\cos t, r\sin t)}{(r,t) \in \R_+\times \R}.\]
Autrement dit, montrer que l'application $\Phi : \R_+\times \R \to \R^2, \quad (r,t) \mapsto (r\cos t, r\sin t)$ est surjective. Plutôt que d'utiliser un calcul direct, il est recommandé de l'écrire comme composée et d'utiliser la proposition \ref{prop-composition-inj-surj-bij}, ainsi que les exercices \ref{exo-param-cercle}, \ref{exo-decomp-polaire}  et \ref{exo-produit-surjections}.
\end{exercice}



\begin{proposition}[réciproques partielles]
Soient $f : E\to F$ et $g : F\to G$ deux fonctions.
\begin{enumerate}
\item Si $g\circ f$ est injective, alors $f$ l'est également.
\item Si $g\circ f$ est surjective, alors $g$ l'est également.
\end{enumerate}
\end{proposition}
\begin{proof}
\begin{enumerate}
\item Soient $x, y\in E$ tels que $f(x)=f(y)$. En appliquant $g$, il vient $g(f(x))=g(f(y))$. Comme $g\circ f$ est injective, $x=y$.
\item Soit $z\in G$. Comme $g\circ f$ est surjective, il existe $x\in E$ tel que $g(f(x))=z$. Posons $y = f(x)$. On a $g(y)=z$ donc $y$ est un antécédent de $z$ par $g$.
\end{enumerate}
\end{proof}




\begin{corollaire}[de la proposition]\label{bijective-si-reciproque}
Soit $f : E\to F$ une application.
\begin{enumerate}
\item Si elle admet une rétraction (inverse à gauche), alors elle est injective.
\item Si elle admet une section (inverse à droite), alors elle est surjective.
\item Si elle admet une fonction réciproque, alors elle est bijective.
\end{enumerate}
\end{corollaire}
\begin{proof}
\begin{enumerate}
\item Soit $r$ une rétraction de $f$. La composée $r\circ f = \Id_E$ est injective donc par la proposition précédente, $f$ est injective.
\item Soit $s$ une section de $f$. La composée $f \circ s= \Id_F$ est surjective donc par la proposition précédente, $f$ est surjective.
\item Une réciproque étant à la fois une section et une rétraction, on applique les deux points précédents.
\end{enumerate}
\end{proof}


\begin{remarque}
\textbf{Attention}, on peut avoir $g\circ f$ injective et $g$ non injective, et on peut aussi avoir $g\circ f$ surjective et $f$ non surjective. Considérons par exemple:
\[
f : \N\to \N, n\mapsto 2n
\quad \text{ et }\quad
g : \N\to \N, n\mapsto \lfloor n/2\rfloor.
\]
Alors $g\circ f = \Id_\N$ donc est bijective, mais $f$ n'est pas surjective et $g$ n'est pas injective.

On peut également considérer les fonctions
\[
f : \R_+\to \R, x\mapsto \sqrt x
\quad \text{ et }\quad
g : \R\to \R_+, x\mapsto x^2
\]
qui vérifient également $g\circ f=\Id_{\R_+}$ sans que $f$ soit surjective ni $g$ injective.
\end{remarque}



On termine la section par la réciproque du corollaire précédent, qui établit entre autres l'équivalence entre bijectivité et existence d'une réciproque.

\begin{proposition}
\label{bijective_ssi_reciproque}
Soit $f : E\to F$ entre ensembles non vides.
\begin{enumerate}
\item Elle est surjective ssi elle admet une section.
\item Elle est injective ssi elle admet une rétraction.
\item Elle est bijective ssi elle admet une réciproque.
\end{enumerate}
\end{proposition}
\begin{proof}\index{section}
Le sens \og si\fg{} a été démontré dans le corollaire \ref{bijective-si-reciproque}. Montrons le sens \og seulement si\fg.
\begin{enumerate}
\item  Pour tout $y\in F$, on choisit un antécédent de $y$ par $f$, que l'on note $x_y$. On définit alors une fonction $g : F\to E$ par $g(y)=x_y$. Par construction, on a $f\circ g = \Id_F$ donc $g$ est une section de $f$.
\item À tout $y\in F$ on associe soit son unique antécédent s'il en existe un, soit un élément de $E$ arbitraire dans le cas contraire (on peut le faire puisque $E$ n'est pas vide). Ceci définit une fonction $g : F\to E$ et par construction on a $g\circ f=\Id_E$.
\item D'une part, comme $f$ est surjective, elle admet (d'après le premier point) une section $s$, qui vérifie donc $f\circ s = \Id_F$. Montrons  que $s\circ f = \Id_E$. Soit $x\in E$ et soit $a = (s\circ f)(x)$. Alors $f(a) = (f\circ s \circ f) (x) = ((f\circ s)\circ f)(x) = f(x)$ et comme $f$ est injective, $a=x$ c'est-à-dire $(s\circ f)(x) = x$. Donc $s\circ f=\Id_E$ et donc $s$ est la réciproque de $f$.
\item \emph{Preuve alternative du dernier point, en suivant le cheminement inverse.} D'une part, comme $f$ est injective, elle admet (d'après le second point) une rétraction $r$, qui vérifie donc $r\circ f=\Id_E$. Montrons  que $f\circ r = \Id_F$. Soit $y\in F$. Comme $f$ est surjective, considérons $x$ un antécédent de $y$. Alors, $(f\circ r)(y) = (f\circ r)(f(x)) =(f\circ (r \circ f)(x) = (f\circ \Id_E)(x) = f(x) = y$. D'où $f\circ r = \Id_F$ et donc $r$ est la réciproque de $f$.
\end{enumerate}
\end{proof}

\begin{exercice}[Zérologie]\index{zérologie} Si les ensembles $E$ ou $F$ ne sont pas supposés non-vides, quelles parties de cette proposition restent vraies ?
\end{exercice}



%--------------------------------------
\section{Images directes et réciproques de parties}
%--------------------------------------





\emph{Résumé : images directes et réciproques de parties, fibres d'une application, union et intersections d'images directes et réciproques.}


\begin{definition}[Image directe d'une partie]
\index{image!directe d'une partie}
Soit $f : E\to F$ et $A\subseteq E$. On appelle image directe de $A$ et on note $f_*(A)$ l'ensemble des images des éléments de $A$ :
\[f_*(A) := \{f(x)\:\mid\: x\in A\} = \{y\in F \:\mid\: \exists x\in A, y=f(x)\}.\]
Autrement dit, pour un élément $y\in F$, on a $y\in f_*(A) \iff \left(\exists x\in A, y=f(x)\right)$.
\end{definition}

\begin{definition}[Application image directe]
Soit $f : E\to F$. L'application de $\mathcal P(E)$ dans $\mathcal P(F)$, qui à $A\in \mathcal P(E)$ associe $f_*(A)\in \mathcal P(F)$, est appelée \og l'image directe par $f$\fg, et est notée $f_*$ (ce qui est cohérent avec les notations choisies auparavant).
\end{definition}


\begin{definition}[Image réciproque d'une partie]
\index{image!réciproque d'une partie}
Soit $f : E\to F$ et $B\subseteq F$. On appelle image réciproque de $B$ l'ensemble de tous les antécédents d'éléments de $B$ :
\[f^*(B) = \{x\in E\:\mid\: f(x)\in B \}.\]
Autrement dit, pour un élément $x\in E$, on a $x\in f^*(B) \iff f(x)\in B$.

\end{definition}

\begin{definition}[Application image réciproque]
Soit $f : E\to F$. L'application de $\mathcal P(F)$ dans $\mathcal P(E)$, qui à $B\in \mathcal P(F)$ associe $f^*(B)\in \mathcal P(E)$, est appelée \og l'image réciproque par $f$\fg, et est notée $f^*$ (ce qui est cohérent avec les notations choisies auparavant).
\end{definition}

Pour récapituler, nous avons donc les diagrammes suivants:
\begin{center}
$\xymatrix{E\ar[rr]^{f} & & F}$, et 
$\xymatrix{
\mathcal P(E) \ar@/^/[rr]^{f_*}& & \mathcal P(F) \ar@/^/[ll]^{f^*}
}$.
\end{center}


\begin{attention}
Quelle que soit l'application $f$, les applications $f_*$ et $f^*$ sont bien définies, même si $f$ ne possède pas d'application réciproque (notée $f^{-1}$).
\end{attention}


\begin{attention}
Les deux notations $f_*(A)$ et $f^*(B)$ ne sont (malheureusement) \textbf{pas  standard} (à ce niveau)
.
\begin{enumerate}
\item Dans la plupart des ouvrages, l'image directe d'une partie est simplement notée $f(A)$. Autrement dit, l'application $f_*$ est simplement notée $f$ par abus de langage, même si formellement ce n'est pas la même application (elle ne prend pas le même type d'argument par exemple).
\item 
L'image réciproque d'une partie $B$ par $f$ est, elle, souvent notée $f^{-1}(B)$, ce qui constitue un abus de langage encore plus dommageable car il entre en collision avec la notation pour la fonction réciproque qui elle, n'existe pas toujours. Ceci provoquant de nombreuses erreurs, il est fréquent que les enseignants utilisent la notation $f^{<-1>}(B)$ pour les images réciproques, pour bien distinguer la notion de cette d'application réciproque, mais cette notation est moins naturelle et universelle que celle utilisée ici. L'abus de langage sur l'image directe, lui, est en général considéré comme bénin.
\end{enumerate}

Cela dit, dans ce cours, on utilisera exclusivement les notations non ambigues $f_*$ et $f^*$, essentiellement à des fins pédagogiques. Ces notations disparaitront à court terme et les deux abus de langage présentés plus haut prendront le dessus, mais elles referont leur apparition en M1 ou M2 dans d'autres contextes proches, où l'abus de notation redevient trop nocif.
\end{attention}

\begin{exemple}\index{image}
Si $f : E\to F$, alors $f_*(E)$ est juste l'image de $f$, introduite plus haut et notée $\Im f$. La notion d'image directe d'une partie est donc une généralisation de celle d'image d'une application.
\end{exemple}

\begin{exercice}
Soit $f : E\to F$. On fixe une partie $A\subseteq E$ et on note $i : A\to E$ l'inclusion de $A$ dans $E$. Vérifier que $f_*(A) = \Im f\circ i$.
\end{exercice}



\begin{proposition}
Soit $f : E\to F$.
\begin{enumerate}
\item $f_*(\varnothing)=\varnothing$.
\item $f_*(E)\subseteq F$ avec égalité si et seulement si $f$ est surjective. En général, $f_*(E)\neq F$.
\item Si $(A_i)_{i\in I}$ est une famille de parties de $E$, alors $f_*\left(\bigcup_{i\in I}A_i\right) = \bigcup_{i\in I} f_*(A_i)$.
\item Par contre, on a en général $f_*\left(\bigcap_{i\in I}A_i\right) \subseteq \bigcap_{i\in I} f_*\left(A_i\right)$ mais pas forcément égalité.
\end{enumerate}
\end{proposition}
\begin{proof}
\begin{enumerate}
\item Clair.
\item On a $f_*(E)=F \iff (\forall y\in F, y\in f_*(E))$ ce qui signifie par définition que tout élément $y\in F$ admet au moins un antécédent, et donc que $f$ est surjective.  
\item Soit $y\in F$. On a 
\begin{align*}
y\in f_*\left( \bigcup_{i\in I} A_i \right)
&\iff \exists x\in  \bigcup_{i\in I} A_i,   y=f(x) \\
&\iff \exists x\in E, \left( x\in \bigcup_{i\in I} A_i\text{ et }  y=f(x)\right)\\
&\iff \exists x\in E, \exists i\in I,  (x\in A_i\text{ et } y=f(x)) \\
&\boxed{\iff \exists i\in I, \exists x\in E}\:, (x\in A_i\text{ et } y=f(x))\\
&\iff \exists i\in I, \exists x\in A_i, y=f(x)\\
&\iff \exists i\in I, y\in f_*(A_i)\\
&\iff y\in \bigcup_{i\in I} f_*(A_i)
\end{align*}
L'interversion de quantificateur signalée est licite car ce sont deux quantificateurs existentiels.
\item On a:
\begin{align*}
y\in f_*\left( \bigcap_{i\in I} A_i \right)
&\iff \exists x\in  \bigcap_{i\in I} A_i,   y=f(x) \\
&\iff \exists x\in E, \left( x\in \bigcap_{i\in I} A_i\text{ et }  y=f(x)\right)\\
&\iff \exists x\in E, \forall i\in I,  (x\in A_i\text{ et } y=f(x)) \\
&\boxed{\implies \forall i\in I, \exists x\in E}\:, (x\in A_i\text{ et } y=f(x)) \quad(*)\\
&\implies \forall i\in I, \exists x\in A_i, y=f(x)\\
&\implies \forall i\in I, y\in f_*(A_i)\\
&\implies y\in \bigcap_{i\in I} f_*(A_i)
\end{align*}
L'interversion des quantificateurs  est licite \textbf{dans ce sens-là seulement : $\exists x \forall i ... \implies \forall i \exists x ...$}, et l'équivalence devient une implication. Ceci prouve l'inclusion. Pour montrer qu'il n'y a pas forcément égalité, il suffit de donner un contre-exemple, par exemple $\sin_*(\R_-^* \cap \R_+^*) = \sin_*(\varnothing) = \varnothing \subsetneq \sin(\R_+^*) \cap \sin_*(\R_-^*)=[-1,1]$.
\end{enumerate}
\end{proof}



\begin{exemple}[Fibres et images réciproques]
\index{fibre}
Soit $f : A\to B$ et $b\in B$. On a vu que la \emph{fibre de $f$ (au-dessus) de $b$} est par définition l'ensemble des antécédents de $b$, c'est-à-dire $\{a\in A \:\mid\: f(a)=b\} $.

En reformulant, la fibre de $f$ au-dessus de $b$ est donc l'ensemble $f^*(\{b\})$. De façon générale, les \emph{fibres} de l'application $f$ sont les images réciproques de singletons de $B$.
\end{exemple}

\begin{exemple}
Soient $f : A\to B$,  $g : B\to C$ et $C'\subseteq C$ une partie de $C$. Alors, on a 
\[\Im g\circ f \subseteq C' \iff \Im f \subseteq g^*(C').\]
\end{exemple}

\begin{proposition}
Soit $f : E\to F$.
\begin{enumerate}
\item $f^*(\varnothing)=\varnothing$.
\item $f^*(F)=E$.
\item Si $(B_i)_{i\in I}$ est une famille de parties de $F$, alors 
\[
f^*\left(\bigcup_{i\in I}B_i\right) 
= \bigcup_{i\in I} f^*(B_i)
\quad \text{ et } \quad 
f^*\left(\bigcap_{i\in I}B_i\right) 
= \bigcap_{i\in I} f^*(B_i)/
\]
\end{enumerate}
\end{proposition}
\begin{proof} Exercice.
\end{proof}

\begin{quote}
\emph{En conclusion, l'image réciproque se comporte mieux que l'image directe vis-à-vis des unions et intersections.}
\end{quote}

%- - - - - -
% factorisations


\section{Compléments : principes de factorisation}

Les définitions et résultats de cette section ne sont pas exigibles comme cours à l'examen mais sont néanmoins importants.

\begin{proposition}[Principe de factorisation à droite]
\index{factorisation!à droite}
\label{prop-CNS-factorisation-droite}
Soient $f : A\to B$ et $g : A\to C$. Rappelons que les trois assertions suivantes sont équivalentes:
\begin{enumerate}
\item  $g$ se factorise à droite par $f$;
\item il existe une application $h : B\to C$ telle que $g = h\circ f$;
\item il existe une application $h : B\to C$ faisant commuter  le diagramme \index{diagramme!commutatif} 
$\xymatrix{
A \ar[r]^g \ar[d]_{f}& C\\
B \ar@{-->}[ur]_{\exists h}& 
}$.
\end{enumerate}
Ces assertions sont vraies si et seulement si :
\[\forall x, y \in A, f(x)=f(y) \implies g(x)=g(y).\]
\end{proposition}
\begin{proof}
\textbf{Sens \og seulement si\fg.} Supposons que $g$ se factorise en $g = h\circ f$. Soient $x, y\in E$ tels que $f(x)=f(y)$. En composant à gauche par $h$, on obtient $h(f(x)) = h(f(y))$, c'est-à-dire $g(x)=g(y)$.\\
\textbf{Sens \og si\fg.} Supposons que $\forall x, y \in A, f(x)=f(y) \implies g(x)=g(y)$. Construisons une fonction $h$ vérifiant les conditions demandées. Soit $b\in B$. S'il existe $a\in A$ tel que $f(a)=b$, on définit $h(b)$ comme étant égal à $g(a)$. Sinon, on définit $h(b)$ comme étant un élément quelconque de $C$. En définissant ainsi un élément $h(b)\in C$ pour tout élément $b\in B$, on définit donc une fonction $h : B\to C$ et par construction, on a $g = h\circ f$.
\end{proof}

\begin{remarque} Si $A=C$ et $g=\Id_A$, cette proposition devient l'équivalence entre injectivité et existence d'une rétraction $h$ (inverse à gauche).
\end{remarque}

Cette proposition peut paraître abstraite mais en pratique elle est très facile à utiliser, comme le montre l'exemple suivant.
\begin{exemple}
Montrer que $g : \R\to\R, x\mapsto |x|$ se factorise à droite par $f : \R\to \R, x\mapsto x^2$.
\end{exemple}
\begin{red}
Soient $x, y\in \R$. Supposons $f(x)=f(y)$ c'est-à-dire $x^2=y^2$. Alors, $x=y$ ou $x=-y$, et donc on en déduit que $|x|=|y|$ c'est-à-dire $g(x)=g(y)$. D'après la proposition \ref{prop-CNS-factorisation-droite}, il existe $h : \R\to \R$ telle que $g =h\circ f$, ou encore : $\forall x\in \R, |x|=h(x^2)$. 
\end{red}

(Attention, dans l'exemple précédent $h$ n'est pas exactement la racine carrée : en effet $h$ doit être définie sur $\R$. D'ailleurs $h$ n'est pas unique.)




\begin{proposition}[Principe de factorisation à gauche]
\index{factorisation!à gauche}
\label{prop-CNS-factorisation-gauche}
Soient $f : A\to C$ et $g : B\to C$. Rappelons que les trois assertions suivantes sont équivalentes:
\begin{enumerate}
\item  $f$ se factorise à gauche par $g$;
\item il existe une application $h : A\to B$ telle que $f = g\circ h$;
\item il existe une application $h : A\to B$ faisant commuter le diagramme \index{diagramme!commutatif} 
$\xymatrix{
&B \ar[d]^{g} \\
A \ar[r]_f \ar@{-->}[ur]^{\exists h}& C
}$.
\end{enumerate}
Ces assertions sont vraies si et seulement si $f_*(A)\subseteq g_*(B)$.
\end{proposition}
\begin{proof}
\textbf{Sens \og seulement si\fg.} Supposons $f = g\circ h$, et soit $a\in A$. Alors $f(a) = g(h(a))$, donc $f(a) \in g_*(B)$. Ceci prouve que $f_*(A)\subseteq g_*(B)$.\\
\textbf{Sens \og si\fg.} Supposons $f_*(A)\subseteq g_*(B)$. Construisons une application $h$ vérifiant les conditions demandées. Soit $a\in A$. Comme $f(a) \in f_*(A) \subseteq g_*(B)$, $f(a) \in g_*(B)$ et donc $f(a)$ possède un antécédent par $g$. On définit $h(a)$ comme étant un tel antécédent. Par construction, on a $g\circ h = f$.
\end{proof}

\begin{remarque}
Si $A = C$ et que $f = \Id_C$, on retombe sur l'équivalence entre surjectivité et existence d'une section (inverse à droite).
\end{remarque}

\begin{exercice} 
(Dans cet exercice les fonctions vont de $\R$ dans $\R$.) 
Montrer l'application $x\mapsto x^2+6x+10$ se factorise à droite par $x\mapsto x+3$.
% l'autre facteur est $x\mapsto x^2+1$
 Montrer qu'elle se factorise à gauche par $x\mapsto x+5$.
% l'autre facteur est $x\mapsto x^2+6x+5$.
\end{exercice}






\section{Compléments : fonctions ayant même source et but}



Les définitions et résultats qui suivent concernent les fonctions ayant même source et but, autrement dit les fonctions d'un ensemble $E$ dans lui-même. De telles fonctions sont parfois appelées \emph{endofonctions
\footnote{Du grec ancien éndon (« dans »). Sert à construire des mots avec l’idée d’intérieur. (Source : Wiktionnaire.) En algèbre linéaire, on croisera fréquemment des \emph{endomorphismes}.}
 (de l'ensemble $E$)}.


\subsection{Permutations}

\begin{definition}\index{permutation}
Une \emph{permutation} d'un ensemble $E$ est une bijection de $E$ dans $E$. 
\end{definition}

L'ensemble de toutes les permutations de $E$ est noté $\Bij(E)$, ou encore $\mathfrak S_E$ ($S$ capitale gothique, en police \emph{fraktur}).

L'ensemble $\Bij(E)$ est donc une partie de $E^E$, l'ensemble de toutes les fonctions de $E$ dans $E$. Il vérifie les trois propriétés suivantes:
\begin{enumerate}
\item L'application identité $\Id_E$ appartient à $\Bij(E)$.
\item Si $f$ et $g$ appartiennent à $\Bij(E)$, alors $f\circ g$ aussi.
\item Si $f$ appartient à $\Bij(E)$, alors son application réciproque $f^{-1}$ aussi.
\end{enumerate}
Ces propriétés importantes (mais immédiates à vérifier dans le cas présent) sont celles qui donneront plus tard naissance au concept abstrait de \emph{groupe} (voir cours sur les structures algébriques usuelles : \emph{groupes, anneaux, corps}).


\subsection{Involutions}
\begin{definition}[Fonction involutive/involution]\index{application!involutive}\index{involution}
Soit $f : E\to E$ une fonction ayant même source et but. On dit que $f$ est \emph{involutive}, ou que c'est une \emph{involution}, si $f\circ f=\Id_E$.
\end{definition}

\begin{remarque}
Une involution est bijective puisqu'elle est sa propre fonction réciproque. Autrement dit une involution est une permutation.
\end{remarque}

\begin{exemple}\label{exo-involution-1-x}
La fonction $f : \R\to \R, x\mapsto 1-x$ est involutive.
\end{exemple}


\begin{exercice}\label{exo-involution-cubique}
Dans cet exercice on note $\sqrt[3]{}$ la racine cubique définie sur $\R$.
Montrer que les fonctions $g : \R\to \R, x\mapsto \sqrt[3]{1-x^3}$ est $h : \R\to \R, x\mapsto (1-\sqrt[3]{x})^3$ sont involutives. Quel est le rapport entre cet exercice et l'exercice \ref{exo-involution-1-x} ? Voir l'exercice \ref{exo-conjugaison} pour le lien général.

\begin{center}
  \begin{tikzpicture}[scale=1.5]
\draw[->,color=black] (-2,0) -- (2,0);
\draw[->,color=black] (0,-2) -- (0,2);
\clip(-2,-2) rectangle (2,2);

\draw[smooth,samples=100,domain=-2:1,thick,blue] plot(\x,{(1-(\x)^3)^(1/3)});
\draw[smooth,samples=100,domain=1:2,thick,blue] plot(\x,{-(-1+(\x)^3)^(1/3)});
\draw[dashed] (-2,-2) -- (2,2);
\end{tikzpicture}
\hfill
\begin{tikzpicture}[scale=.75]
\draw[->,color=black] (-2,0) -- (6,0);
\draw[->,color=black] (0,-2) -- (0,6);
\clip(-2,-2) rectangle (6,6);
\draw[smooth,samples=100,domain=-2:0,thick,blue] plot(\x,{(1+(-\x)^(1/3))^3});
\draw[smooth,samples=100,domain=0:6,thick,blue] plot(\x,{(1-(\x)^(1/3))^3});
\draw[dashed] (-2,-2) -- (6,6);
\end{tikzpicture}
\end{center}
\end{exercice}

\begin{exercice}\label{exo-involution-cases}
Montrer que $f:\R\to \R,\: x\mapsto \begin{cases} -x/2 & \text{ si }x\geq 0\\-2x & \text{ si }x<0\end{cases}$ est involutive.
\end{exercice}

\begin{exercice}\label{exo-involution-graphe-sym}
Montrer qu'une fonction $f:\R\to \R$ est involutive ssi son graphe $\Gamma_f \subseteq \R^2$ est symétrique par rapport à la droite d'équation $y=x$. 
\end{exercice}

\begin{exercice}\label{exo-involution-abs}
\begin{enumerate}
\item Montrer que $f : \R\to \R, x\mapsto \frac{|x|-x\sqrt 5}{2}$ est involutive.
\item Écrire $f$ de façon plus simple sur chacun des ensembles $\R_+$ et $\R_-$, comme dans l'exercice \ref{exo-involution-cases}.
\item Réciproquement, définir la fonction de l'exercice \ref{exo-involution-cases} à l'aide d'une seule formule valable sur $\R$, en utilisant des valeurs absolues.
\item Bonus : cet exercice a quelque chose à voir avec le fameux \emph{nombre d'or} $\phi = \frac{\sqrt 5 +1}{2}$. Se documenter sur le sujet et comprendre pourquoi, au-delà de la simple coïncidence des valeurs de l'énoncé.
\end{enumerate}
\end{exercice}



\subsection{Itérées, points fixes, parties stables}

% % % % % % % % 
\begin{definition}[itérées successives]\index{itérée}\index{itération}
Soit $f : E\to E$ une application d'un ensemble dans lui-même, et $n\in \N$. On  note $f^{\circ n}$ et on appelle \emph{$n$-ème itérée de $f$}  l'application définie par récurrence de la façon suivante :
\[ f^{\circ 0} = \Id_E \text{ et } \forall n\in \N, f^{\circ n+1} = f\circ f^{\circ n}.\]
Autrement dit, on a :
\[ 
f^{\circ n} = \begin{cases}
\Id_E & \text{ si } n=0;\\
\underbrace{f\circ \dots \circ f}_{n \text{ fois}} & \text{ si } n\geq 1.
\end{cases}
\]
Il arrive que l'on écrive $f^n$ au lieu de $f^{\circ n}$ bien que ce soit déconseillé, pour éviter la confusion entre produit et composition.
\end{definition}

\begin{definition}[Point fixe et lieu fixe]\label{def-point-fixe}
\index{point fixe}\index{lieu fixe}
Soit $f : E\to E$ une application d'un ensemble dans lui-même. Un élément $x\in E$ est dit \emph{fixe par $f$} (ou simplement \emph{fixe} s'il n'y a pas d'ambiguïté sur la fonction), si $f(x)=x$. On dit aussi que $x$ est un \emph{point fixe} de $f$. Le \emph{lieu fixe} de $f$ est la partie de $E$ constituée de tous les points fixes : $\Fix(f) = \{x\in E\:|\: f(x)=x \}$.
\end{definition}

\begin{exemple}
\begin{enumerate}
\item Soit $f : \R\to \R, x\mapsto x^2-2$. Ses points fixes sont $-1$ et $2$ (ce sont les solutions de l'équation $f(x)=x$). Son lieu fixe est donc $\{-1,2\}$.
\item L'application $\tau : \R^2\to \R^2, (x,y) \mapsto (x+2,y-3)$, n'a aucun point fixe. (C'est la translation de vecteur $\vec u = (2,-3)$.)
\item Soit $\Delta$ une droite du plan $\R^2$. La symétrie axiale $\sigma$ d'axe $\Delta$ a pour lieu fixe $\Fix(\sigma)=\Delta$.
\item De façon générale pour les transformations classiques du plan euclidien, une translation non triviale n'a aucun point fixe, une rotation non triviale a exactement un point fixe (le centre de la rotation), une symétrie axiale a une infinité de points fixes, dont l'union est l'axe de la symétrie.
\end{enumerate}
\end{exemple}

\begin{exercice}
Déterminer les points fixes de $f : \R\to \R, x\mapsto x^3$.
\end{exercice}

% a besoin des images directes.
\begin{definition}[partie stable]\index{partie!stable}
Soit $f : E\to E$ une application d'un ensemble dans lui-même. Une partie $A\subseteq E$ est \emph{stable par $f$}, (ou \emph{$f$-stable}, ou simplement \emph{stable} s'il n'y a pas d'ambiguïté pour la fonction), si $f_*(A)\subseteq A$.
\end{definition}

\begin{exercice}
Soit $f : \R\to \R, x\mapsto x^3-x$. Montrer que $[-1,1]$ est stable.
\end{exercice}

\begin{exercice}[Conjugaison\footnote{(aucun lien avec la conjugaison complexe malgré le nom). On rencontrera souvent ce type de situation, en particulier en algèbre linéaire, puis en algèbre générale.} par une bijection]
\index{conjugaison!par une bijection}\label{exo-conjugaison}
Soit $E$ un ensemble, $f \in E^E$ une endofonction de $E$, et $g \in \Bij(E)$. L'application $g\circ f \circ g^{-1}$ est appelée la \emph{conjuguée de $f$ par $g$} (on \emph{conjugue} $f$ par la bijection $g$). On peut visualiser ce que fait cette application grâce au diagramme suivant (attention au sens des flèches):
\[\xymatrix{
E \ar[d]^g \ar[r]^f & E \ar[d]^g\\
E \ar[r]_{g\circ f\circ g^{-1}}& E
}\]
\begin{enumerate}
\item Montrer que si $f$ est une involution, alors $g\circ f \circ g^{-1}$ aussi. Quel est le rapport avec les exercices \ref{exo-involution-1-x} et \ref{exo-involution-cubique} (écrire les fonctions sous forme de composées) ?
\item Montrer que si $x\in E$ est fixe par $f$, alors $g(x)$ est fixe par $g\circ f \circ g^{-1}$.
\item Montrer que si $A \subseteq E$ est stable par $f$, alors $g_*(A) \subseteq E$ est stable par $g\circ f \circ g^{-1}$.
\end{enumerate}
\end{exercice}

\begin{definition}[Point cyclique]\index{point cyclique}\label{def-point-cyclique}
Soit $f : E\to E$ une application d'un ensemble dans lui-même. Un élément $x\in E$ est dit \emph{cyclique par $f$} (ou simplement \emph{cyclique} s'il n'y a pas d'ambiguïté sur la fonction), s'il existe $k\in\N^*$ tel que $f^{\circ k}(x)=x$. 
\end{definition}

Cette notion généralise la notion de point fixe : un point fixe est cyclique, mais un point cyclique n'est pas forcément fixe.

\begin{exercice}
Soit $f : \R\to \R, x\mapsto -x$. Montrer que tous les points sont cycliques. Lesquels sont fixes ?
\end{exercice}



\section{Compléments: saturation}


\begin{propdef}[Saturation, partie saturée]
\index{partie!saturée}\index{saturation!relativement à une fonction}
Soit $f : E\to F$ et $A \subseteq E$ une partie de la source de $f$. La plus grande partie $B\subseteq E$ telle que $f_*(B) \subset f_*(A)$ est égale à $f^*(f_*A))$.

Cette partie est appelée la \emph{saturation} de $A$ (relativement à la fonction $f$). On dit que $A$ est \emph{saturée} (relativement à la fonction $f$) si elle est égale à sa saturation.
\end{propdef}
\begin{proof}
Par définition de $B$, on a $B = \{x\in E\:|\: f(x)\in f_*(A) \}$ et donc par définition d'une image réciproque, $B = f^*(f_*(A))$.
\end{proof}

\begin{exemple}
\begin{enumerate}
\item La saturation $\operatorname{Sat}_f(A)$ d'une partie $A$ contient toujours $A$ mais elle peut être plus grande (voir ci-dessous).
\item La saturation d'une partie $A$ est saturée, autrement dit avec la notation plus haut  on a  $\operatorname{Sat}_f(\operatorname{Sat}_f(A)) = \operatorname{Sat}_f(A)$. Reformulation : l'application $\operatorname{Sat}_f : \mathcal P(E) \to \mathcal P(E), A\mapsto \operatorname{Sat}_f(A)$ vérifie la propriété $\operatorname{Sat}_f \circ \operatorname{Sat}_f = \operatorname{Sat}_f$.
\item Soit $f : \R\to \R_+, x\mapsto x^2$, et soit $A=\{2\} \subseteq \R$. La saturation de $A$ est 
\[f^*(f_*(A)) = f^*(\{4\}) = \{x\in \R\:|\: x^2=4\} = \{-2,2\}
\]
La saturation de $A$ contient donc strictement $A$, qui n'est donc pas saturée. Par contre, les parties $B = \{0\}$ et $C=\{-2,2\}$ sont saturées (relativement à $f$, toujours).
\end{enumerate}
\end{exemple}

\begin{exercice}
Montrer qu'une application $f:E\to F$ est injective si et seulement si toute partie de $E$ est saturée relativement à $f$.
\end{exercice}







\section{Exercices d'approfondissement}

Les premiers exercices tournent autour des coordonnées sphériques. Les résultats sont des analogues de ceux sur les coordonnées polaires dans le plan, mais sont plus techniques à énoncer et à démontrer.


\begin{exercice}[Coordonnées sphériques sur la sphère]\index{coordonnées!sphériques}\label{exo-coord-spheriques-sphere}
On note $\S^2$ la sphère unité de $\R^3$, c'est-à-dire 
\[ \S^2 = \ensemble{(x,y,z)\in\R^3}{x^2+y^2+z^2=1}\]
Soit 
\[ f : 
\begin{cases}
 [-\pi,\pi] \times \left[-\frac{\pi}{2},\frac{\pi}{2}\right] &\longrightarrow \R^3\\
(\theta,\phi) &\mapsto (\cos\phi\cos\theta,\: \cos\phi\sin\theta, \: \sin\phi)
\end{cases}
\]
Montrer que l'image de $f$ est $\S^2$. Ceci signifie que l'on peut écrire:
\[ \S^2 = \ensemble{(\cos\phi\cos\theta,\: \cos\phi\sin\theta, \: \sin\phi)}{\theta\in [-\pi,\pi], \phi\in [-\pi/2,\pi/2]}
\]
On dit que l'application $f$ est le paramétrage de la sphère $\S^2$ par longitude et latitude. Remarquer que les deux pôles n'ont pas de longitude bien définie.
\end{exercice}

\begin{exercice}\label{exo-decomp-spherique}
Montrer que $g : \R_+\times \S^2 \to \R^3, \quad (r,(x,y,z))\mapsto (rx,ry,rz)$ est surjective.
\end{exercice}

\begin{exercice}[\og produit\fg{} de surjections]\label{exo-produit-surjections}
Soient $f:X\to Y$ et $g:X'\to Y'$ deux surjections. Montrer que l'application
\[ 
\Phi : X\times X' \to Y\times Y', (x,x') \mapsto (f(x),g(x'))
\]
est surjective.
\end{exercice}


\begin{exercice}[Coordonnées sphériques sur l'espace]\index{coordonnées!sphériques}\label{exo-coord-spheriques-espace}
Montrer que 
\[ \R^3 = \ensemble{(r\cos\theta\cos\phi, r\sin\theta\cos\phi, r\sin\phi)}{r\in\R_+, \: \theta \in [-\pi,\pi], \: \phi\in\left[-\frac{\pi}{2},\frac{\pi}{2}\right]}\]
Autrement dit, montrer que l'application 
\[\Phi : \R_+\times [-\pi,\pi]\times \left[-\frac{\pi}{2},\frac{\pi}{2}\right] \to \R^3, \quad 
(r,t) \mapsto (r\cos\theta\cos\phi, r\sin\theta\cos\phi, r\sin\phi)
\]
est surjective. Plutôt que d'utiliser un calcul direct, il est recommandé de l'écrire comme composée et d'utiliser la proposition \ref{prop-composition-inj-surj-bij}, ainsi que les exercices \ref{exo-coord-spheriques-sphere}, \ref{exo-decomp-spherique}  et \ref{exo-produit-surjections}.
C'est le paramétrage de $\R^3$ par coordonnées sphériques  ( par opposition aux coordonnées cartésiennes). Les conventions utilisées ici sont celles des mathématiciens et géographes (longitude et latitude), pas celles des physiciens (colatitude). Elles ont l'avantage d'être cohérentes avec les coordonnées polaires du plan, et les coordonnées cylindriques dans l'espace.
\end{exercice}

\begin{exercice}(Curryfication)\index{curryfication}
Soient $X$, $Y$ et $Z$ des ensembles. On définit une application $\phi : Z^{X\times Y} \to \left(Z^Y\right)^X$ comme suit. Si $f\in Z^{X\times Y}$ est une fonction, $\phi(f)$ est la fonction 
\[ \phi(f) : \begin{cases}
X \to &Z^Y\\
x\mapsto& \phi(f)(x):\begin{cases}Y&\to Z\\y&\mapsto f\left( (x,y) \right)\end{cases}
\end{cases}
\]
Montrer que $\phi$ est une bijection de $Z^{X\times Y} $ dans $\left(Z^Y\right)^X$. Indication : quelle est l'application réciproque ?
(En informatique, l'application $\phi$ est appelée opération de \emph{curryfication} : elle transforme une fonction de plusieurs variables en une fonction d'une seule variable, mais qui retourne une fonction. L'application réciproque est la \emph{décurryfication}.)
% voir https://fr.wikipedia.org/wiki/Curryfication
\end{exercice}

\begin{exercice}[Axiomes de recollement]
% source : Maxime 9932
Soit $E$ un ensemble et $A, B \in \mathcal P(E)$.

On considère l'application
\[ \Phi : \begin{cases}
\R^E &\to \R^A \times \R^B\\
f&\mapsto (f|_A, f|_B).
\end{cases}\]
\begin{enumerate}
\item Montrer que $\Phi$ est injective si et seulement si $A \cup B = E$.
\item Donner une condition nécessaire et suffisante pour que $\Phi$ soit surjective.
\item En général, décrire l'image de $\Phi$.
\end{enumerate}
\end{exercice}

% restriction, corestriction 
\begin{exercice}
Soit $f : E\to F$, et $B$ une partie de $F$ contenant toutes les images des éléments de $E$ (autrement dit, $\forall x\in E, f(x)\in B$).
Soit $i : B\to F$ l'inclusion de $B$ dans $F$, et soit $r : F\to B$ une fonction définie comme suit : pour $y\in F$, on définit $r(y)=y$ si $y\in B$, et sinon, $r(y)$ est un élément arbitraire de $B$. 
\begin{enumerate}
\item Vérifier que $r$ est une rétraction de $i$.
\item Vérifier que la corestriction $f|^{B}$ est égale à la composée $r\circ f$.
\end{enumerate}
\end{exercice}

\begin{exercice}[Applications simplifiables  à gauche]
Soit $f : X\to Y$ une application. On dit qu'elle est \emph{simplifiable à gauche} si pour tout ensemble $Z$ et pour tout couple d'applications $g$ et $h$ de $Z$ dans $X$, on a $(\: f\circ g = f\circ h \implies g=h \: )$.

Montrer qu'une application est injective si et seulement si elle est simplifiable à gauche, d'abord en utilisant exclusivement la définition d'application injective, puis en utilisant la caractérisation en termes de rétractions.
\end{exercice}

\begin{exercice}[Monomorphismes, épimorphismes, tiré en arrière]
% Maxime 5924
Soit $E$ et $F$ deux ensembles non vides et $G$ un ensemble ayant au moins deux éléments. Soit $f : E \to F$. On considère l'application
\[ \phi : \begin{cases}G^F&\to G^E\\ g&\mapsto g\circ f.\end{cases}\]
\begin{enumerate}
\item Montrer que $\phi$ est injective si et seulement si $f$ est surjective.
\item Donner une condition nécessaire et suffisante sur $g, h : F \to G$ pour que $\phi(g) = \phi(h)$.
\item Montrer que $\phi$ est surjective si et seulement si $f$ est injective.
  \end{enumerate}
(Remarque : l'application $\Phi$ est parfois notée $f^\#$, et appelée le \og tiré en arrière (de $F$ à $E$, par $f$)\fg. C'est juste la composition à droite par $f$, mais cette notation et ce nom sont plus parlants : lorsque l'on a une fonction de $F$ dans $G$, on peut la \og tirer en arrière à $E$\fg, ce qui donne une fonction de $E$ dans $G$.)
\end{exercice}

\begin{exercice}
Soient $f : A\to B$ et $g : A\to C$. On suppose que $g$ se factorise à droite par $f$.

Montrer que si $f$ est surjective, alors l'application $h$ qui factorise $g$ est forcément unique.
\end{exercice}

\begin{exercice}[Fonctions sous-modulaires]
Soit $E$ un ensemble, et $f : \mathcal P(E) \to \R$. Montrer que les deux conditions suivantes sont équivalentes:
\begin{enumerate}
\item Pour toutes parties $A$ et $B$ de $E$, on a 
\[
f(A\cup B) \leq f(A)+f(B)-f(A\cap B)
\]
\item Pour toutes parties $A$ et $B$ de $E$ avec $A\subseteq B$, et pour tout $x\in E\setminus B$,
\[
f(A\cup \{x\}) - f(A) \geq f(B\cup \{x\}) - f(B)
\]
\end{enumerate}
Une fonction qui vérifie ces conditions est dite \emph{sous-modulaire}.
\end{exercice}

% exercices sur les probas sur ensembles finis ? Sur l'entropie ?



Les deux exercices suivants montrent que l'ensemble des surjections et l'ensemble des injections forment ce qui est parfois appelé un \emph{système orthogonal de factorisation}, une notion que l'on recroise dans plusieurs contextes en mathématiques.

\begin{exercice}[Factorisation en surjection puis injection]
Montrer que toute application se factorise en une surjection suivie d'une injection. Autrement dit, montrer que pour toute fonction $f : E\to F$, il existe un ensemble $G$, et des fonctions $\phi : E\to G$ et $\psi : G\to F$ telles que $f = \psi\circ \phi$, avec $\phi$ surjective et $\psi$ injective.
\end{exercice}

\begin{exercice}[Propriété du relèvement]
Soient $e : A\to B$ et $m : C\to D$ deux applications. On dit que $e$ est orthogonale (à gauche) à $m$ (ou que $m$ est orthogonale à droite à $e$), et on note $e\perp m$, si pour tout diagramme commutatif
\[
\xymatrix{
A \ar[r] \ar[d]_{e}& C\ar[d]^{m}\\
B \ar[r]& D 
},
\]
il existe un relèvement c'est-à-dire une application $r : B\to C$ faisant commuter le diagramme :
\[
\xymatrix{
A \ar[r] \ar[d]_{e}& C\ar[d]^{m}\\
B \ar[r]\ar[ur]^{r}& D 
}
\]
Montrer que si $e$ est surjective et $m$ est injective, alors $e\perp m$.
\end{exercice}

Les deux exercices qui suivent donnent un éclairage alternatif sur le produit cartésien et sa contrepartie, la \og somme cartésienne\og{}, aussi appelée \og union disjointe\fg.

\begin{exercice}[Propriété universelle du produit]\index{propriété universelle!du produit}
Soient $A$ et $B$ des ensembles, et $ A\times B$ leur produit (cartésien). On note $\operatorname{pr}_A : A\times B \to A, (a,b)\mapsto a$ et $\operatorname{pr}_B : A\times B \to B, (a,b)\mapsto b$. Ces deux applications sont appelées les \emph{projections canoniques sur les facteurs du produit}.

Montrer que l'ensemble $A\times B$ ainsi que les deux projections vérifient la propriété suivante (dite \emph{propriété universelle du produit}):
\begin{quote}
Pour tout ensemble $D$ muni d'applications $f$ et $g$ vers (respectivement) $A$ et $B$, il existe une unique application $h : D\to A\times B$ telle que $f =\operatorname{pr}_A \circ  h$ et $g =\operatorname{pr}_B \circ  h$, c'est-à-dire qu'il existe une unique application $h : D\to A\times B$ faisant commuter le diagramme suivant:
\[ 
\xymatrix{
& & & A\\
D \ar@{-->}[rr]^{\exists !h} \ar[urrr]^{f} \ar[drrr]_{g} & & A\times B \ar[ur]_{pr_A} \ar[dr]^{pr_B}& \\
& & & B
}\]
\end{quote}
\end{exercice}

\begin{exercice}[Union disjointe]\label{exo-coproduit}
\index{union disjointe}\index{propriété universelle!de l'union disjointe}
Soient $A$ et $B$ deux ensembles. On appelle \emph{union disjointe  de $A$ et $B$} l'ensemble
\[
A\coprod B = (\{0\}\times A) \bigcup (\{1\}\times B)
\]
Cet ensemble est muni de deux applications $i_A : A\to A\coprod B, a\mapsto (0,a)$ et  $i_B : B\to A\coprod B, b\mapsto (1,b)$.

Montrer que $A\coprod B$ ainsi que $i_A$ et $i_B$ vérifient la propriété suivante (dite \emph{propriété universelle  de l'union disjointe}): 
\begin{quote}
Pour tout ensemble  $C$ muni d'applications $f : A\to C$ et $g : B\to C$, il existe une unique application $h : A\coprod B \to C$ telle que $f = h\circ i_A$ et $g = h\circ i_B$, autrement dit il existe une unique application $h$ faisant commuter le diagramme  suivant:
\[ 
\xymatrix{
A \ar[dr]_{i_A} \ar[drrr]^{f}& & & \\
&  A\coprod B \ar@{-->}[rr]^{\exists !h}  & & C\\
B \ar[ur]^{i_B} \ar[urrr]_{g}& & &
}\]
\end{quote}
(Remarque : l'union disjointe de $A$ et de $B$ est aussi souvent notée  $A\bigsqcup B$.)
\end{exercice}


\begin{exercice}[Égalisateurs]\label{exo-egalisateur}
\index{égalisateur}\index{propriété universelle!de l'égalisateur}
Soient $A$ et $B$ deux ensembles et $f$ et $g$ deux applications entre $A$ et $B$. Soit $E = \{x\in A\:|\: f(x)=g(x)\}$ le lieu d'égalité des deux fonctions. On appelle cette partie l'\emph{égalisateur de $f$ et $g$}. On note de plus $i_E : E\to A$ l'application d'inclusion de $E$ dans $A$. On a donc $f\circ i_E = g\circ i_E$, ce que l'on peut résumer par le fait que le diagramme suivant commute:
\[
\xymatrix{
E \ar[r]_{i_E}& A\ar@/^/[r] ^{f} \ar@/_/[r]_{g} & B
}
\]
Montrer que $E$ et $i_E$ vérifient la propriété suivante (dite \emph{propriété universelle de l'égalisateur}):
\begin{quote}
Pour tout ensemble $X$ et application $\phi : X\to A$ vérifiant $f\circ \phi = g\circ \phi$,  il existe une unique application $h : X\to E$ telle que $\phi = i_E\circ h $. Autrement dit, il existe une unique application $h : X\to E$ faisant commuter le diagramme 
\[
\xymatrix{
& X \ar[d]^{\phi} \ar@{-->}_{\exists ! h}[dl]& \\
E \ar[r]_{i_E}& A\ar@/^/[r] ^{f} \ar@/_/[r]_{g} & B
}
\]
\end{quote}
\end{exercice}



\begin{exercice}[Produit fibré]\label{exo-produit-fibre}
\index{propriété universelle!du produit fibré}\index{produit fibré d'ensembles}
Soient $A$, $B$ et $C$ des ensembles et $\phi : A\to C$, $\psi : B\to C$ des applications.

Le \emph{produit fibré de $A$ et $B$ au-dessus de $C$ (les applications $\phi$ et $\psi$ étant sous-entendues)} est l'ensemble:
\[
A\times_C B := \{(x,y)\in A\times B\:|\: \phi(x)=\psi(y)\}
\]

On note $\operatorname{pr}_A : A\times_C B \to A, (a,b)\mapsto a$ et $\operatorname{pr}_B : A\times_C B \to B, (a,b)\mapsto b$. Ces deux applications sont appelées les \emph{projections canoniques sur les facteurs du produit fibré}. Par construction, on a donc $\phi\circ \operatorname{pr}_A = \psi\circ \operatorname{pr}_B$, autrement dit le diagramme suivant commute:
\[ 
\xymatrix{
 & A \ar[dr]^{\phi} &\\
 A\times_C B \ar[ur]_{pr_A} \ar[dr]^{pr_B}& & C\\
 & B \ar[ur]_{\psi}&
}\]

\begin{enumerate}
\item (Mise en garde de zérologie) Trouver un exemple d'ensembles $A$, $B$ et $C$ non vides et  d'applications $\phi$ et $\psi$ comme plus haut, tels que $A\times_C B$ soit vide.
\item Montrer que l'ensemble $A\times_C B$ ainsi que les deux projections vérifient la propriété suivante (dite \emph{propriété universelle du produit fibré}):
\begin{quote}
Pour tout ensemble $D$ muni d'applications $f$ et $g$ vers (respectivement) $A$ et $B$ vérifiant $\phi \circ f =  \psi\circ g$, il existe une unique application $h : D\to A\times_C B$ telle que $f =\operatorname{pr}_A \circ  h$ et $g =\operatorname{pr}_B \circ  h$, c'est-à-dire qu'il existe une unique application $h : D\to A\times_C B$ faisant commuter le diagramme suivant:
\[ 
\xymatrix{
& & & A \ar[dr]^{\phi} &\\
D \ar@{-->}[rr]^{\exists !h} \ar[urrr]^{f} \ar[drrr]_{g} & & A\times_C B \ar[ur]_{pr_A} \ar[dr]^{pr_B}& & C\\
& & & B \ar[ur]_{\psi}&
}\]
\end{quote}
\end{enumerate}
\end{exercice}

% produits cofibrés et coégalisateurs après les ensembles quotients


\begin{exercice}
Soient $A$, $B$ et $C$ des ensembles et $\phi : A\to C$, $\psi : B\to C$ des applications. On note $\operatorname{p}_A$ et $\operatorname{p}_A$ les surjections canoniques de $A\times B$ vers ses deux facteurs $A$ et $B$. Soit $E$ l'égalisateur de $\phi \circ \operatorname{p}_A$ et $\psi \circ \operatorname{p}_B$.

Montrer qu'il existe une bijection entre $E$ et $A\times_C B$ sans utiliser les définitions explicites de $E$ ni de $A\times_C B$, uniquement leur propriétés universelles.
% en fait vu les définitions, on a égalité ensembliste.
\end{exercice}

\begin{exercice}[Fibre et produit fibré]
Soit $f : A\to C$.
\begin{enumerate}
\item Si $c\in C$, montrer que la fibre de $f$ au-dessus de $c$ est exactement le produit fibré $A\times_C \{c\}$, où les applications de $A$ et $\{c\}$ vers $C$ sont $f$ et l'inclusion.
\item De manière plus générale, si $B\subseteq C$ est une partie, montrer que l'image réciproque de $B$ est exactement $A\times_C B$.
\end{enumerate}
Les notions de fibre et d'image réciproque sont donc des cas particuliers de la notion de produit fibré.
\end{exercice}

\begin{exercice}
Existe-t-il une bijection $\phi : \N\to \N$ sans point cyclique ? % oui : bijection avec $\Z$, puis translation.
\end{exercice}

% exercice : défintion d'un graphe, orienté ou non orienté.

% exercice : définition d'une catégorie.

% exercice : définition d'une loi de composition



\chapter{Entiers, ensembles finis et combinatoire}
\minitoc
\hyperlink{toc}{\retourTOC}

\section{L'ensemble $\N$ et la récurrence}
\index{$\N$}

L'ensemble $\N$ est supposé connu. Il vérifie les propriétés suivantes (les ensembles ordonnés sont traités dans un chapitre ultérieur mais le vocabulaire devrait être connu):
\begin{enumerate}
\item tout partie non vide majorée admet un plus grand élément;
\item toute partie non vide admet un plus petit élément.
\end{enumerate}

C'est la deuxième propriété qui permet de distinguer $\N$ de $\Z$ et qui fonde le principe de récurrence.


\begin{definition}[Propriété héréditaire]
Soit $A(n)$ une assertion dépendant d'un paramètre $n\in \N$. 
On dit qu'elle est \emph{héréditaire} si:
\[
\forall n\in \N,\: A(n)\implies A(n+1).
\]
(On définit de manière similaire l'hérédité à partir d'un certain rang $n_0$, au lieu du rang $0$.)
\end{definition}

\begin{theoreme}[Principe de récurrence]
\index{principe de récurrence}
Soit $A(n)$ une propriété dépendant d'un paramètre $n\in \N$. 

Si $A(0)$ est vraie et que la propriété est héréditaire, alors la propriété est vraie pour tout $n\in\N$, autrement dit:
\[
\big( A(0) \text{ et }(\forall n\in \N,\: A(n)\implies A(n+1))\big) \implies (\forall n\in \N, \: A(n)).
\]
\end{theoreme}
\begin{proof}Supposons la propriété vraie au rang $0$ et héréditaire.

Montrons que la partie $A = \{n\in \N\:\mid\: A(n)\text{ est fausse}\} \subseteq \N$ est vide.

Si $A$ est non-vide, elle possède un plus petit élément $m$ avec $m\geq 1$ puisque $A(0)$ est vraie. Donc $m-1 \in \N$ et $A(m-1)$ est vraie par définition de $m$. Par hérédité, $A(m)=A((m-1)+1)$ est vraie, contradiction. Donc $A=\varnothing$ ce qui signifie exactement: $\forall \in\N,\:A(n)$.
\end{proof}

\begin{theoreme}[Récurrence forte]\index{principe de récurrence!forte}
Soit $A(n)$ une propriété dépendant d'un paramètre $n\in \N$. 

Si $A(0)$ est vraie et que $\forall n\in \N, (\forall k\leq n, A(k))\implies A(n+1)$, alors $A(n)$ est vraie pour tout $n\in N$.
\end{theoreme}
\begin{proof}
Exercice. Appliquer le principe de  récurrence simple à la propriété $B(n)$ : \og $\big(\forall k\leq n,\: A(k)\big)$\fg.
\end{proof}
\section{Ensembles finis}

\subsection{Ensembles $\llbracket a,b\rrbracket$}
\index{$\llbracket a,b\rrbracket$}

Si $a, b\in \Z$, on note $\llbracket a,b\rrbracket$ l'ensemble $\{n\in \Z\:\mid\: a\leq n \leq b\}$. Si $b<a$, cet ensemble est vide.

\begin{lemme} Soient $m\geq 1$ et $a\in \llbracket 1,m\rrbracket$. L'application 
\[
\phi : \llbracket 1,m\rrbracket \setminus \{a\} \to \llbracket 1,m-1\rrbracket, 
x\mapsto \begin{cases}x & \text{ si }x<a\\x-1 & \text{ si }x>a\end{cases}
\]
est une bijection
\end{lemme}
\begin{proof} Exercice. Remarquer que si $m=1$, on obtient juste une bijection entre l'ensemble vide et lui-même.
\end{proof}


Les deux lemmes suivants établissent des résultats qui semblent \og évident\fg{} mais qui doivent être démontrés rigoureusement afin d'asseoir la définition de cardinal sur des bases solides.

\begin{lemme}
Soient $m,n\in \N$.
S'il existe une injection de $\llbracket 1,m\rrbracket$ dans $\llbracket 1,n\rrbracket$, alors $m\leq n$.
\end{lemme}
\begin{proof}
Pour $m$ entier, notons $A(m)$ l'assertion 
\begin{center}
$\forall n\in \N, \quad  (\text{il existe une injection } \llbracket 1,m\rrbracket \to \llbracket 1,n\rrbracket) \implies m\leq n$.
\end{center}
Montrons $\forall m, A(m)$ par récurrence, ce qui prouve la proposition.\\
\textbf{Initialisation.} $A(0)$ est vraie car pour tout $n$, $0\leq n$ est vraie donc l'implication dans $A(0)$ est vraie.\\
\textbf{Hérédité.} Soit $m\in\N$ et supposons $A(m)$. Montrons $A(m+1)$.

Soit $n\in \N$ et soit $f : \llbracket 1,m+1\rrbracket \to \llbracket 1,n\rrbracket$ une injection. Alors la restriction $f|_{\llbracket 1,m\rrbracket} : \llbracket 1,m\rrbracket \to \llbracket 1,n\rrbracket\setminus \{f(m+1)\}$ est également injective.

En composant avec une bijection $\phi : \llbracket 1,n\rrbracket\setminus \{f(m+1)\} \to \llbracket 1,n-1\rrbracket$ (par exemple celle fournie par le lemme), on obtient une injection $\phi\circ f|_{\llbracket 1,m\rrbracket} : \llbracket 1,m\rrbracket \to \llbracket 1,n-1\rrbracket$.

Par hypothèse de récurrence, on a donc $m\leq n-1$, et donc $m+1\leq n$, donc $A(m+1)$ est vraie.
\end{proof}

\begin{lemme}
Soient $m, n\in \N$. S'il existe une bijection entre $\llbracket 1,m\rrbracket$ et $\llbracket 1,n\rrbracket$, on a  $m=n$
\end{lemme}
\begin{proof}
Soit $f$ une telle bijection. Comme elle est injective, on a $m\leq n$.

Considérons alors la bijection réciproque $f^{-1} : \llbracket 1,n\rrbracket \to \llbracket 1,m\rrbracket$. Comme elle est également  injective, on a $n\leq m$. D'où $m=n$.
\end{proof}

\subsection{Ensembles finis et cardinal}

\begin{definition}
\index{ensemble fini}
Un ensemble $E$ est \emph{fini} s'il existe $n\in \N$ et une injection de $E$ dans $\llbracket 1,n\rrbracket$. Un ensemble qui n'est pas fini est \emph{infini}.
\end{definition}

\begin{remarque}
\begin{enumerate}[label=\alph*)]
\item On en déduit immédiatement qu'un ensemble qui s'injecte dans un ensemble fini est lui-même fini (la composée de deux injections est une injection).
\item À priori, l'entier $n$ de la définition n'est pas unique, car si $n$ convient, alors $n+1$ aussi.
\item (Zérologie) L'ensemble vide\index{ensemble vide} est fini : il existe une (unique) application de l'ensemble dans tout ensemble $F$, c'est celle dont le graphe est la partie vide de $\varnothing\times F$ (cette partie vérifie bien les conditions pour être un graphe de fonction). On l'appelle \og l'application vide\fg.
On vérifie ensuite que cette application est injective (en appliquant la définition).
\end{enumerate}
\end{remarque}


\begin{propdef}
\index{cardinal!d'un ensemble fini}
Soit $E$ un ensemble fini. Alors, il existe un unique $n\in \N$ tel que $E$ soit en bijection avec $\llbracket 1,n\rrbracket$.

Cet entier $n$ est appelé le \emph{cardinal} de $E$. Il est noté $\Card(A)$, ou $|A|$ ou $\sharp A$.
\end{propdef}
\begin{proof}
L'unicité découle des lemmes précédents.

Pour l'existence, Soit $n\in \N$ et soit $f : E\to \llbracket 1,n\rrbracket$ une injection.
Si $f(E) = \llbracket 1,n\rrbracket$, l'application est surjective donc bijective.
Sinon, il existe $a\in \llbracket 1,n\rrbracket \setminus f(E)$, donc en composant $f$ avec une injection $\phi  : \llbracket 1,n\rrbracket \setminus \{a\} \to \llbracket 1,n-1\rrbracket$, on obtient une injection $\phi\circ f : E \to \llbracket 1,n-1\rrbracket$.
On itère ce processus tant que l'application n'est pas bijective, ce qui finit par se produire puisque l'ensemble d'arrivée des injections diminue strictement à chaque étape.
\end{proof}




\begin{proposition}
\begin{enumerate}
\item Un ensemble en bijection avec un ensemble fini de cardinal $n$ est également fini de cardinal $n$.
\item L'ensemble vide est fini de cardinal zéro. Réciproquement, un ensemble fini de cardinal zéro est vide.
\end{enumerate}
\end{proposition}
\begin{proof}
\begin{enumerate}
\item Soit $f:A\to B$ une bijection. Si $B$ est fini de cardinal $n$, alors il existe une bijection $\phi : B\to \llbracket 1,n\rrbracket$, et donc  $\phi\circ f : A\to \llbracket 1,n\rrbracket$ est une bijection.
\item On a déjà vu qu'il existe une (unique) application entre $\varnothing$ et $\llbracket 1,0\rrbracket=\varnothing$ et qu'elle est injective. On peut vérifier qu'elle est surjective, toujours en appliquant la définition. Réciproquement, un ensemble de cardinal zéro est par définition en bijection avec $\llbracket 1,0\rrbracket = \varnothing$, donc est vide.
\end{enumerate}
\end{proof}

\begin{proposition}\index{cardinal!d'une union disjointe}
\begin{enumerate}
\item Si $A$ et $B$ sont disjoints et finis, alors $A\cup B$ est fini et $|A\cup B|=|A|+|B|$.
\item  Si $A$ est fini et $B \subseteq A$, alors $B$ est fini et $|B| \leq |A|$.
\item Si de plus $|B| = |A|$, alors $B=A$.
\item Si $A$ et $B$ sont finis, alors $|A\cup B| = |A| + |B| - |A\cap B|$.\index{cardinal!d'une union}
\end{enumerate}
\end{proposition}

\begin{proof}
\begin{enumerate}
\item Soient $n$ et $m$ des entiers et $f : A\to \llbracket 1,m\rrbracket$, $g : B\to \llbracket 1,n\rrbracket$ des bijections. L'application 
\[
\phi : A\cup B \to \llbracket 1,m+n\rrbracket, \quad x\mapsto
\begin{cases}
f(x) & \text{ si }x\in A\\
m+g(x) & \text{ si }x\in B
\end{cases}
\]
est bien définie, et c'est une bijection de $A\cup B$ dans $\llbracket 1,m+n\rrbracket$.
\item Si $A$ est fini, alors $B$ s'injecte dans un ensemble fini donc est fini. De même, la partie $A\setminus B$ de $A$ est également finie. On peut alors écrire $A$ comme l'union disjointe d'ensembles finis $A=B \cup (A\setminus B)$ et par ce qui précède, on a $|A|=|B|+|A\setminus B|$. On en déduit que $|B|\leq A$ et que s'il y a égalité, $A\setminus B$ est de cardinal $0$, donc vide, d'où $A=B$.
\item On a l'union disjointe $A = A\cup (B\setminus A)$  donc $|A\cup B| = |A|+|B\setminus A|$.
D'autre part, on a l'union disjointe $B = (B\cap A) \cup (B\setminus A)$, donc $|B| = |B\cap A|+|B\setminus A|$.
En remplaçant $|B\setminus A|$ par $|B|-|B\cap A|$ dans la première égalité, on obtient le résultat.
\end{enumerate}
\end{proof}

\subsection{Applications et ensembles finis}

\begin{proposition}
Soit $f : A\to B$ une application.
\begin{enumerate}
\item Si $B$ est fini, alors $|f(A)|\leq |B|$ et si de plus $|f(A)|= |B|$ alors $f$ est surjective.
\item Si $A$ est fini et $f$ est injective, alors $|f(A)|=|A|$.
\end{enumerate}
\end{proposition}
\begin{proof}
\begin{enumerate}
\item On a $f(A)\subseteq B$ donc $f(A)$ est fini et $|f(A)|\leq |B|$. S'il y a égalité des cardinaux, alors on a $f(A)=B$ ce qui signifie que $f$ est surjective.
\item Soit $g : A\to f(A)$ la corestriction de $f$ à $f(A)$, c'est-à-dire l'application déduite de $f$ en remplaçant le codomaine $B$ par $f(A)$. L'application $g$ est surjective par construction, que $A$ soit fini ou pas.

Si $f$ est injective, $g$ l'est également. On en déduit que $A$ et $f(A)$ sont en bijection. Si de plus $A$ est fini, ils ont donc le même cardinal.
\end{enumerate}
\end{proof}

\begin{proposition}
Soit $f : A\to B$ une application.
\begin{enumerate}
\item Si $B$ est fini et $f$ est injective, alors $A$ est fini et $|A|\leq |B|$.
\item Si $A$ est fini et $f$ est surjective, alors $B$ est fini et $|A|\geq |B|$.
\end{enumerate}
\end{proposition}
\begin{proof}
\begin{enumerate}
\item Si $B$ est fini, alors $A$ s'injecte dans un ensemble fini donc est fini. De plus, on a $|A|=|f(A)|\leq |B|$.
\item Soit $g:B\to A$ une \emph{section}\index{section} de $f$, c'est-à-dire une application  qui à $y\in B$ associe un antécédent quelconque de $y$. Par construction, on a $f\circ g=\Id_{B}$ donc $g$ est injective, et par le premier point $B$ est fini et $|B| \leq |A|$.
\end{enumerate}
\end{proof}

\begin{theoreme}[IMPORTANT]
Soient $A$ et $B$ finis \textbf{de même cardinal}, et soit $f : A\to B$. Alors, on a les équivalences suivantes:
\begin{center}
$f$ est injective $\iff$ $f$ est surjective $\iff$ $f$ est  bijective.
\end{center}
\end{theoreme}
\begin{proof}
Il suffit de prouver la première équivalence.

Sens $\implies$ : Si $f$ est injective, on a $|f(A)|=|A|=|B|$, et comme $f(A)\subseteq B$, l'égalité des cardinaux force $f(A)=B$ c'est-à-dire que $f$ est surjective.

Sens $\impliedby$, par contraposée : Si $f$ n'est pas injective, soient $x$ et $y$ distincts tels que $f(x)=f(y)$. Alors $f(A) = f(A\setminus \{y\})$, donc 
\[
|f(A)| \leq |A\setminus \{y\}| = |A|-1 = |B|-1,
\]
donc $f(A) \neq B$ et donc $f$ n'est pas surjective.
\end{proof}

Ce théorème est à retenir, il est indispensable dans tous les domaines des mathématiques. En particulier, il est crucial pour la théorie de la dimension des espaces vectoriels, au prochain semestre.

\begin{corollaire}
Soit $f : A\to B$ entre ensembles finis. Alors $f$ est injective si et seulement si $|f(A)|=|A|$.
\end{corollaire}
\begin{proof}On a déjà prouvé le sens \og seulement si\fg.

Si $|f(A)|=|A|$, alors la corestriction\index{corestriction} $g : A\to f(A), x\mapsto f(x)$ qui est par définition surjective, est également injective par le précédent théorème. Donc $f$ est injective.
\end{proof}



\subsection{Remarque sur les définitions équivalentes}

Il existe d'autres définitions (équivalentes) d'ensemble fini et de cardinal.
Par exemple, on aurait pu donner comme définition  : un ensemble $E$ est fini s'il existe $n\in \N$ et une surjection de $\llbracket 1,n\rrbracket$ dans $E$.

Dans ce cas, on aurait commencé par prouver le lemme suivant : \og s'il existe une surjection de $\llbracket 1,m\rrbracket$ dans $\llbracket 1,n\rrbracket$, alors $m\geq n$\fg, et l'ordre des résultats établis, ainsi que les preuves, auraient été différents.

\begin{exercice} \'Etablir tous les résultats du cours en prenant  cette définition pour base, au lieu de celle avec les injections.
\end{exercice}

On peut aussi définir les ensembles finis en utilisant $\N$.

\begin{exercice}
Soit $E$ un ensemble.
Prouver que $E$ est fini si et seulement si aucune application de $\N$ dans $E$ n'est injective.
\'Etablir une formulation équivalente avec des surjections.
\end{exercice}

L'essentiel est d'avoir une définition équivalente, mais surtout une définition maniable et efficace pour prouver les résultats du cours.

\section{Sommes et produits}

\begin{definition}[Notation $\sum$ et $\prod$]
\index{$\sum$}\index{$\prod$}
Soit $E$ un ensemble fini, et $f : E\to \C$ (ou autre codomaine que $\C$, l'essentiel étant de pouvoir sommer et multiplier).

On note $\sum_{x\in E} f(x)$ le nombre $0$ auquel on ajoute la somme des valeurs prises par $f(x)$ lorsque $x$ parcourt $E$.

On note $\prod_{x\in E} f(x)$ le nombre $1$ que l'on multiplie par le produit des valeurs prises par $f(x)$ lorsque $x$ parcourt $E$.
\end{definition}

\begin{remarque}[\og Un produit vide vaut $1$, une somme vide vaut $0$\fg]
Faire intervenir $0$ et $1$ sert à avoir les propriétés: $\sum_{x\in \varnothing} (...)=0$ et $\prod_{x\in \varnothing} (...) = 1$.
\end{remarque}

\begin{remarque} Dans la définition, si $E=A\cup B$ et $A\cap B = \varnothing$, alors $\sum_{x\in E}f(x) = \sum_{x\in A}f(x)+\sum_{x\in B}f(x)$.
\end{remarque}

Si $I$ est un ensemble fini et $(u_i)_{i\in I}$ est une famille de complexes, on note $\sum_{i\in I} u_i$ la somme de (zéro plus) tous ces complexes. 

\begin{definition}
La notation $\sum_{i=0}^n u_i$ signifie par définition $\sum_{i\in \llbracket 0,n\rrbracket} u_i$, et plus généralement,  $\sum_{i=a}^b u_i$ signifie par définition $\sum_{i\in \llbracket a,b\rrbracket} u_i$. 
\end{definition}

\begin{proposition}[Linéarité de la somme]
\index{linéarité de $\sum$}
Soit $A$ un ensemble fini, $f$, $g$ deux fonctions de $A$ dans $\C$ et $\lambda$, $\mu$ deux complexes. Alors :
\[
\sum_{x\in A}(\lambda f(x)+\mu g(x)) 
= \lambda\sum_{x\in A}f(x)+ \mu\sum_{x\in A} g(x).
\]
\end{proposition}

\begin{exemple}
Soit $n\in N$. On peut écrire 
\[
\sum_{k=0}^n (2k+3) 
= 2\sum_{k=0}^n k + 3\sum_{k=0}^n1 
= 2\frac{n(n+1)}{2}+3(n+1) 
= (n+1)(n+3).
\]
\end{exemple}

\begin{theoreme}[Théorème de Fubini pour les sommes finies]
\index{Fubini, théorème}
Soient $A$ et $B$ des ensembles finis et $f : A\times B \to \C$. On a les égalités:
\[
\sum_{x\in A}\left(\sum_{y\in B} f(x,y)\right)
=
\sum_{(x,y)\in A\times B} f(x,y)
=
\sum_{y\in B}\left(\sum_{x\in A} f(x,y)\right)
\]
\end{theoreme}
\begin{proof}
Par récurrence sur $|B|$. 

\noindent\textbf{Initialisation.} Si $|B|=0$, alors $B$ est vide, et les sommes indexées par $B$ sont nulles. D'autre part, $A\times B=\varnothing$, donc au final les trois sommes sont nulles.

\noindent\textbf{Hérédité.} Soit $n\in \N$, et supposons que pour toute partie $B$ de cardinal $n$ et tout ensemble fini $A$, les égalités soient vraies. Soit $B$ un ensemble de cardinal $n+1$. Écrivons $B = B'\cup \{b\}$, avec $b\not\in B'$, et $B'$ de cardinal $n$. On a alors une union disjointe $A\times B = A\times B' \cup A\times\{b\}$ et donc
\[ 
\sum_{(x,y)\in A\times B}f(x,y)
=
\sum_{(x,y)\in A\times B'}f(x,y) + \sum_{(x,y)\in A\times \{b\}}f(x,y)
=
\sum_{(x,y)\in A\times B'}f(x,y) + \sum_{x\in A} f(x,b).
\]
D'autre part, on a 
\begin{align*}
\sum_{x\in A}\left(\sum_{y\in B}f(x,y)\right)
&=
\sum_{x\in A}\left(\sum_{y\in B'}f(x,y)+\sum_{y=b}f(x,y)\right)\\
&=
\sum_{x\in A}\sum_{y\in B'}f(x,y) + \sum_{x\in A}f(x,b)\\
&=
\sum_{(x,y)\in A\times B'}f(x,y) + \sum_{x\in A}f(x,b)\text{ par hypothèse de récurrence}\\
&=
\sum_{(x,y)\in A\times B} f(x,y) \text{ par la rémarque précédente}.
\end{align*}
La deuxième égalité se prouve de la même façon en échangeant les rôles de $A$ et $B$.
\end{proof}

\begin{definition}[Factorielle]
\index{factorielle}
Soit $n\in \N$. La factorielle de $n$, notée $n!$, est le produit de tous les entiers strictement positifs et inférieurs à $n$. Autrement dit, $n! = \prod_{k \in \llbracket 1,n\rrbracket} k$. En particulier, si $n=0$, le produit est vide et donc $0!=1$ par définition d'un produit vide. 
\end{definition}


\section{Combinatoire}
\subsection{Principes élémentaires de combinatoire}

\begin{proposition}\label{prop-cardinal-produit}
\index{cardinal!d'un produit}
Soient $E$ et $F$ finis de cardinal $n$ et $p$. Alors $E\times F$ est de cardinal $np$.
\end{proposition}
\begin{proof}
L'application 
\[
\phi : \llbracket 1,n\rrbracket \times \llbracket 1,p\rrbracket\to \llbracket 1,np\rrbracket,\quad
(x,y)\mapsto (x-1)p+y
\]
est bijective.
\end{proof}
\begin{corollaire}\label{prop-cardinal-puissance}
\index{cardinal!d'une puissance}
Soit $E$ un ensemble fini. Alors pour tout $k\in \N^*$, $|E^k|=|E|^k$.
\end{corollaire}
\begin{proof}
Par récurrence sur $k$. 
\textbf{Initialisation. }Lorsque $k=1$, on a bien  $|E^1|=|E|=|E|^1$.\\
\textbf{Hérédité.} Soit $k\in \N^*$ et supposons que $|E^k|=|E|^k$.
On a $|E^{k+1}| = |E^k\times E|=|E^k|\cdot |E|$ par la proposition précédente, et d'autre part par hypothèse de récurrence, on a $|E^k|=|E|^k$. Finalement,  $|E^{k+1}|=|E|^{k+1}$. 
\end{proof}

\begin{corollaire}\label{prop-cardinal-fonctions}
\index{cardinal!d'un ensemble de fonctions}
Soient $E\neq \varnothing$ et $F$ des ensembles finis de cardinal  $n$ et $p$. L'ensemble $\mathcal F(E,F)$ des fonctions de $E$ dans $F$ est de cardinal $p^n$. 
\end{corollaire}
\begin{proof}
L'ensemble $\mathcal F(E,F)$ est en bijection avec $F^n$. (Une fonction correspond au choix d'un élément de $F$ pour chacun des $n$ éléments de $E$.) 
\end{proof}

\begin{remarque}
\begin{enumerate}
\item Si $E$ est vide, il existe une unique application de $E$ dans n'importe quel ensemble, fût-il vide : l'application vide\index{vide!application}\index{application!vide}. Donc $|\mathcal F(E,F)|=1$, ce qui permet d'étendre la formule lorsque $E$ est vide. Lorsque $E$ et $F$ sont tous deux vides, on pose $0^0=1$ (ou plutôt on démontre, si on a la \og bonne\fg{} définition de $a^b$) et la formule reste valable.
\item L'ensemble $\mathcal F(E,F)$ est également noté $F^E$. Avec cette notation, on a la formule $\left| F^E\right| = |F|^{|E|}$.
\end{enumerate}
\end{remarque}

\begin{proposition}\label{prop-card-injections}
Soient $n, p\in \N$ avec $p\leq n$.
\begin{enumerate}
\item Il y a $n(n-1)(n-2)...(n-p+1)=\frac{n!}{(n-p)!}$ injections de $\llbracket 1,p\rrbracket$ dans $\llbracket 1,n\rrbracket$.
\item En particulier, il y a $n!$ bijections de $\llbracket 1,n\rrbracket$ dans lui-même.
\end{enumerate}
\end{proposition}

Une bijection de $\llbracket 1,n\rrbracket$ dans lui-même s'appelle une \emph{permutation} $\llbracket 1,n\rrbracket$.

\begin{proposition}
\index{cardinal!de $\mathcal P(E)$}\label{prop-cardinal-parties}
Soit $E$ un ensemble fini. L'ensemble $\mathcal P(E)$ des parties de $E$ est fini, de cardinal $2^{|E|}$.
\end{proposition}
\begin{proof}
Il y a une bijection entre $\mathcal P(E)$ et $\mathcal F(E,\{0,1\})$, donnée par $A\mapsto \mathbf{1}_A$, l'application qui à une partie $A$ associe sa fonction caractéristique. Or, on a $|\mathcal F(E,\{0,1\})| = |\{0,1\}|^{|E|} =2^{|E|} $.
\end{proof}

\subsection{Coefficients binomiaux}

\begin{definition}
\index{coefficients binomiaux}
Soit $n\in\N$, et $k\in \Z$. 

On note $\mathcal P_k(\llbracket 1,n\rrbracket)$ ou même $\mathcal P_k(n)$ l'ensemble des parties de $\llbracket 1,n\rrbracket$ qui sont de cardinal $k$.

On note $\binom{n}{k}$ le nombre de parties de $\llbracket 1,n\rrbracket$ de cardinal $k$, c'est-à-dire $\binom{n}{k}=|\mathcal P_k(n)|$.
\end{definition}

Remarque : si $k<0$ ou si $k>n$, $\binom{n}{k}=0$ car il n'y a aucune partie de $\llbracket 1,n\rrbracket$ de cardinal $k$.

\begin{proposition}
On a $\binom{n}{k} = \binom{n}{n-k}$.
\end{proposition}
\begin{proof}
L'application $\phi : \mathcal P_k(n) \to \mathcal P_{n-k}(n), \: A\mapsto  \complement A$, est une bijection.
\end{proof}

\begin{proposition}
Soit $n\in\N$. On a $\sum_{k=0}^n \binom{n}{k} =2^n$.
\end{proposition}
\begin{proof}
On classe les parties de $P(\llbracket 1,n\rrbracket)$ suivant leur cardinal $k$, c'est-à-dire qu'on écrit  $\mathcal P(\llbracket 1,n\rrbracket) = \bigcup_{k=0}^n \mathcal P_k(n)$, l'union étant disjointe. En prenant le cardinal des deux membres on obtient $2^n=|P(\llbracket 1,n\rrbracket)| = \sum_{k=0}^n |\mathcal P_k(n)| = \sum_{k=0}^n \binom{n}{k}$.
\end{proof}

\begin{proposition}[Formule de Pascal]
\index{formule de Pascal}
Soient $k,n\in \N$. Alors $\binom{n}{k}+\binom{n}{k+1} = \binom{n+1}{k+1}$.
\end{proposition}
\begin{proof}
On compte les parties à $k+1$ éléments de $\llbracket 1,n+1\rrbracket$ selon qu'elles contiennent ou non $n+1$. Celles qui ne contiennent pas $n+1$ sont en bijection avec $\mathcal P_{k+1}(n)$, et celles qui contiennent $n+1$ contiennent $k$ autres éléments de $\llbracket 1,n\rrbracket$ et sont donc en bijection avec $\mathcal P_k(n)$.
\end{proof}

\begin{proposition}[Formule du binôme de Newton]
\index{formule!du binôme de Newton}
Soient $a, b\in \C$ et $n\in \N$. Alors $(a+b)^n = \sum_{k=0}^n \binom{n}{k} a^k b^{n-k}$.
\end{proposition}
\begin{proof}
On développe le produit $(a+b)^n = (a+b)(a+b)...(a+b)$, ce qui donne $2^n$ termes tous de la forme $a^kb^{n-k}$, pour certains $0\leq k\leq n$.
À chaque façon de choisir un terme ($a$ ou $b$) dans chaque parenthèse, on associe une partie $X \in \llbracket 1,n\rrbracket$ qui correspond aux parenthèses où on choisit $a$ au lieu de $b$. On peut alors écrire:
\begin{align*}
(a+b)^n
&= \sum_{X \in \mathcal P(\llbracket 1,n\rrbracket)} a^{|X|}b^{n-|X|}\\
&= \sum_{k=0}^n \left(\sum_{X\in\mathcal P_k(n)} a^{|X|}b^{n-|X|}\right)\\
&= \sum_{k=0}^n \left( \sum_{X\in\mathcal P_k(n)} a^k b^{n-k}\right)\\
&= \sum_{k=0}^n a^kb^{n-k} \left|\mathcal P_k(n)\right|\\
&= \sum_{k=0}^n \binom{n}{k}a^kb^{n-k}\\
\end{align*}
\end{proof}

Remarque : il existe aussi une preuve par récurrence sur $n$ qui utilise la formule de Pascal, qui est moins parlante du point de vue combinatoire.

\begin{proposition}Soient $n, k\in \N$. Alors $\binom{n}{k} = \frac{n!}{k!(n-k)!}$.
\end{proposition}

La preuve qui suit est la version rigoureuse de la phrase \og pour compter le nombre de parties de cardinal $k$ de $\llbracket 1,n\rrbracket$, on compte le nombre de  listes ordonnées de cardinal $k$, c'est-à-dire $\frac{n!}{(n-k)!}$, puis on divise par le nombre de façons de désordonner ces listes, c'est-à-dire $k!$, puisqu'on ne s'occupe pas de l'ordre\fg. (La fin de la phrase est floue et non justifiée : pourquoi est-il correct de \og diviser\fg{} lorsqu'on ne \og s'occupe pas\fg{} de quelque chose ?)

\begin{proof} 

Soit $I$ l'ensemble des injections de $\llbracket 1,k\rrbracket$ dans $\llbracket 1,n\rrbracket$ (en bijection avec les listes ordonnées de $k$ éléments de $\llbracket 1,k\rrbracket$). Il est de cardinal $\frac{n!}{(n-k)!}$. Montrer le résultat revient à montrer que $|I| = k! |\mathcal P_k(n)|$.

Or, on peut écrire $|I| = \sum_{X\in \mathcal P_k(n)} \left| \{f\in I\:,\: f(\llbracket 1,k\rrbracket)=X\}\right|$. Mais si $X$ est de cardinal $k$, une injection $f\in I$ telle que  $f(\llbracket 1,k\rrbracket)=X$ est forcément une bijection, et on sait qu'il y a $k!$ telles bijections.

Donc, $|I| = \sum_{X\in \mathcal P_k(n)} k! = k! |\mathcal P_k(n)|$, ce qu'il fallait démontrer.
\end{proof}

\begin{remarque}
Le principe combinatoire général derrière cette preuve est le suivant : Si $\phi : A\to B$ entre ensembles finis, alors $|A| = \sum_{b\in B} \left|f^{-1}(\{b\})\right|$. Cela revient à compter le nombre d'éléments de $a$ en les classant d'abord selon leur image dans $B$, puis en sommant, pour chaque $b$, le nombre d'antécédents de $b$. Ici, ce principe serait appliqué avec $A=I$, $B=\mathcal P_k(n)$, et $\phi$ serait l'application qui à $f\in I$ associe son image, qui est un élément de $\mathcal P_k(n)$. Dans ce cas particulier, toutes les images réciproques ont le même cardinal $k!$.
\end{remarque}

\begin{remarque}
On peut trouver d'autres preuves des résultats présentés ici : des preuves par récurrence, ou bien des preuves calculatoires utilisant la formule $\binom{n}{k} = \frac{n!}{k!(n-k)!}$ qui doit alors être démontrée le plus tôt possible. Les preuves combinatoires sont souvent plus riches de sens.
\end{remarque}


Attention, la présentation qui suit diffère sans doute beaucoup de celle vue en terminale : il faut faire l'effort de l'étudier en détail même si l'ordre dans lequel les notions sont introduites semble \og mauvais\fg : en fait, c'est le \og bon\fg{}  ordre.

Le cours d'arithmétique des polynômes suivra le même canevas (définitions semblables, mêmes lemmes aux mêmes endroits, mêmes preuves), de même que le cours d'algèbre générale sur les anneaux par la suite.

\section{Préliminaires}

\subsection{Division euclidienne}
\begin{proposition}[Division euclidienne]
Soit $a\in \N$ et $b\in \N*$. Il existe un unique couple $(b,r) \in \N^2$ vérifiant les deux propriétés suivantes:
\begin{enumerate}
\item $a=bq+r$;
\item $r < b$.
\end{enumerate}
L'entier $b$ est le \emph{quotient} de la division euclidienne de $a$ par $b$, et $r$ est le \emph{reste}.
Effectuer la division euclidienne de $a$ par $b$, c'est écrire $a = bq+r$ avec $b$ et $q$ comme plus haut.
\end{proposition}

Exemple : $17=5\times 3 + 2$ est la division euclidienne de $17$ par $5$. Le quotient est $3$ et il reste $2$. Par contre, l'écriture $17=5\times 2+7$ bien que correcte  n'est pas une division euclidienne, car le reste \emph{doit} être strictement inférieur à $5$, dans une division euclidienne.

\subsection{Idéaux de $\Z$}

Soit $\alpha$ un entier. On rappelle que $\alpha\Z = \{ \alpha k\:\mid\: k\in \Z\} = \{..., -2\alpha, -\alpha, 0, \alpha, 2\alpha, 3\alpha, ...\}$. Les ensembles $\alpha\Z$ et $(-\alpha)\Z$ sont identiques.

Exemples : $3\Z = \{-6,-3,0,3,6,9,12,...\}$. Si $\alpha=0$, alors $\alpha\Z = \{0\}$. On a $\alpha\Z = \Z$ ssi $\alpha$ est égal à $1$ ou $-1$.

Une partie de $\Z$ de la forme $\alpha\Z$ contient zéro, est stable par addition, et par opposé. Réciproquement, on peut s'intéresser aux parties qui vérifient ces trois  propriétés et se demander si elles sont toutes de la forme $\alpha\Z$.

\begin{definition}
Soit $I \subseteq \Z$ une partie de $\Z$. On dit que $I$ est un \emph{idéal} de $\Z$ si
\begin{enumerate}
\item C'est un \emph{sous-groupe} de $\Z$, c'est-à-dire $I$ contient $0$ et est stable par addition et opposé : 
\[ \forall x, y\in I, \: x+y \in I \text{et} -x \in I.\]
\item Il est \emph{absorbant pour la multiplication} c'est-à-dire:
\[  \text{Si } x \in I \text{ et }n\in \Z, \text{alors } n\cdot x \in I.\]
\end{enumerate}
\end{definition}

\begin{proposition}
Tout idéal de $\Z$ est de la forme $\{k\alpha\:\mid \: k \in \Z\}$, avec $\alpha /in \N$. (Un tel ensemble est noté $\alpha\Z$.)
\end{proposition}

\begin{proof}
On va montrer que les sous-groupes de $\Z$ sont de cette forme, et que ce sont des idéaux.

Soit $G \subseteq \Z$ un sous-groupe. Soit $G^*_+ = G \cap \N^*$. Il y a deux cas:
\begin{enumerate}
\item Si $G^*_+$ est vide, cela signifie que $G$ ne possède aucun élément strictement positif. Comme $G$ est stable par opposé, il ne peut pas non plus contenir d'éléments strictement négatifs. Cela signifie que $G=\{0\}$.
\item Sinon, c'est une partie non vide de $\N$, qui possède donc un plus petit élément, notons-le $\alpha$.
Par définition, $G$ est stable par somme et opposé, donc $2\alpha\in G$ et $-\alpha \in G$ et plus généralement, pour tout $k\in \Z$, on a $k\alpha \in G$.
Donc $\alpha \subseteq G$. Montrons l'inclusion inverse.
Soit $x\in G$, positif. \'Ecrivons la division euclidienne de $x$ par $\alpha$. On a $x = \alpha q + r$, avec $r<\alpha$. Comme $G$ est stable par somme et différence et que $\alpha q \in G$, on en déduit que $r = x-\alpha q$ est également dans $G$. Or, $r<\alpha$, donc par minimalité de $\alpha$, $r=0$, ce qui montre que $x = \alpha q$, donc que $x\in \alpha\Z$.
Si $x$ est négatif, ce qui précède montre que $-x\in \alpha\Z$, donc que $x\in \alpha\Z$.
\end{enumerate}

On vérifie ensuite sans peine que tous les sous-groupes de $\Z$ sont en fait des idéaux de $\Z$.
\end{proof}


L'entier $\alpha$ dans la définition est appelé le générateur principal de $G$.


\section{Pgcd}

\begin{proposition}
Soient $a$ et $b$ deux entiers. L'ensemble $\{ak+bl\:\mid\: k, l\in \Z\}$ noté par définition $a\Z+b\Z$ ou $\langle a,b\rangle$, est un idéal de $\Z$. 
\end{proposition}
\begin{proof}
Appliquer la définition de sous-groupe, ce qui prouve que c'est un idéal par la section précédente.
\end{proof}

\begin{definition}
Soient $a$ et $b$ des entiers. Le générateur principal de $a\Z+b\Z$ est appelé le pgcd de $a$ et $b$, il est noté $\pgcd(a,b)$ ou bien $a\wedge b$.
\end{definition}

\begin{proposition}
Soient $a$ et $b$ des entiers, et $d=\pgcd(a,b)$.
On a les propriétés suivantes
\begin{enumerate}
\item L'entier $a$ est dans $a\Z+b\Z$, donc $d$ divise $a$. De même, $d$ divise $b$.
\item L'entier $df$ est dans $d\Z$, donc il existe $k$ et $l$ dans $\Z$, tels que $d = ak+bl$. On dit que $(k,l)$ est un couple (ou paire, par abus de langage) de Bézout pour $a$ et $b$. L'égalité $d=ak+bl$ est appelée \emph{relation de Bézout}.
\item Au sens de la divisibilité, $d$ est le plus grand diviseur commun  de $a$ et $b$. Ceci explique le nom (\emph{plus grand commun diviseur} de $d$. Cette propriété est précisée dans la proposition suivante.
\item Si $d=0$, alors $a=b=0$.
\item On a $\pgcd(a,b)=\pgcd(a,b) = \pgcd(a,-b)$, car $a\Z+b\Z = b\Z+a\Z=a\Z+(-b)\Z$.
\item $\pgcd(a,0)=|a|$.
\item $\pgcd(a,1)=1$.
\end{enumerate}
\end{proposition}

\begin{proposition}
Si $x>0$, $x|a$ et $x|b$, et $\forall m, m|a \text{ et } m|b \implies m|x$, alors $x=d$.
\end{proposition}
\begin{proof}
Si $x|a$ et $x|b$, alors $x|d$. D'autre part, $d|a$ et $d|a$, donc $d|x$. Donc finalement, $d=x$.
Attention, la condition $x>0$ est indispensable pour ce raisonnement. Deux entiers relatifs peuvent se diviser l'un l'autre, comme $1$ et $-1$, sans être égaux.
\end{proof}

\begin{proposition}
Soit $k>0$. On a $\pgcd(ka,kb)=k\pgcd(a,b)$.
\end{proposition}
\begin{proof}
Notons provisoirement $d_1 = \pgcd(a,b)$ et $d_2 = \pgcd(ka,kb)$.

Comme $d_1|a$ et $d_1|b$, on a $kd_1|ka$ et $kd_1|kb$ donc finalement $kd_1|d_2$. En  particulier, $k|d_2$ donc $\frac{d_2}{k}$ est un entier.

D'autre part, $d_2|ka$ et $d_2|kb$, donc  en divisant par $k$ et en utilisant la remarque précédente, on a $\frac{d_2}{k} | a$ et $\frac{d_2}{k} | b$ donc $\frac{d_2}{k} | d_1$, d'où $d_2 | kd_1$. 

Comme $kd_1$ et $d_2$ sont positifs, on en déduit $d_2=kd_1$.
\end{proof}

\subsection{Algorithme d'Euclide}

\begin{lemme}[d'Euclide]
Soient $a$, $b$ et $k$ des entiers relatifs. Alors:
\[ \pgcd(a,b) = \pgcd(a+kb,b).\]
\end{lemme}

\begin{proof}
Ils y a au moins deux façons de prouver le résultat : on peut montrer que les idéaux $a\Z+b\Z$ et $(a+kb)\Z+b\Z$ sont les mêmes, ce qui implique qu'ils ont le même générateur principal, ou alors on peut montrer que $(a,b)$ et $(a+kb,b)$ ont les mêmes diviseurs communs, donc le même plus grand diviseur commun.\\
\underline{Première preuve (mêmes idéaux)}.
D'une part, $(a+kb)\Z+b\Z \subseteq a\Z+b\Z$ car si $i$ et $j$ sont des entiers, alors $i(a+kb)+jb = ia+(iK+j)b \in a\Z+b\Z$.
D'autre part, $a\Z+b\Z \subseteq (a+kb)\Z+b\Z$ car si $i$ et $j$ sont des entiers, alors $ia+jb = i(a+kb) +(j-ik)b \in (a+kb)\Z+b\Z$.
Finalement, les idéaux $a\Z+b\Z$ et $(a+kb)\Z+b\Z$ sont identiques donc ont le même générateur principal.\\
\underline{Deuxième preuve (mêmes diviseurs)}.
Si $m|a$ et $m|b$, alors $m|a+kb$ et $m|b$.\\
Si $m|a+kb$ et $m|b$, alors $m|a+kb-kb$ et $m|b$, donc $m$ divise $a$ et $b$.\\
On en déduit que les couples $(a,b)$ et $(a+kb,b)$ ont les mêmes diviseurs communs. Ils ont donc le même pgcd.
\end{proof}

\begin{corollaire} En particulier, si $a = bq+r$ est la division de $a$ par $b\neq 0$, alors 
\[ \pgcd(a,b) = \pgcd(b,r).\]
\end{corollaire}

\begin{theoreme}[Algorithme d'Euclide]

Appliquer l'algorithme d'Euclide aux entiers naturels $a$ et $b$, c'est effectuer une suite de divisions euclidiennes:

\begin{align*}
a &= q_1 b + r_1\\
b &= q_2 r_1+r_2\\
r_1 &= q_3 r_2+ r_3\\
\cdots & \\
r_{n-2} &= q_nr_{n-1}+r_n
\end{align*}
en continuant tant que $r_n$ n'est pas nul. Alors, on a les résultats suivants:

\begin{enumerate}
\item (terminaison de l'algorithme) Au bout d'un certain nombre d'étapes, on a $r_n=0$, donc l'algorithme termine en un nombre fini d'étapes.
\item Le dernier reste non nul $r_{n-1}$ est le pgcd de $a$ et $b$.
\end{enumerate}
\end{theoreme}
\begin{proof}

\begin{enumerate}
\item (Preuve de terminaison) Il s'agit de montrer que l'on ne peut pas continuer indéfiniment à faire des divisions euclidiennes. Par définition de ce qu'est une division euclidienne, on a  : $b>r_1$, $r_1>r_2$ et plus généralement $r_i>r_{i+1}$. La suite des restes est une suite strictement décroissante d'entiers positifs, elle ne peut pas être infinie.
\item (Preuve de correction du calcul de pgcd) Par le lemme d'Euclide et son corollaire appliqués à chaque étape, on a 
\[
\pgcd(a,b) = \pgcd(b,r_1)=\pgcd(r_1,r_2) = ... = \pgcd(r_{n-1},r_n) = \pgcd(r_{n-1},0) = r_{n-1}.
\]

\end{enumerate}
\end{proof}

On remarque qu'il n'est pas nécessaire que $a>b$ dans l'algorithme : si ce n'est pas le cas, l'algorithme les replace dans le bon ordre au cours de la première étape.

L'algorithme d'Euclide permet également d'obtenir une relation de Bézout en \og remontant \fg{} les étapes de l'algorithme : 
\[ d = r_{n-1} = r_{n-3} - q_{n-1}r_{n-2} = r_{n-3} - q_{n-1}(r_{n-4} - q_{n-2}r_{n-3})... = au+bv.\]

\subsection{Nombres premiers entre eux, lemmes de Gauss}

\begin{definition}
Deux nombres relatifs $a$ et $b$ sont premiers entre eux si $\pgcd(a,b)=1$. On note :  $a\wedge b = 1$.
\end{definition}

\begin{proposition} Soient $a$ et $b$ des entiers. On a 
\[
a\wedge b = 1 \iff \left(\exists u, v\in \Z\mid au+bv=1\right)
\]
\end{proposition}
\begin{proof}
Sens $\implies$ : il existe une relation de Bézout.\\
Sens $\impliedby$ : si $au+bv=1$, alors  $\pgcd(a,b)$ divise $1$, donc vaut $1$.
\end{proof}

\begin{proposition}
Soient $a$ et $b$ des entiers.
Si $a\wedge b = 1$ et $a\wedge c = 1$, alors $a\wedge bc=1$.
\end{proposition}
\begin{proof}
Si $au+bv=1$ et $au'+cv'=1$ sont des relations de Bézout, on a en multipliant les deux:
\[ 1 = (au+bv)(au'+cv') = a(auu'+bvu'+ucv')+bcvv'.\]
\end{proof}
\begin{corollaire}
Soient $a$, $b$ et $n>0$, $m>0$ des entiers.
Si $a\wedge b = 1$, alors $a^n\wedge b^m=1$.
\end{corollaire}
\begin{proof}
On a $a\wedge b = 1 \implies a\wedge b^2= ... =  a\wedge b^m=1$, puis $b^m\wedge a  1\implies b^m\wedge a^2 = ... = b^m\wedge a^n=1$.
\end{proof}

\textbf{Attention}, ceci n'est \textbf{pas} un résultat de passage au produit avec le symbole $\wedge$ ! Si on a $a\wedge b = 1$ et $c\wedge d = 1$, on n'a \textbf{pas} $ac\wedge bd=1$. Exemple  : $2\wedge 3=1$ et $3\wedge 2=1$ et pourtant $6\wedge 6\neq 1$.

\begin{theoreme}[\og théorème/lemme de Gauss\fg]
Soient $a$, $b$ et $c$ des entiers.
Si $a\wedge b = 1$ et $a|bc$, alors $a|c$.
\end{theoreme}
\begin{proof}
Soit $ak+bl=1$ une relation de Bézout pour $a$ et $b$.
Si $a$ divise $bc$, alors il divise également $blc$. D'autre part, $a$ divise $akc$. Donc $a | (bl+ak)c$ c'est-à-dire $a|c$.
\end{proof}

\subsection{Résolution des équations diophantiennes du type $ax+by=c$}


\begin{definition}
Une équation diophantienne est une équation du type $F(x_1, x_2, ...x_k)=0$, les inconnues $x_1$, ... $x_k$ appartiennent à $\Z$, ou une partie de $\Z$.
\end{definition}

Exemples :\\
$12x+3y=8$, d'inconnues $x$ et $y$ dans $\Z$.\\
$2^n-3^m=7$ d'inconnues $n$ et $m$ dans $\N$.\\
$x^n+y^n=z^n$ d'inconnues $x$, $y$, $z$, $n$ dans $\N$. (C'est l'équation de Fermat; il a été démontré en 1994 après trois siècles d'efforts que l'équation n'admet des solutions que si $n=2$.)

Dans ce cours, on s'intéresse aux équations du type $ax+by=c$ d'inconnues $x$ et $y$ dans $\Z$, et avec $a$, $b$ et $c$ des paramètres entiers. 

Géométriquement, cela revient à trouver les points à coordonnées entières de la droite du plan d'équation cartésienne $ax+by=c$.

La méthode de résolution consiste\footnote{comme pour les équations différentielles, ou les systèmes linéaires} à trouver une solution particulière de l'équation, puis à y ajouter les solutions de l'équation homogène associée, qui est par définition l'équation obtenue en remplaçant le second membre par zéro: $ax+by=0$. C'est le contenu de la proposition suivante:

\begin{proposition}
Soient $a$, $b$ des entiers non tous deux  nuls, $c$ un entier. On considère l'équation $(E) \: : \: ax+by=c$ d'inconnue $(x,y)\in \Z^2$, ainsi que l'équation homogène associée $(E_h) \: : \: ax+by=0$. 

Si $(x_p,y_p)$ est une solution particulière de $(E)$, alors son ensemble de solutions est 
\[
\left\{(x_p,y_p) + (s,t)\:\mid\: (s,t) \text{ solution de } E_h\right\}
\]
\end{proposition}

\begin{proof}
Soit $(x,y)$ un couple d'entiers. 
\begin{align*}
ax+by=c 
&\iff ax+by = ax_p+by_p\\
&\iff a(x-x_p) + b(y-y_p) = c-c=0,
\end{align*}
donc $(x,y)$ est solution de $(E)$ si et seulement si $(x-x_p,y-y_p)$ est solution de l'équation homogène $(E_h)$ associée à $(E)$. On en déduit le résultat.
\end{proof}

Il reste donc à établir un critère pour l'existence de solutions, et à donner une méthode pour trouver des solutions particulières, et pour résoudre les équations homogènes.

\begin{proposition}[Résolution des équations homogènes]
Soient $a$, $b$ des entiers non tous deux  nuls, et notons $d = \pgcd(a,b)$. L'équation $ax+by=0$ d'inconnue $(x,y)\in \Z^2$ a pour ensemble de solutions :
\[
\left\{k\left(\frac{-b}{d} , \frac{a}{d}\right),\: k\in \Z\right\}
\]
\end{proposition}

(Remarque : si $a$ et $b$ sont nuls, alors l'ensemble des solutions est $\Z^2$ tout entier...)

\begin{proof}(de la proposition)
\'Ecrivons $a=da'$ et $b=db'$. L'équation s'écrit donc $da'x+db'y=0$ et en simplifiant par $d$ qui est non nul, on obtient l'équation équivalente $a'x+b'y=0$, avec $a'\wedge b'=1$.

Si un des deux entiers $a$ ou $b$ est nul, le résultat est facile.

Sinon, le théorème de Gauss donne alors $a'|y$, donc il existe $k\in \Z$ tel que $y = ka'$.  On trouve alors $x=-kb'$ en simplifiant par $a'$.
\end{proof}

\begin{exemple}
L'ensemble des solutions entières de l'équation $2x+6y=0$ est $\{k(3,-1)\:\mid\:k\in\Z\}$.
\end{exemple}








\section{Nombres premiers}

\subsection{Définition}

\begin{definition}
Un entier naturel $p$ est dit \emph{premier} s'il possède exactement deux diviseurs positifs distincts : $1$ et $p$.
En particulier, un nombre premier est toujours $\geq 2$.
Un entier naturel $\geq 2$ qui n'est pas premier est dit \emph{composé}.
\end{definition}
% anecdote : voir la page nLab "too simple to be simple"


 Le nombre $1$ n'est pas premier. Les nombres premiers sont $2$, $3$, $5$, $7$, $11$, $13$, $17$, $19$, $23$ etc.




\subsection{Factorisation en produit d'irréductibles}

\section{Ppcm}

\section{Compléments}

\subsection{Petit théorème de Fermat}
\subsection{Théorème chinois}
\subsection{Indicatrice d'Euler}


\chapter{Relations d'ordre, relations d'équivalence}
\minitoc
\hyperlink{toc}{\retourTOC}

\section{Relations binaires}

\begin{definition}[Relation binaire]
Soit $E$ un ensemble. Une \emph{relation binaire}\index{relation binaire} ${\mathcal R}$ sur $E$ est une application de $E\times E$ dans $\{\text{vrai, faux}\}$. On note en général \og$x{\mathcal R}y$\fg{} au lieu de \og${\mathcal R}(x,y)=\text{vrai}$\fg{}.
\end{definition}

\begin{definition}[Graphe d'une relation binaire]
Soit $\mathcal R$ une relation binaire sur $E$. La partie de $E\times E$ constituée des couples $(x,y)$ tels que ${\mathcal R}(x,y)=\text{vrai}$ est appelée le \emph{graphe} de la relation $\mathcal R$. Ce graphe est parfois noté $\Gamma_{\mathcal R}$.

Réciproquement, si $E$ est un ensemble et $\Gamma \subseteq E\times E$, on obtient une relation binaire ${\mathcal R}_\Gamma$ qui est définie par $\forall x, y\in E, \: x{\mathcal R}_\Gamma y \iff (x,y)\in \Gamma$. Le graphe de cette relation binaire est exactement $\Gamma$.
\end{definition}






Si $E$ est un ensemble fini dont on note $x_1$, ... $x_n$ les éléments, on peut visualiser une relation binaire comme un tableau à double entrée dans les éléments de $E$, dont les valeurs sont \og vrai\fg{} ou \og faux\fg (ou \og oui\fg{} ou \og non\fg), du type suivant:

\begin{center}
\begin{tabular}{|l|c|c|c|c|}\hline
		& $x_1$ & $x_2$ & $\dots$ & $x_n$ \\ \hline
$x_1$ 	& oui & non & $\dots$ & oui \\ \hline
$x_2$ 	& non & non & $\dots$ & non \\ \hline
$\vdots$& $\vdots$ & $\vdots$ &  $\ddots$ & $\vdots$ \\ \hline
$x_n$	& non & non & $\dots$ & oui \\ \hline
\end{tabular}
\end{center}

Un tel tableau détermine si, étant donné deux éléments $x$ et $y$, on a $x\mathcal R y$ ou pas. 

\begin{exemple}
Si $E$ est fini de cardinal $n$, il existe $2^{n^2}$ relations binaires sur $E$. En effet, d'après \ref{prop-cardinal-produit}, l'ensemble $E\times E$ est de cardinal $n^2$ et donc d'après \ref{prop-cardinal-parties} il admet $2^{n^2}$ parties.
\end{exemple} 

\begin{exemples}
Les symboles $=$, $\leq$, $<$, $\geq$, $>$, $|$ (divise), $//$ (parallèle à), $\perp$ (perpendiculaire à), $\subseteq$ (inclus dans) désignent des relations binaires sur des ensembles. La relation d'égalité $=$ correspond à la partie diagonale de $E\times E$, c'est-à-dire à la partie 
\[ \Delta_E = \{(x,x)\:|\: x\in E\}\]
\index{partie!diagonale}
Les deux tableaux suivants illustrent deux relations distinctes sur le même ensemble $\{1,2,3,4\}$: l'inégalité large $\leq$ et la divisibilité.
\begin{center}
\begin{tabular}{|l|c|c|c|c|}\hline
$\leq$ & $1$ & $2$ & $3$ & $4$ \\ \hline
$1$ 	& oui & oui & oui & oui \\ \hline
$2$ 	& non & oui & oui & oui \\ \hline
$3$ & non & non & oui & oui \\ \hline
$4$	& non & non & non & oui \\ \hline
\end{tabular}
~~~
\begin{tabular}{|l|c|c|c|c|}\hline
divise & $1$ & $2$ & $3$ & $4$ \\ \hline
$1$ 	& oui & oui & oui & oui \\ \hline
$2$ 	& non & oui & non & oui \\ \hline
$3$ & non & non & oui & non \\ \hline
$4$	& non & non & non & oui \\ \hline
\end{tabular}
\end{center}

\end{exemples}

% visualisation d'une relation binaire avec un tableau, pour un ensemble fini

\begin{exemple}[Zérologie]
\begin{enumerate}
\item La \emph{relation vide} est celle dont le graphe est la partie vide de $E\times E$. Pour cette relation, un élément n'est relié à aucun autre : $x\mathcal R y$ est toujours faux.
\item À l'autre extrême, la partie pleine de $E\times E$ est la relation binaire pour laquelle tout élément est relié à tout autre élément : $x\mathcal R y$ est toujours vrai. Son graphe est la partie pleine $E\times E$.
\end{enumerate}
\end{exemple}

\begin{definition}
Soit $E$ un ensemble. Une relation binaire ${\mathcal R}$ sur $E$ est : 
\begin{enumerate}
\item réflexive\index{relation!réflexive} ssi $\forall x\in E, x{\mathcal R}x$;
\item transitive\index{relation!symétrique} ssi $\forall x, y, z\in E, x\mathcal Ry \text{ et } y{\mathcal R}z \implies x{\mathcal R}z$;
\item antisymétrique\index{relation!antisymétrique} ssi $\forall x, y \in E, x{\mathcal R}y\text{ et } y{\mathcal R}x \implies x=y$.
\item symétrique\index{relation!symétrique} ssi $\forall x, y\in E, x\mathcal R y \implies y\mathcal R x$.
\end{enumerate}
\end{definition}

\begin{attention}
Contrairement à ce qu'on peut croire, une relation peut être à la fois symétrique et antisymétrique, comme par exemple la relation $=$. (Si la relation est de plus supposée réflexive, c'est le seul exemple.)
\end{attention}

Le tableau suivant résume sans preuve quelques propriétés des relations classiques. (Certains points seront détaillés par la suite.)\\

\begin{tabular}{|l|c|c|c|c|}\hline
relation & réflexive & transitive & symétrique & antisymétrique \\ \hline
$=$ & oui & oui & oui & oui \\ \hline
$\neq$ & non & non & oui & non \\ \hline
$\leq$ sur $\R$ & oui & oui & non & oui \\ \hline
$<$ sur $\R$ & non & oui & non & oui\footnote{L'implication est vraie car la prémisse est fausse.} \\ \hline
$|$ sur $\N$ & oui & oui & non & oui \\ \hline
$\not |$ sur $\N$ & non & non & non & non \\ \hline
$|$ sur $\Z$ & oui & oui & non & non \\ \hline
$\subseteq$ sur $\mathcal P(E)$ & oui & oui & non & oui \\ \hline
$//$ sur les droites du plan & oui & oui & oui & non \\ \hline
$\perp$ sur les droites du plan & non & non & oui & non \\ \hline
\end{tabular}

\begin{definition}[Raffinement d'une relation]
\index{raffinement!d'une relation}\index{relation!plus fine}
Soient $\mathcal R$ et $\mathcal S$ des relations binaires sur $E$. On dit que $\mathcal R$ est plus fine que $\mathcal S$, ou encore que c'est est un raffinement, si $\forall x, y\in E, x\mathcal R y \implies x\mathcal S y$. 

De façon équivalente, $\mathcal R$ est plus fine que $\mathcal S$ si on a l'inclusion de graphes $\Gamma_{\mathcal R} \subseteq \Gamma_{\mathcal S}$.
\end{definition}

% exemples après, avec les relations d'ordre et d'équivalence
% exemple (zérologie) la relation vide est plus fine que tout autre...


% mettre dans l'exercice sur symétrique ss égal à la transposée ?
\begin{definition}
Soit $E$ un ensemble muni d'une relation binaire $\mathcal R$. La relation \emph{réciproque} ou \emph{transposée}, notée ${}^t\mathcal R$, est définie par : $x{}^t\mathcal R y \iff y\mathcal R x$. 
\end{definition}

% ne pas confondre avec la relation complémentaire / complément logique


\section{Relations d'ordre}

\subsection{Définitions et vocabulaire}


\begin{definition}
Une relation est une \emph{relation d'ordre}\index{relation d'ordre} ssi elle est réflexive, transitive et antisymétrique.

Un ensemble $E$ muni d'une relation d'ordre ${\mathcal R}$ est appelé \emph{ensemble ordonné}\index{ensemble ordonné}.
\end{definition}

\begin{exemples}
\begin{enumerate}[label=\alph*)]
\item La relation $\leq$ est une relation d'ordre sur $\N$, ou sur $\Z$, $\Q$, $\R$. (Mais pas sur $\C$ : la relation $\leq$ n'est même pas \emph{définie} sur $\C$.)
\item La relation $\subseteq$ est une relation d'ordre sur $\mathcal P(E)$. L'antisymétrie est exactement le principe de double inclusion.
\item La relation $|$ (\og divise\fg) est une relation d'ordre sur $\N^*$, ainsi que sur l'ensemble $\N$. Attention: dans $\N$, tout entier divise $0$! 
\end{enumerate}
\end{exemples}

\begin{attention}
\begin{enumerate}
\item La relation $<$ n'est pas une relation d'ordre sur $\R$ (ni sur $\N$, $\Z$ ou $\Q$), car elle n'est pas réflexive, et la relation de divisibilité $|$ n'est pas une relation d'ordre sur $\Z^*$ ni sur $\Z$, car elle n'est pas antisymétrique : $1|-1$ et $-1|1$ et pourtant $1\neq -1$.
% C'est un préordre...
\item il faut systématiquement préciser l'ordre auquel on se réfère, même pour un ensemble \og connu\fg. Par exemple, il faut éviter de parler de \og l'ensemble ordonné $\N$\fg: en effet $\N$ peut être muni de l'ordre usuel $\leq$ ou bien de la divisibilité $|$ et les deux ordres sont fréquemment utilisés.
\end{enumerate}
\end{attention}


\begin{definition}
Si ${\mathcal R}$ est une relation d'ordre sur $E$, on peut lui associer une relation d'\emph{ordre strict}\index{ordre strict}, définie par \og$ x{\mathcal R}y\text{ et }x\neq y$\fg. (Remarque : une relation d'ordre strict n'est pas une relation d'ordre puisqu'elle n'est pas réflexive.)
\end{definition}

\begin{exemple}
Sur $\R$, l'ordre strict associé à la relation d'ordre $\leq$ est l'inégalité stricte $<$.
\end{exemple}


Dans ce cours, on notera en général $\leq_E$ au lieu de ${\mathcal R}$ une relation d'ordre générique sur $E$ (même si la relation n'a rien à voir avec l'inégalité $\leq$ sur $\R$), afin de distinguer les relations d'ordre des relations binaires générales. On notera $<_E$ la relation d'\emph{ordre strict}\index{relation!d'ordre!strict} qui lui est associée.

Enfin, les notations $\geq_E$ et $>_E$ désignent les relations transposées de $\leq_E$ et $<_E$. Autrement dit  $x \geq_E y$ et $x >_E y$ sont synonymes de $y \leq_E x$ et $y <_E x$.

\begin{attention}
Contrairement au cas particulier de $(\R,\leq)$, dans un ensemble ordonné général $(E,\leq_E)$ la négation de \og $x\leq_E y$\fg{} n'est \textbf{pas} \og $y<_E x$\fg. Par exemple, le contraire de \og$2|n$\fg{} n'est pas \og$n$ divise strictement $2$.\fg{}
\end{attention}

\begin{definition}
Une relation d'ordre $\leq_E$ sur un ensemble $E$ est \emph{totale}\index{relation!d'ordre!total} si tous les éléments sont comparables, c'est-à-dire si:
\[ \forall x, y\in E, x\leq_Ey\text{ ou } y\leq_E x.\]
Un ensemble muni d'un ordre total est appelé \emph{ensemble totalement ordonné} \index{ensemble totalement ordonné}. Une relation d'ordre qui n'est pas totale est dite d'ordre \emph{partiel}\index{relation!d'ordre!partiel}.
\end{definition}

\begin{remarque}
Si $x$ et $y$ sont deux éléments d'un ensemble totalement ordonné $(E,\leq_E)$, alors le contraire de $x\leq_E y$ est $x>_E y$. Comme remarqué plus haut, ceci est \textbf{faux} si l'ordre n'est pas total, justement à cause d'éventuels éléments non comparables. 
\end{remarque}

\begin{exemples}
La relation d'ordre $\leq$ sur $\R$ (ou $\N$, $\Q$ ou  $\Z$) est totale. Par contre, $\subseteq$ et $|$ ne sont pas totales. Par exemple, dans $\mathcal P(\R)$, les parties $\R_+$ et $]-3,6]$ ne sont pas comparables pour l'inclusion. Dans $\N^*$, les éléments $2$ et $3$ ne sont pas comparables pour la divisibilité.
\end{exemples}

\begin{exercice}
Soit $E$ un ensemble. Montrer que \og est plus fine que\fg{} est une relation d'ordre sur l'ensemble des relations binaires sur $E$.
\end{exercice}

\subsection{Applications croissantes, décroissantes, monotones}


\begin{definition}
\index{application!croissante}\index{application!strictement croissante}
\index{application!décroissante}\index{application!strictement décroissante}
\index{application!monotone}\index{application!strictement monotone}
Soient $(E,\leq_E)$ et $(F,\leq_F)$ des ensembles ordonnés et $f : E\to F$. On dit que $f$ est 
\begin{enumerate}
\item \emph{croissante} si $\forall x, y\in E, x\leq_E y \implies f(x) \leq_F f(y)$;
\item \emph{décroissante} si $\forall x, y\in E, x\leq_E y \implies f(x) \geq_F f(y)$;
\item \emph{monotone} si elle est croissante ou décroissante;
\item \emph{strictement croissante} si $\forall x, y\in E, x<_E y \implies f(x) <_F f(y)$;
\item \emph{strictement décroissante} si $\forall x, y\in E, x<_E y \implies f(x) >_F f(y)$;
\item \emph{strictement monotone} si elle est strictement croissante ou strictement décroissante.
\end{enumerate}
\end{definition}

(Remarque : dans cette situation, il est important de distinguer les relations d'ordre sur $E$ et sur $F$.)

\begin{exemple}
\begin{enumerate}
\item L'application $f : \R\to \R, x\mapsto x+e^x$ est croissante pour l'ordre usuel $\leq $ sur $\R$.
\item Si $E$ est fini, l'application $f : \mathcal P(E) \to \N, \: A\mapsto \operatorname{Card}(A)$ est croissante entre les ensembles ordonnés $(\mathcal P(E), \subseteq)$ et $(\N, \leq)$.
\item L'application $f : \mathcal P(E) \to \mathcal P(E), \: A\mapsto \complement A$ est décroissante pour l'inclusion, car $A\subseteq B \implies \complement B\subseteq \complement A$.
\item Une application \emph{décroissante} entre $(E,\leq_E)$ et $(F,\leq_F)$ est la même chose qu'une application \emph{croissante} entre $(E,\leq_E)$ et $(F,\geq_F)$.
\end{enumerate}
\end{exemple}

\begin{exercice}
Montrer qu'une application strictement croissante entre ensembles totalement ordonnés est injective.
\end{exercice}


\begin{proposition}
\begin{enumerate}
\item La composée de deux applications croissantes est croissante.
\item La composée de deux applications décroissantes est croissante.
\item La composée d'une application décroissante et d'une croissante est décroissante.
\end{enumerate}
\end{proposition}

\begin{proof}
Application directe de la définition.
\end{proof}

\subsection{Plus grand et plus petit élément}
\begin{definition}
Soit $(E,\leq_E)$ un ensemble ordonné et $A\subseteq E$ une partie non vide.
\begin{enumerate}
\item Un élément $m\in E$ est un \emph{majorant}\index{majorant} de $A$ si $\forall a\in A, a\leq_E m$.
\item La partie $A$ est \emph{majorée} si elle possède des majorants.
\item Un élément $m\in A$ qui est un majorant de $A$ est appelé un \emph{plus grand élément de $A$}\index{plus grand élément}, ou \emph{maximum}\index{maximum} de $A$.
\item On définit de même les \emph{minorants}\index{minorant}, les parties minorées et les plus petits éléments\index{plus petit élément}.
\end{enumerate}
\end{definition}

\begin{exemple}
\begin{enumerate}[label=\alph*)]
\item Dans l'ensemble ordonné $(\R,\leq)$, la partie $[2,5]$ est majorée par $5$, mais aussi par $6$, $10$ etc. La partie $\R_+$ est minorée, mais pas majorée. La partie $\Z$ n'est ni minorée ni majorée.
\item Toute partie non vide de $\N$ admet un plus petit élément pour l'ordre usuel $\leq$ (c'est la propriété fondamentale de $\N$), mais pas forcément de plus grand élément.
\item Dans un ensemble ordonné non vide $(E,\leq_E)$, la partie vide est majorée: tout élément $m$ est un majorant, car l'assertion $\forall x\in \varnothing, x\leq_E m$ est vraie. De la même façon, dans un ensemble non-vide, la partie vide est minorée par n'importe quel élément.
\end{enumerate}
\end{exemple}

\begin{remarque}
Attention aux reformulations hâtives. Si $x\in E$ est un élément qui ne possède aucun majorant strict, on ne peut pas pour autant en conclure que $x$ est un plus grand élément de $E$. Par exemple, dans l'ensemble ordonné $\{2,3,4\}$ muni de la divisibilité, l'élément $4$ ne possède aucun majorant strict, pourtant il ne majore pas $3$ : encore une fois, cela est dû au fait que l'ordre n'est pas forcément total, et que le contraire de $x\geq y$ n'est pas $x<y$.
\end{remarque}



\begin{proposition}[Unicité du plus grand élément, s'il existe]
Si $A\subseteq E$ possède un plus grand élément, il est unique. On le note alors $\max(A)$.
De même, si $A\subseteq E$ possède un plus petit élément, il est unique. On le note alors $\min(A)$.
\end{proposition}
\begin{proof}
Soient $m$ et $m'$ deux plus grands éléments de $A$. Comme $m$ est un plus grand élément, on a par définition $\forall x\in A, x\leq_E m$ et donc en particulier $m'\leq_E m$. De même, comme $m'$ est un plus grand élément, on a $m\leq_E m'$. Par antisymétrie de la relation d'ordre, on a $m=m'$.

On prouve le résultat pour le plus petit élément de la même manière.
\end{proof}

\begin{exemples}
\begin{enumerate}[label=\alph*)]
\item  La partie $[0,1]$ est majorée dans $\R$ car $1$, $2$ ou encore $5$ sont des majorants. Elle possède un plus grand élément : $1$.
\item La partie $]3,+\infty[$ de $\R$ n'a pas de plus grand élément car elle n'est pas majorée.
\item La partie $A=[0,1[$ de $\R$ est majorée. Par contre, elle n'a pas de plus grand élément. 
\item La partie $B=\{x\in \Q\:\mid\: x^2\leq 2\}$ est majorée (par $\sqrt2$ par exemple), mais n'admet pas de plus grand élément (rappel : $\sqrt 2 \not\in\Q$).
\item Si $E$ est un ensemble, alors $\mathcal P(E)$ muni de l'inclusion possède un plus grand élément : $E$, et un plus petit élément : $\varnothing$.
\item Dans l'ensemble ordonné $(\N^*,|)$, la partie $\{2,3,4\}$ n'a pas de plus grand élément.
\item Dans l'ensemble ordonné $(\N,|)$, il y a un plus petit élément au sens de la divisibilité, c'est $1$ (et non zéro). D'autre part, l'élément $0$ est en fait le plus grand élément au sens de la divisibilité: tout nombre entier $k$ divise $0$.
\end{enumerate}
\end{exemples}

\subsection{Borne supérieure, borne inférieure}

\begin{definition}[Borne supérieure]
 La partie $A\subseteq E$ admet une borne supérieure\index{borne supérieure} $s\in E$ ssi:
\begin{enumerate}
\item $s$ est un majorant de $A$;
\item tout majorant de $A$ majore $s$.
\end{enumerate}
(En d'autres termes, $s$ est le plus petit des majorants de $A$, ou encore : l'ensemble de tous les majorants de $A$ possède un plus petit élément $s$.)
\end{definition}

Attention, contrairement à un plus grand élément, une  borne supérieure de $A$, s'il en existe, n'appartient pas forcément à $A$. 

\begin{proposition}[Unicité de la borne supérieure, s'il en  existe une]
Soit $(E,\leq_E)$ un ensemble ordonné, et $A\subseteq E$. Si $A$ possède une borne supérieure, elle est unique et on la note $\sup(A)$.
\end{proposition}
\begin{proof}
Soient $s$ et $s'$ deux bornes supérieures de $A$. Comme $s$ est une borne supérieure et $s'$ un majorant, on a $s\leq_E s'$. Un raisonnement symétrique montre que  $s'\leq_E s$, et finalement $s'=s$.
\end{proof}

\begin{exemple}
La partie $\R_+ \subseteq \R$ n'a pas de borne supérieure. 
La partie $A=[0,1[ \subseteq \R$ n'a pas de plus grand élément, mais possède une borne supérieure : $1$.
\end{exemple}
\begin{proof} Pour le premier point, la partie n'a même pas de majorant donc c'est clair. 
D'une part, il est clair que $1$ est un majorant de $[0,1[$, c'est-à-dire que $\forall x\in [0,1[, \: x\leq 1$.

Vérifions la seconde partie de la définition.  Il s'agit de montrer qu'un élément $m$ de $[0,1[$ ne peut pas être un majorant de $[0,1[$. Mais si $m \in [0,1[$, alors on peut considérer le réel $m'=m+\frac{1-m}{2}$. Comme $0\leq m< 1$, on a l'encadrement $0< \frac{1-m}{2} < 1-m$ et donc en sommant $m$ on obtient
\[
m< m' < 1
\]
\begin{center}
\begin{tikzpicture}[line cap=round,line join=round,>=triangle 45,x=1.0cm,y=1.0cm]
\clip(-0.5,-1.5) rectangle (10.5,1);
\draw (0,0)-- (10,0);
\begin{scriptsize}
\draw[color=black] (0.0,-0.5) node {$0$};
\draw [fill=black] (6,0) circle (2pt);
\draw[color=black] (6.0,-0.5) node {$m$};
\draw [fill=black] (8,0) circle (2pt);
\draw[color=black] (8.0,-0.5) node {$m'=m+\frac{1-m}{2}$};
\draw[color=black] (10.0,-0.5) node {$1$};
\end{scriptsize}
\end{tikzpicture}
\end{center}
Ceci montre que $m$ ne majore pas $m'$, qui est dans $[0,1[$. Donc $m$ n'est pas un majorant de $[0,1[$.
\end{proof}

Autre exemple important de borne supérieure qui n'est pas un plus grand élément: la partie $\{x\in \Q, x^2<2\}$ de $\R$ est majorée et admet une borne supérieure égale à $\sqrt 2$ et qui n'appartient pas à $A$ car $\sqrt2\not\in \Q$. 

\begin{proposition}
Soit $(E,\leq_E)$ un ensemble ordonné $A\subseteq E$.
Si $A$ admet une borne supérieure et que $\sup(A) \in A$, alors c'est son plus grand élément.
Si $A$ admet un plus grand élément, c'est aussi sa borne supérieure.
\end{proposition}
\begin{proof}
Exercice, appliquer les définitions.
\end{proof}

Enfin, on définit de même ce qu'est une \emph{borne inférieure}\index{borne inférieure}, et on montre que si une partie admet une borne inférieure, alors celle-ci est unique. On la note $\inf(A)$. 

La borne inférieure d'une partie, même si elle existe, n'appartient pas forcément à la partie. Par exemple, $0$ est la borne inférieure de $]0,1]$.

\begin{theoreme}[$\R$ possède la propriété de la borne supérieure]
Dans $(\R,\leq)$, toute partie non vide et majorée admet une borne supérieure.
\end{theoreme}
\begin{proof}
Admis provisoirement. Pour prouver ce théorème, il faut disposer d'une définition rigoureuse de l'ensemble $\R$. Voir le cours d'analyse de second semestre.
\end{proof}

Il existe des ensembles ordonnés ne possédant pas la propriété de la borne supérieure, c'est-à-dire possédant des parties non-vides, majorées, et sans borne supérieure. C'est le cas de $(\Q,\leq)$, si l'on considère la partie $\{x\in \Q\:\mid\: x^2\leq 2\}$ : il n'existe pas de borne supérieure de cette partie \underline{dans $\Q$}.


\subsection{Ordre produit et ordre lexicographique}

\begin{propdef}
Soient $(E\leq_E)$ et $(F,\leq_F)$ des ensembles ordonnés.
L'\emph{ordre produit}\index{relation!d'ordre!produit} sur $E\times F$ est défini par :
\[
(x,y) \leq_{E\times F} (x',y') \iff \left(x\leq_E x' \text{ et } y\leq_F y'\right).
\]
\end{propdef}
\begin{proof}
Il s'agit de prouver que la relation binaire définie est bien une relation d'ordre donc réflexive, antisymétrique et transitive. Exercice.
\end{proof}

Attention, même si $\leq_E$ et $\leq_F$ sont totales, l'ordre produit n'est pas forcément un ordre total. Par exemple, pour $E=F=\R$ et l'ordre usuel sur $\R$ qui est bien total, on remarque que l'ordre produit $\leq_{\R\times\R}$ sur $\R\times \R$ n'est pas total car $(1,2)$ et $(2,1)$ ne sont pas comparables.

\begin{propdef}
Soient $(E\leq_E)$ et $F,\leq_F)$ des ensembles \textbf{totalement} ordonnés.
L'ordre lexicographique\index{relation!d'ordre!lexicographique} sur $E\times F$ est défini par :
\[
(x,y) \leq_{E\times F} (x',y') \iff \left(x<_E x' \text{ ou } (x=x' \text{ et } y\leq_F y')\right).
\]
C'est un ordre total.
\end{propdef}
\begin{proof}
La propriété de relation d'ordre est laissée en exercice. Prouvons que l'ordre est total.

Soient en effet $(x,y)$ et $(x',y')$ distincts.
Si $x\neq x'$, alors comme $\leq_E$ est un ordre total, on a forcément $x<_E x'$ ou bien $x'<_E x$.
Si $x=x'$, alors on a forcément $y\neq y'$ et comme $\leq_F$ est un ordre total, on a forcément $y <_F y'$ ou bien $y'<_F y$.

En conclusion, on a bien soit $(x,y) \leq_{E\times F} (x',y')$, soit  $(x',y') \leq_{E\times F} (x,y)$.
\end{proof}

\begin{exemple}
Avec l'ordre usuel sur l'alphabet, l'ordre lexicographique sur les mots est l'ordre dans lequel les mots sont classés dans un dictionnaire.
\end{exemple}



\subsection{Exercices d'application directe}


\begin{exercice}
Montrer qu'une relation $\mathcal R$ est symétrique si et seulement si $\mathcal R = {}^t\mathcal R$.
\end{exercice}
\begin{exercice}
Montrer qu'un ensemble fini de cardinal $n$ possède $2^{n^2-n}$ relations réflexives et $2^{n(n+1)/2}$ relations symétriques.
\end{exercice}

%%%%%%%%%%%%%%%%%%%%%%%%%%%%%%%%%%%%%%%%%%%%%%%%%%%%%%%

%%%%%%%%%%%%%%%%%%%%%%%%%%%%%%%%%%%%%%%%%%%%%%%%%%%%%%%

%%%%%%%%%%%%%%%%%%%%%%%%%%%%%%%%%%%%%%%%%%%%%%%%%%%%%%%


\section{Relations d'équivalence}

\subsection{Définitions}

\begin{definition}[Relation d'équivalence]
Une relation binaire ${\mathcal R}$ sur un ensemble $E$ est une \emph{relation d'équivalence}\index{relation!d'équivalence} ssi elle est:
\begin{enumerate}
\item réflexive (rappel : $\forall x\in E, x{\mathcal R}x$);
\item transitive (rappel : $\forall x, y, z\in E, x{\mathcal R}y\text{ et } y{\mathcal R}z \implies x{\mathcal R}z$);
\item symétrique (rappel : $\forall x, y\in E, x{\mathcal R}y \implies yRx$).
\end{enumerate}
\end{definition}

\begin{exemples}
\begin{enumerate}[label=\alph*)]
\item Les relations $=$, $//$ (parallélisme), sont des relations d'équivalence.
\item La relation $\perp$ (perpendiculaire) n'est \textbf{pas} une relation d'équivalence car elle n'est pas réflexive, ni transitive.
\item Tout ensemble possède la relation d'équivalence triviale : celle où tous les éléments sont équivalents.
\item Sur $\R$, la relation $x\mathcal R y \iff \left(x=y\text{ ou }x=-y\right)$ est une relation d'équivalence.
\item L'ensemble vide\index{ensemble vide} possède une seule relation d'équivalence, la relation vide (la seule fonction de $\varnothing \times \varnothing$ dans $\{\text{vrai,\: faux}\}$ à savoir la fonction vide : on vérifie qu'elle définit bien une relation d'équivalence).
\item Un singleton\index{singleton}, c'est-à-dire un ensemble contenant un unique élément, possède une seule relation d'équivalence (celle où l'unique élément est relié à lui-même).
\item Un ensemble $\{a,b\}$ à deux éléments possède deux relations d'équivalence distinctes : la première est l'\emph{égalité}, la seconde est la \emph{relation d'équivalence triviale}, celle où $a$ et $b$ sont équivalents.
\item Un ensemble à trois éléments possède cinq relations d'équivalence (exercice).
\end{enumerate}
\end{exemples}

\begin{proposition}[Relation donnée par les fibres d'une application]\index{fibre}
Si $f : E\to F$ est une application, alors la relation 
\[
x\mathcal R y \iff (x\text{ et }y \text{ sont dans la même fibre de }f)
\]
est une relation d'équivalence sur $E$.

(Rappelons que par définition de ce que sont les fibres d'une application, on peut reformuler la définition de cette relation en: $x\mathcal R y \iff f(x)=f(y)$.)
\end{proposition}
\begin{proof}
Soit $x \in E$. Alors on a bien $f(x)=f(x)$ donc $x\mathcal R x$, donc $\mathcal R$ est réflexive. Si $x, y\in E$, on a bien $x\mathcal R y \iff f(x)=f(y) \iff f(y)=f(x) \iff y\mathcal x$ donc $\mathcal R$ est symétrique. Et enfin, Si $x,y,z\in E$ et que $x\mathcal R y$ et $y\mathcal R z$, alors $f(x)=f(y)$ et $f(y)=f(z)$, donc $f(x)=f(z)$ et donc $x\mathcal R z$, donc $\mathcal R$ est transitive. Ceci montre que $\mathcal R$ est bien une relation d'équivalence.

On verra dans la dernière section que toutes les relations d'équivalence sont de ce type, pour une application $f$ bien choisie : la \emph{surjection canonique sur le quotient}.
\end{proof}

\begin{exemple} La proposition précédente implique que les relations suivantes sont des relations d'équivalence:
\begin{enumerate}
\item Sur $\R$, la relation définie par $x\mathcal R y \iff \sin(x)=\sin(y)$;
\item Sur $\C$, la relation définie par $z\mathcal R z' \iff |z|=|z'|$.
\item Sur $\R$, la relation définie par $x\mathcal R y \iff xe^y=ye^x$. (Dans ce cas, la fonction est $f :\R\to \R, t\mapsto te^{-t}$.)
\end{enumerate}
\end{exemple}

D'autres exemples importants de relations d'équivalence sont les congruences. Commençons par rappeler les définitions.

\begin{propdef}
Soit $a\in \R^*$ et $x,y\in\R$. On a :
\[
\frac{x-y}{a}\in \Z 
\iff
x-y \in a\Z
\iff 
(\exists k\in \Z\:\mid\: x=y+ka).
\]
Si ces conditions équivalentes sont vérifiées, on dit que $x$ et $y$ sont \emph{congrus modulo $a$} et on note 
\[ x\equiv y\quad [a].\]
La relation de congruence\index{congruence!modulo un réel} modulo $a$ entre deux réels $x$ et $y$ est une relation d'équivalence.
\end{propdef}
\begin{proof}
Exercice.
\end{proof}

De toutes ces formulations, la plus efficace pour rédiger des preuves est en général la première.

Les relations de congruence les plus courantes sont celles modulo des entiers, ou bien modulo $\pi$ ou $2\pi$ etc.


\begin{exemples}
\begin{enumerate}[label=\alph*)]
\item $1 \equiv 5 \: [2]$, car $1-5 = -4$ est un multiple de $2$.
\item $4\equiv -9\sqrt{3}+4 \: [\sqrt{3}]$, car $4 - (-9\sqrt{3}+4) = 9\sqrt{3}$ est un multiple de $\sqrt{3}$.
\item $\pi/3 \equiv 7\pi/3 \: [2\pi]$, car $\pi/3 - 13\pi/3 = -12\pi/3 = -4\pi$ est un multiple de $2\pi$.
\end{enumerate}
\end{exemples}

Les congruences se comportent relativement bien par rapport aux opération algébriques, comme le montre la proposition suivante (avec un bémol pour la multiplication, voir l'énoncé et la remarque qui suit).

\begin{proposition} Soit $a\neq 0$ et $b\neq 0$ des réels non nuls, et $x$, $y$, $x'$, $y'$ des réels tels que $x \equiv y\: [a]$ et $x' \equiv y'\: [a]$. Alors:
\begin{itemize}
\item[i)]{$x+x' \equiv y+y'\: [a]$.}
\item[ii)]{$bx\equiv by \: [ba]$.}
\end{itemize}
\end{proposition}
\begin{proof}
\begin{itemize}
\item[i)]{Si $\frac{x-y}{a} \in \Z$ et $\frac{x'-y'}{a} \in \Z$, alors $\frac{x-y}{a}+\frac{x'-y'}{a} \in \Z$.

On a donc $\frac{(x+x')-(y+y')}{a} \in \Z$, c'est-à-dire $x+x' \equiv y+y'\: [a]$.}
\item[ii)]{On a:
\[\left(x\equiv y \: [a]\right) \Leftrightarrow  \left(\frac{x-y}{a} \in \Z\right) \Leftrightarrow \left(\frac{bx-by}{ba} \in \Z\right) \Leftrightarrow \left(bx\equiv by \: [ba]\right).\]}
\end{itemize}
\end{proof}

\begin{remarque}
Attention au second point, multiplier une congruence par $b$ change la base de congruence, qui est également multipliée par $b$.
\end{remarque}

\begin{definition}
Si $a\in \Z^*$, la relation de congruence modulo $a$ sur $\Z$ induit une relation d'équivalence sur $\Z$, également appelée la relation de congruence modulo $a$ sur $\Z$, et notée de la même façon.
\end{definition}

\subsection{Classes d'équivalence}
\begin{definition}[Classe d'équivalence]\index{classe d'équivalence}
Soit $E$ un ensemble muni d'une relation d'équivalence ${\mathcal R}$. Soit $x\in E$. On note $\overline{x}$ (ou parfois $Cl(x)$) et on appelle la \emph{classe d'équivalence de $x$ modulo $\mathcal R$ (ou : sous $\mathcal R$)} l'ensemble  de tous les éléments qui sont équivalents à $x$, c'est-à-dire l'ensemble:
\[
\overline x = \left\{y\in E\:\mid\: y{\mathcal R}x\right\}
\]


Soit $A\in \mathcal P(E)$ une partie de $E$. On dit que $A$ est une \emph{classe d'équivalence modulo $\mathcal R$} si c'est la classe d'équivalence d'un certain élément, c'est-à-dire si : $\exists x\in E, \: A=\overline x$.
\end{definition}

Attention au type des objets : $x \in E$, mais $\overline{x} \subseteq E$.

\begin{proposition}
\begin{enumerate}
\item $\forall x\in E, x\in \overline{x}$ (en particulier une classe d'équivalence n'est jamais vide).
\item $\forall x, y\in E, x{\mathcal R}y \iff \overline{x}=\overline{y}$.
\item $\forall x, y\in E, \overline{x} = \overline{y} \text{ ou } \overline{x}\cap \overline{y}=\varnothing$. (Deux classes sont égales ou disjointes.)
\end{enumerate}
\end{proposition}
\begin{proof}
\begin{enumerate}
\item Découle de la réflexivité.
\item Sens $\impliedby$ : Supposons $\overline{x}=\overline{y}$. Comme $y\in \overline{y}$, on a $y\in \overline{x}$, donc $y{\mathcal R}x$.\\
Sens $\implies$ : Soit $z\in \overline{x}$. Alors $z{\mathcal R}x$ et comme $x{\mathcal R}y$, on a $z{\mathcal R}y$ par transitivité, et donc $z\in \overline{y}$. Ceci montre $\overline{x}\subseteq \overline{y}$. Pour montrer l'inclusion réciproque, on a $y{\mathcal R}x$ par symétrie de $R$ puis on termine de la même manière.
\item Soient $x$ et $y$, et supposons $\overline{x}\cap \overline{y} \neq \varnothing$. Soit $z\in \overline{x}\cap \overline{y}$. Alors $z\mathcal R x$ et $z\mathcal R y$, donc par symétrie et transitivité, $x\mathcal R y$, d'où $\overline{x}=\overline{y}$.
\end{enumerate}
\end{proof}

\begin{definition}[Partie saturée]
\index{partie!saturée sous une relation d'équivalence}\index{saturation!relativement à une relation d'équivalence}
Soit $E$ un ensemble muni d'une relation d'équivalence ${\mathcal R}$ et soit $A\in \mathcal P(E)$. La \emph{saturation} de $A$ relativement à la relation $\mathcal R$ est la partie
\[
\{y\in E\:|\: \exists x\in A, y\mathcal R x\} 
\]
On dit qu'une partie est \emph{saturée} (relativement à $\mathcal R$), si elle est égale à sa saturation.
\end{definition}

\begin{remarque} La saturation de $A$ est égale à $\bigcup_{x\in A} \overline x$. En effet, si $y\in E$, alors 
\[
\exists x\in A, y\mathcal R x
\iff \exists x\in A, y\in \overline x
\iff y\in \bigcup_{x\in A} \overline x 
\]
\end{remarque}

\begin{exercice} Montrer qu'une classe d'équivalence est saturée, mais que la réciproque est fausse en général. Montrer que les classes d'équivalence sont exactement les parties saturées minimales pour l'inclusion.
\end{exercice}

L'outil principal pour manipuler les classes d'équivalence est l'\emph{ensemble quotient}, que l'on définit maintenant.

\begin{definition}

L'ensemble des classes d'équivalence est appelé \emph{ensemble quotient de $E$ par $\mathcal R$}\index{ensemble quotient sous une relation d'équivalence} et est noté $E/{\mathcal R}$.

L'application $p : E \to E/\mathcal R, x\mapsto \overline{x}$ qui à un élément de $E$ lui associe sa classe d'équivalence est appelée \emph{application de passage au quotient}, ou \emph{projection canonique sur le quotient}\index{projection canonique sur le quotient}\index{application de passage au quotient}. (Cette application étant surjective, on l'appelle aussi la \emph{surjection canonique sur le quotient}\index{surjection canonique sur le quotient}.)
\end{definition}

\begin{exemple}
Pour l'ensemble $E$ des droites du plan $\R^2$ muni de la relation d'équivalence $//$, les classes d'équivalence sont appelées \emph{directions} : deux droites sont parallèles si et seulement si elles ont la même \emph{direction}. L'ensemble quotient de $E$ par la relation de parallélisme est l'ensemble des directions du plan. On le note $\P^1(\R)$ et on l'appelle la droite projective.
\end{exemple}

\begin{proposition}
Soit $E$ un ensemble muni d'une relation d'équivalence ${\mathcal R}$ et $p = E\to E/\mathcal R$ la projection sur le quotient. Alors 
\begin{enumerate}
\item $p$ est surjective.
\item $x\mathcal R y \iff p(x)=p(y)$.
\item Les fibres\index{fibre} sont exactement les classes d'équivalence modulo la relation $\mathcal R$.
\end{enumerate}
\end{proposition}
\begin{proof}
\begin{enumerate}
\item Soit $A$ une classe d'équivalence. Par définition, il existe $x\in E$ tel que $A = \overline{x} = p(x)$. Donc $p$ est surjective.
\item On a $x\mathcal R y \iff \overline{x} = \overline{y} \iff p(x)=p(y)$.
\item C'est une relation du deuxième point, puisque par définition de la fibre d'une application quelconque $f$, deux éléments $x$ et $y$ sont dans la même fibre si et seulement $f(x)=f(y)$.
\end{enumerate}
\end{proof}




\subsection{Partitions et classes d'équivalence}

\begin{definition}[Partition d'un ensemble]
Soit $E$ un ensemble, et $\mathcal A$ un ensemble de parties de $E$, c'est-à-dire $\mathcal A\subset \mathcal P(E)$. L'ensemble $\mathcal A$ est une \emph{partition de $E$ en ensembles non vides}, ou simplement \emph{partition\footnote{Notation adoptée dans tout ce cours} de $E$}\index{partition}, si : 
\begin{enumerate}
\item les parties sont non vides c'est-à-dire $\forall A\in \mathcal A, \: A\neq \varnothing$.
\item les parties recouvrent $E$ c'est-à-dire que leur union égale $E$, autrement dit $\bigcup_{A\in \mathcal A} A = E$.
\item Les parties sont deux à deux disjointes, c'est-à-dire $\forall A, A' \in \mathcal A, \: A\cap A' = \varnothing$.
\end{enumerate}
\end{definition}

\begin{exemple}[Partition définie par une famille]
Soit $E$ un ensemble, et soit $(A_i)_{i\in I}$ une famille de parties de $E$. Cette famille définit une partition de $E$ si :
\begin{enumerate}
\item Les $A_i$ sont toutes non vides.
\item On a $\bigcup_{i\in I} A_i = E$.
\item Les parties $A_i$ sont deux à deux disjointes : $\forall i, j\in I, i\neq j \implies A_i\cap A_j=\varnothing$.
\end{enumerate}
\end{exemple}

\begin{exemple}
\begin{enumerate}
\item L'ensemble vide\index{ensemble vide} possède une seule partition, la partition vide qui ne contient aucune partie (car les parties elles, doivent être non-vides).
\item Un ensemble $\{a,b\}$ à deux éléments possède deux partitions : la partition triviale $\big\{\: \{a,b\} \:\big\}$ et la partition en deux singletons $\big\{\: \{a\},\{b\} \:\big\}$.
\item Un ensemble à trois éléments possède cinq partitions distinctes (exercice).
\item Tout ensemble possède la partition $\big\{\: \{x\}\:\mid\: x\in E \:\big\}$ qui est la partition en singletons inclus dans $E$. (Si $E$ est vide, la partition est vide).\index{partition!en singletons}
\item Tout ensemble non-vide $E$ possède toujours au moins la partition triviale\index{partition!triviale} en une seule partie, l'ensemble lui-même. C'est bien une partition car $E$ est non-vide. Cette partition s'écrit donc $\big\{\: E \:\big\}$.
\item L'ensemble $\Z$ possède la partition en deux parties suivante : $\big\{\: 2\Z, \: 2\Z+1\:\big\}$. C'est la partition en nombres pairs et nombres impairs.
\end{enumerate}
\end{exemple}

\begin{definition}[Raffinement d'une partition]
\index{raffinement!d'une partition}\index{partition!plus fine}
Soient $\mathcal A$ et $\mathcal B$ deux partitions de $E$. On dit que $\mathcal A$ est \emph{plus fine} que $\mathcal B$ (ou : qu'elle est un raffinement de $\mathcal B$) si les éléments de $\mathcal B$ sont des unions d'éléments de $\mathcal A$, autrement dit si $\mathcal A$ fractionne les éléments de $\mathcal B$ en sous-parties.
\end{definition}

\begin{exercice}
Soient $\mathcal R$ et $\mathcal S$ deux relations d'équivalence sur $E$. Montrer que $\mathcal R$ est plus fine que $\mathcal S$ si et seulement si la partition en classes d'équivalence modulo $\mathcal R$ est plus fine que la partition en classe d'équivalence modulo $\mathcal S$.
\end{exercice}

\begin{exercice}
Montrer que la relation binaire \og être plus fine que\fg{} est une relation d'ordre sur l'ensemble de toutes les partitions de $E$, et que l'ordre n'est en général pas total.

Si $E$ est un ensemble à trois éléments, dire, parmi les cinq partitions possibles, lesquelles sont comparables.
\end{exercice}


\begin{remarque}
Le plus grand élément de cet ensemble ordonné est la partition la  plus fine\index{partition!la plus fine} : c'est la partition en singletons, c'est-à-dire l'ensemble de tous les singletons inclus dans $E$. Cette partition est plus fine que toute autre. Si $E$ est non-vide, la partition la moins fine\index{partition!la moins fine} est la partition triviale (celle à une seule partie).  
\end{remarque}



\begin{proposition}[Partition en classes d'équivalence]\index{partition!en classes d'équivalence}
Soit $\mathcal R$ une relation d'équivalence sur $E$. Alors $E/\mathcal R$ est une partition de $E$.
\end{proposition}
\begin{proof}
\begin{enumerate}
\item Une classe d'équivalence n'est jamais vide, puisque qu'elle est toujours de la forme $\overline{x}$ et donc contient un élément $x$.
\item Soit $a\in E$. On a $\overline{a} \in E/\mathcal R$, et $a\in \overline{a}$. Donc $a\in \bigcup_{A\in E/\mathcal R} A$. On en déduit que $E\subseteq \bigcup_{A\in E/\mathcal R} A$.
\item On a déjà montré que deux classes d'équivalence sont soit égales soit disjointes.
\end{enumerate}
\end{proof}

\begin{remarque}[Zérologie] Le quotient de l'ensemble vide\index{ensemble vide} par son unique relation d'équivalence est l'ensemble des classes d'équivalence : comme il n'y a aucune classe d'équivalence, l'ensemble quotient est vide. La projection canonique est l'application $p : \varnothing \to \varnothing$ (dite application vide). Elle est bien surjective...
\end{remarque}

Ce résultat admet une \og réciproque\fg : 

\begin{proposition}
Soit $\{A_i\:\mid i\in I\}$ une partition d'un ensemble $E$. Alors la relation
\[
x\mathcal R y \iff \left( \exists i\in I\:,\: x\in A_i\text{ et } y\in A_i\right)
\]
est une relation d'équivalence.
\end{proposition}
\begin{proof}
Voir TD.
\end{proof}

Ces deux propositions permettent de montrer qu'\og une relation d'équivalence sur $E$ est la même chose qu'une partition de $E$\fg : attention, à proprement parler ce ne sont pas les mêmes objets (pas le même type), mais le sens précis de cette phrase est qu'il existe une bijection entre d'une part l'ensemble des relations d'équivalence sur $E$, et d'autre part, l'ensemble des partitions de $E$.

\begin{corollaire}[Application à la combinatoire]
Si $\mathcal R$ est une relation d'équivalence sur un ensemble fini $E$, alors $|E| = \sum_{A\in E/\mathcal R} |A|$.
\end{corollaire}
\begin{proof}
On a  $E= \bigcup_{A\in E/\mathcal R} A$ et l'union est disjointe, donc on obtient le résultat en prenant le cardinal des deux membres.
\end{proof}

% chercher un exo d'application avec des bijections : nombres de points fixes

\begin{exercice}
\index{involution}\index{point fixe}
Soit $E$ un ensemble fini et $f : E\to E$ une involution, c'est-à-dire vérifiant $f\circ f = \Id_E$.
\begin{enumerate}
\item Montrer que si $f$ n'a pas de point fixes, alors $|E|$ est pair.
\item Plus généralement, montrer que $|E|$ a la même parité que le nombre de points fixes de $f$.
\end{enumerate}
\end{exercice}

\section{Compléments sur le passage au quotient}

\begin{propdef}[Passage au/factorisation par le quotient]
Soit $f : E\to F$ une application, $\mathcal R$ une relation d'équivalence sur $E$  et $p : E\to E/\mathcal R$ la projection canonique sur le quotient.
On dit que $f$ \emph{passe (ou descend) au quotient} si elle se factorise à droite par $p$, autrement dit s'il existe une application $\overline f : E/\mathcal R\to F$ telle que $f = \overline{f} \circ p$, autrement dit telle que le diagramme suivant commute:
\[
\xymatrix{
E \ar[r]^f \ar[d]^p & F\\
E/\mathcal R \ar[ur]_{\overline f}& 
}
\] 
Si une telle application $\overline{f}$ existe, elle est unique.
\end{propdef}
\begin{proof}(de l'unicité). Soit $\alpha \in E/\mathcal R$. Comme la projection canonique $p : E\to E/\mathcal R$ est surjective, il existe $x\in E$ tel que $\alpha=p(x)$. Or par hypothèse, on a $f = \overline{f} \circ p$, donc $\overline f(\alpha) =\overline f(p(x))= f(x)$. Ceci montre que les valeurs de $\overline f$ sont déterminées par la fonction $f$.
\end{proof}


\begin{proposition}[Condition nécessaire et suffisante de passage au quotient]
Soit $f : E\to F$ une application, et $\mathcal R$ une relation d'équivalence sur $E$.

Alors, $f$ descend au quotient en une application $\overline f : E/\mathcal R \to F$ si et seulement si elle est constante sur les classes d'équivalence, c'est-à-dire si et seulement si:
\[
\forall x,y\in E, x\mathcal R y \implies f(x)=f(y).
\]
\end{proposition}

% référer aux lemmes de factorisation généraux dans "applications"


\section{Exercices d'approfondissement}

% cloture transitive, réflexive, façons de symétriser une relation etc etc. Relations asymétriques, pré-odres

\begin{exercice}[Pré-ordre]
Une relation binaire est appelée un \emph{pré-ordre} si elle est réflexive et transitive. (Ainsi les relations d'ordre et d'équivalence sont donc toutes deux des cas particuliers de pré-ordres.) Un ensemble muni d'une telle relation est dit \emph{pré-ordonné}.

Soit $(E,\leftarrow)$ un ensemble pré-ordonné. On définit la relation binaire $\mathcal R$ comme suit : 
\[ \forall x, y \in E, \: x\mathcal R y \iff (x\leftarrow y \text{ et } y\leftarrow x)\]
Montrer que $\mathcal R$ est une relation d'équivalence sur $E$.
\end{exercice}

% relation d'équivalence ou d'ordre associée à un préordre


\begin{exercice}[Relation d'équivalence la plus/moins fine vérifiant une propriété]

\end{exercice}

\begin{exercice}[Coégalisateur]\index{coégalisateur}\index{propriété universelle!du coégalisateur}
Soient $A$ et $B$ deux ensembles et $f$ et $g$ deux applications entre $A$ et $B$. On définit sur $B$ la relation binaire suivante : $\mathcal R$ est la relation d'équivalence la plus fine telle que $\forall a\in A, f(a)\mathcal g(a)$. Le \emph{coégalisateur de $f$ et $g$} est par définition l'ensemble quotient $C = B/\mathcal R$. On note $\pi : B \to C$ la surjection canonique sur le quotient. On a alors $\pi\circ f = \pi \circ g$, ce que l'on peut résumer par le fait que le diagramme suivant commute:
\[
\xymatrix{
 A\ar@/^/[r] ^{f} \ar@/_/[r]_{g} & B \ar[r]^{\pi} & C
}
\]
Montrer que $C$ et $\pi$ vérifient la propriété suivante (dite \emph{propriété universelle du coégalisateur}):
\begin{quote}
Pour tout ensemble $X$ et application $\phi : B \to X$ vérifiant $\phi\circ f = \phi\circ g$,  il existe une unique application $h : C\to X$ telle que $\phi = h \circ \pi $. Autrement dit, il existe une unique application $h : C\to X$ faisant commuter le diagramme 
\[
\xymatrix{
 A\ar@/^/[r] ^{f} \ar@/_/[r]_{g} & B \ar[r]^{\pi} \ar[d]^{\phi}& C\ar@{-->}[dl]^{\exists!h}\\
 & X & 
}
\]
\end{quote}
\end{exercice}



% (co) produits cofibrés

\begin{exercice}[coproduit fibré]
Soient $A$,  $B$ et $C$  des ensembles et $f :  C\to A$, $g : C\to B$ des applications.

Soit $X = A\coprod B$ le coproduit de $A$ et $B$ (aussi appelé somme disjointe, et défini dans l'exercice \ref{exo-coproduit}), et $i_A$ et $i_B$ les injections canoniques de $A$ et $B$ dans $X$.



Le \emph{coproduit fibré (ou somme amalgamée) de $A$ et $B$  sous $C$} est l'ensemble quotient $X/\mathcal R$. On le note $A\coprod_C B$ La surjection canonique de $X$ vers son quotient $A\coprod_C B$ est notée $\pi$.

Montrer que $A\coprod_C B$ vérifie la propriété universelle suivante:
\begin{quote}
Pour tout ensemble  $D$ muni d'applications $\phi : A\to D$ et $\psi : B\to D$, il existe une unique application $h : A\coprod_C B \to D$ telle que $\phi = h\circ i_A$ et $\psi = h\circ i_B$, autrement dit il existe une unique application $h$ faisant commuter le diagramme  suivant:
\[ 
\xymatrix{
&A \ar[dr]_{i_A} \ar[drrr]^{\phi}& & & \\
C\ar[ur]^f\ar[dr]_g& &  A\coprod B \ar@{-->}[rr]^{\exists !h}  & & D\\
& B \ar[ur]^{i_B} \ar[urrr]_{\psi}& & &
}\]
\end{quote}

Soit $\mathcal R$ la relation d'équivalence la plus fine sur $X$ telle que 
\[
\forall c\in C, i_A(f(c))\mathcal R i_B(g(c))
\]

\end{exercice}



\printindex
\end{document}