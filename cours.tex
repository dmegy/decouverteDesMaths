\documentclass[11pt,a4paper]{book}

\usepackage{geometry}% gestion des marges etc
\usepackage[utf8]{inputenc} % caractères utf8 dans le fichier source
\usepackage[T1]{fontenc} % encodage en sortie
\usepackage[francais]{babel} % paramètres de langue : guillemets etc
\usepackage{amssymb,mathtools,amsthm}
\usepackage{stmaryrd,mathrsfs} % polices et symboles supplémentaires
\usepackage{fancybox,graphicx}
\usepackage[francais]{minitoc} % sommaires en début de chapitre
\setcounter{minitocdepth}{2} % profondeur des sommaires (1 = sections)
\setcounter{tocdepth}{1} % profondeur de la table des matières

\usepackage{fontawesome}
\usepackage{xypic,multicol,comment,variations,enumitem,datetime}

\usepackage{hyperref}
\hypersetup{
    colorlinks=true,       % false: boxed links; true: colored links
    linkcolor=[rgb]{0,0.2,0.6},          % color of internal links
    citecolor=[rgb]{0,0.2,0.6},        % color of links to bibliography
    filecolor=[rgb]{0,0.2,0.6},      % color of file links
    urlcolor=[rgb]{0.7,0.2,0.2}           % color of external links
}

\usepackage{pgf,pgfmath,tikz}
\usetikzlibrary{arrows}
\usetikzlibrary[patterns]
\tikzset{every picture/.style={execute at begin picture={
   \shorthandoff{:;!?};}
}}

\usepackage{makeidx}% imakeidx ne met pas les liens?


% - - - - - - -
% Spécifique à ce document :

\usepackage{fourier} % police de caractères : Adobe Utopia + Fourier math
\everymath{\displaystyle} % plus lisible mais casse l'homogénéité de la mise en page

\newcommand{\retourTOC}{Retour à la table des matières principale}

\newcommand{\N}{\mathbb{N}}
\newcommand{\Z}{\mathbb{Z}}
\newcommand{\Q}{\mathbb{Q}}
\newcommand{\R}{\mathbb{R}}
\newcommand{\C}{\mathbb{C}}
\newcommand{\K}{\mathbb{K}}
\renewcommand{\P}{\mathbb{P}}
\newcommand{\U}{\mathbb{U}}
\DeclareMathOperator{\pgcd}{pgcd}
\DeclareMathOperator{\ppcm}{ppcm}
\DeclareMathOperator{\Id}{Id}
\DeclareMathOperator{\Card}{Card} % cardinal

% Environnements : 
\swapnumbers
\theoremstyle{definition}
\newtheorem{theoreme}{Th\'eor\`eme}[section]
\newtheorem{proposition}[theoreme]{Proposition}
\newtheorem{corollaire}[theoreme]{Corollaire}
\newtheorem{lemme}[theoreme]{Lemme}
\newenvironment{preuve}{{\bf Preuve. }}{$\Box$}
\newenvironment{red}{\begin{quote}\emph{Exemple de rédaction:}\\}{\end{quote}}

\newtheorem{propdef}[theoreme]{Proposition et Définition}
\newtheorem{axiomedef}[theoreme]{Axiome et Définition}
\newtheorem{definition}[theoreme]{D\'efinition}
\newtheorem{vocabulaire}[theoreme]{Vocabulaire}
\newtheorem{exercice}[theoreme]{Exercice}
\newtheorem{exemple}[theoreme]{Exemple}
\newtheorem{exemples}[theoreme]{Exemples}
\theoremstyle{plain}
\newtheorem{remarque}[theoreme]{Remarque}
\newtheorem{methode}[theoreme]{Méthode}



\makeindex
% et faire un "makeindex"
\dominitoc
\begin{document}

\title{Découverte des mathématiques\\
Résumé de cours\\
L1, année 2017-2018\\
{\small 
\'Etat d'avancement : quasiment fini, en cours de relecture.}}
\date{\faGithub{}  Ce document et sa source \LaTeX{} sont disponibles  l'adresse
\url{http://github.com/dmegy/decouverteDesMaths.git}\\Version du \today{} à \currenttime}
\maketitle


\addtocontents{toc}{\protect\hypertarget{toc}{}}
\tableofcontents







\section{Préambule : vocabulaire et ensembles classiques}
Afin de pouvoir illustrer les notions de ce chapitre dans le contexte des mathématiques, on part du principe qu'un certain nombre de choses sont connues:

\begin{enumerate}
\item Les ensembles classiques :  $\N$, $\Z$, $\Q$, $\R$, $\C$, les mêmes privés de zéro : $\N^*$, ..., $\C^*$. Les lois de composition classiques sur ces ensembles : addition, multiplication, avec leurs règles de calcul.
\item L'égalité dans ces ensembles, la relation d'ordre dans $\R$ : $x< y$ se lit \og$x$ est strictement inférieur à $y$\fg, $x\leq y$ se lit \og $x$ est inférieur à $y$\fg{} (on précise parfois \og inférieur ou égal\fg{} même si sans précision, une inégalité est toujours prise au sens large).
\item La relation de divisibilité dans $\Z$ : la suite de symboles $a|b$ se lit \og $a$ divise $b$\fg.
\item Les notations d'appartenance d'un élément à un ensemble:on écrit $x \in E$ pour dire que $x$ est un élément de l'ensemble $E$ et $x\not\in E$ sinon. Par exemple, $\frac23 \in \Q$, mais $\sqrt{2} \not\in\Q$ (cela sera prouvé dans la suite du chapitre). 
\item Les notations $\R_+$, $\Q_-$, etc pour des contraintes de signe (au sens large : $0$ appartient à $\Q_+$ par exemple). On peut combiner : l'ensemble $\R_+^*$ est l'ensemble des réels strictement positifs. 
\item Les fonctions classiques, comme la racine carrée et la valeur absolue.
\end{enumerate}

Tout ceci sera revu en détail de toutes façons.


\section{Propositions / assertions logiques}

\begin{definition}
Une proposition est une phrase à laquelle on peut attribuer le statut \og VRAI\fg{} ou \og FAUX \fg{}. La phrase peut en outre comporter des symboles qui désignent des objets mathématiques (comme des chiffres) et d'autres symboles qui désignent des relations mathématiques entre objets (par exemple l'égalité, inégalité, divisibilité, appartenance à un ensemble...)
\end{definition}

Par exemple \og$2+2=3$\fg{} et \og $2+3=5$\fg{} sont des propositions (la première est fausse, la seconde vraie). La phrase \og le nombre complexe $i$ est positif\fg{} (ou encore \og quelle heure est-il?\fg) ne sont pas des propositions, on ne peut pas leur affecter de statut : la première n'a pas de sens (un nombre complexe n'a pas de signe), la seconde a un sens mais on ne peut pas lui affecter de statut VRAI ou FAUX.\\
 
 
\paragraph{Variables, paramètres, assertions ouvertes et fermées}


Une proposition peut dépendre d'un ou plusieurs paramètres, ou variables. Un paramètre est un symbole qui désigne un élément (explicité ou pas) d'un ensemble.

Les symboles nouveau doivent \textbf{toujours} être définis (ou déclarés) avec leur type (l'ensemble auquel appartient l'objet), de sorte à pouvoir être sûr du fait que la phrase est bien une assertion, c'est-à-dire possède un statut VRAI ou FAUX.

Par exemple : Dans \og$x\geq 0$\fg, le symbole $x$ n'est pas défini, on ne peut pas être sûr que la phrase ait un sens. Si $x$ était un nombre complexe par exemple, la phrase n'aurait aucun sens. Le symbole $x$ pourrait désigner beaucoup d'autres objets mathématiques, par exemple ... un cercle, les coordonnées d'un point du plan, auquel cas la phrase n'a pas non plus de sens.

 D'autre part si $x$ est un nombre naturel, la phrase a un sens mais elle est trivialement vraie car tous les naturels sont positifs. Tout ceci montre qu'il est crucial de déclarer clairement les variables et leur type \emph{avant} de commencer à les utiliser.

On déclare des objets à l'aide de la locution \og Soit\fg. La phrase \og Soit $x$ un réel.\fg{} déclare un réel, que l'on note $x$. La phrase \og Soit $k\in \Z$.\fg{} déclare un entier relatif (l'usage de $\in$ comme abréviation pour \og appartenant à\fg est toléré dans ce cas-là, même si en général on interdit d'utiliser les symboles mathématiques comme des abréviations).

La phrase \og Soit $x$.\fg{} n'est pas une déclaration correcte  d'objet mathématique : on doit préciser le type.

Si on précise que $x$ est un nombre réel, \og$x\geq 0$\fg devient une assertion mathématique bien formée. Le statut de cette proposition dépend de la valeur de $x$ : elle est vraie si $x\in \R_+$, elle est fausse si $x\in \R_-^*$. Le fait ne pas pouvoir connaître explicitement le statut n'est pas un problème. De fait que lorsqu'on déclare un réel $x$, on ne sait pas a priori lequel c'est.



\section{Construction de propositions}

Considérons deux propositions $A$ et $B$. Dans les exemples qui suivent, sauf précision, $x$ est un nombre réel.

\paragraph{Conjonction : \og A et B\fg} 

La proposition \og A et B \fg{} est vraie si A et B sont vraies. Elle est fausse dès que l'une au moins des deux est fausse.

Exemple : \og$x>2$ et $x<5$\fg{} est vraie si $x\in]2,5[$. Elle est fausse sinon.

\paragraph{Disjonction : \og A ou B\fg} 

La proposition  \og A ou B\fg{} est vraie dès que l'une des deux est vraie, elle est fausse si les deux sont fausses. Lorsqu'on affirme que \og A ou B\fg est vraie, l'un n'exclut pas l'autre.

Exemple : \og $x>2$ ou $x<5$\fg{ est vraie pour tout nombre réel $x$.

\paragraph{Négation : \og non A\fg} La proposition \og non A\fg{} est vraie si A est fausse et inversement.

\paragraph{Implication logique : \og$A \Rightarrow B$\fg}

La proposition \og $A \Rightarrow B$\fg signifie par définition \og B ou non-A\fg. Elle est vraie si $A$ est fausse ou si $B$ est vraie.\\
Exemples :  $2+2=4 \Rightarrow 2\times2 = 4$ est vraie. $2+2=5 \Rightarrow 2\times2 = 4$ est vraie. $2+2=5 \Rightarrow 2\times2 = 5$ est vraie. $2+2=4 \Rightarrow 2\times2 = 5$ est fausse. Autre exemple:  si $x$ est un nombre réel, la proposition $x>3 \Rightarrow x>4$  est vraie pour $x\leq 3$ ou pour $x>4$. Elle est fausse si $3<x\leq 4$.

Attention: le symbole $\Rightarrow$ n'est en aucun cas une abréviation pour \og donc\fg. La proposition $A \Rightarrow B$ ne veut pas dire \og  A est vraie donc B est vraie\fg !

\paragraph{\'Equivalence logique : \og $A \Leftrightarrow B$\fg}

La proposition \og$A \Leftrightarrow B$\fg{} signifie par définition \og $A \Rightarrow B$ et $B \Rightarrow A$\fg{}. Elle est  vraie si $A$ et $B$ ont même statut, que ce soit vrai ou faux. Elle est fausse si A et B ont des statuts différents.\\
Exemples : $2+2=5 \Leftrightarrow 2\times 3 = 7$ est vraie. $1>0 \Leftrightarrow 2+2=4$ est vraie. Si $x$ est un nombre réel, la proposition $x>3 \Leftrightarrow x<4$ est vraie pour $x\in]3,4[$. Elle est fausse sinon.

\section{Quantificateurs}
Soit $A(x)$ une proposition dépendant d'un paramètre $x$ appartenant à un ensemble $E$ (exemple :  \og $x>3$\fg, où $x \in \Z$).

\paragraph{Quantificateur universel : $\forall$ (quelque soit/pour tout)}$ $\\
La proposition \og$\forall x\in E,\:A(x)$\fg{} se lit \og pour tout $x$ dans $E$, $A(x)$\fg. Elle est vraie si $A(x)$ est vraie pour toutes les valeurs que peut prendre $x$ dans l'ensemble $E$. Elle est fausse dès qu'il existe une valeur spéciale de $x$ pour laquelle $A(x)$ est fausse.
Attention, contrairement à la proposition $A(x)$, la proposition $\forall x\in E,\:A(x)$ est une proposition qui ne dépend d'aucun paramètre : elle est soit vraie soit fausse : on dit que $x$ est une variable muette, ou interne.
Exemples :  $\forall x\in R,\: x^2>1$ est fausse. La proposition $\forall x\in \Z^*,\: x^2\geq 1$ est vraie.

\paragraph{Quantificateur existentiel : $\exists$ (il existe)}$ $\\
La proposition \og$\exists x\in E\slash A(x)$\fg{} se lit \og il existe $x$ dans $E$ tel que $A(x)$\fg. Elle est vraie s'il y a une valeur de $x$ dans l'ensemble $E$ telle que $A(x)$ soit vraie. Elle est fausse si $A(x)$ est fausse pour toutes les valeurs de $x$.



\begin{theoreme} On a les équivalences suivantes:\\
non (non A) $\Leftrightarrow$ A.\\
non ( A ou B ) $\Leftrightarrow$ (non A) et (non B).\\
non ( A et B ) $\Leftrightarrow$ (non A) ou (non B).\\
$(\forall x\in E,\; A(x))\Leftrightarrow (\forall y\in E,\; A(y))$.\\
$\text{non}(\forall x\in E,\: A(x)) \Leftrightarrow \exists x\in E,\: \text{non}(A(x)))$.\\
$\text{non}(\exists x\in E,\: A(x)) \Leftrightarrow \forall x\in E,\: \text{non}(A(x)))$.
\end{theoreme}

Démonstration : voir TD.

\section{Méthodes de démonstration}


\paragraph{Démonstration directe}$ $\\
\\
Exemple : soit $n \in Z$; montrer que \og $n$ pair $\Rightarrow n^2$ pair\fg.\\
\begin{red} Si $n$ est pair, il existe $k \in Z$ tel que $n = 2k$. Alors, $n^2 = 4k^2 = 2(2k^2)$ est pair. (et si $n$ est impair, l'implication est vraie par définition, il n'y a rien à prouver).\end{red}

\paragraph{Démonstration par contraposée}$ $\\
\\
Principe : $(A\Rightarrow B)$ est équivalente à $(\text{non-}B \Rightarrow \text{non-}A)$.\\
Preuve du principe: $(\text{non-}B \Rightarrow \text{non-}A)\Leftrightarrow (\text{non-}A \text{ ou } non-non-B)\Leftrightarrow (B\text{ ou }non-A)\Box$.\\
Exemple d'application : soit $n\in Z$; montrer que $n^2$ pair $\Rightarrow n$ pair.\\
\begin{red} On va montrer la contraposée, autrement dit on va montrer 
\og $n$ impair $\Rightarrow n^2$ impair\fg, qui est équivalente, mais plus facile à montrer. Supposons donc $n$ impair. Alors il existe $k \in\Z$ tel que $n = 2k+1$. Mais alors $n^2 = 4k^2+4k+1 = 2(2k^2+2k)+1$ est impair.\\
En combinant avec le résultat précédent, on a donc prouvé : \og$n^2$ pair $\Leftrightarrow n$ pair\fg\end{red}

\paragraph{Démonstration par l'absurde}$ $\\
\\
Principe : Si $F$ désigne n'importe quelle proposition fausse, on a $A \Leftrightarrow (\text{ non-}A \Rightarrow F)$.\\
Preuve du principe: $(\text{ non-}A \Rightarrow F) \Leftrightarrow (F\text{ ou non-non-}A)\Leftrightarrow A$.\\
 Donc pour montrer $A$, il suffit de supposer $A$ faux et d'en déduire une contradiction (c'est-à-dire n'importe quelle proposition fausse).\\
Exemple d'application : Montrer que $\sqrt{2}$ n'est pas rationnel.\\
\begin{red} Par l'absurde, supposons $\sqrt{2}\in\Q$. Alors il existe deux entiers $p$ et $q$ premiers entre eux tels que $\sqrt{2} = p/q$. Donc $p = q\sqrt{2}$ et donc $p^2 = 2 q^2$, donc $p^2$ est pair, donc par l'exemple précédent $p$ est pair. Donc il existe $k\in Z$ tel que $p = 2k$, d'où en remplacant $4k^2 = 2 q^2$, donc en simplifiant $q^2$ est pair donc $q$ est pair. Donc $p$ et $q$ sont tous les deux pairs, contradiction car ils sont premiers entre eux. Finalement cette contradiction prouve que $\sqrt{2} \not\in \Q$.\end{red}

\paragraph{Démonstrations de propositions avec quantificateur universel}$ $\\

Pour démontrer $\forall x\in E,\:A(x)$, on écrit:\\
\og Soit $x\in E$ un élément quelconque\fg.\\
Puis, on démontre $A(x)$.\\
Puis, pour conclure, on écrit : \og $x$ étant pris quelconque dans $E$, la propriété est bien démontrée\fg.\\


\begin{exemple}Montrer que $\forall x\in \R, x^2+x+1> 0$\fg.\\

Exemple de rédaction:\\
\begin{tabular}{lr}
Soit $x\in \R$. & (déclaration de $x$)\\
On a $x^2+x+1 = (x+\frac12)^2+\frac34$. & (Début preuve de $A(x)$)\\
Comme un carré est toujours positif, on a $(x+\frac12)^2 \geq 0$ & \\
 et donc  $x^2+x+1>0$. & (fin preuve de $A(x)$)\\
Ceci montre donc bien $\forall x\in \R, x^2+x+1>0$ & (Conclusion)\\
\end{tabular}
\end{exemple}


Exemple : Démontrer que $\forall x\in\R,\; x^2+\cos(x)>0$.\\
\begin{red} Soit $x\in\R$.\\
On distingue deux cas possibles suivant la valeur de $x$.\\
Si $0\leq |x|< \pi/2$, alors $x^2\geq 0$ et $\cos(x)> 0$ donc $x^2+\cos(x)>0$.\\
Si  $\pi/2\leq|x|$, alors $x^2+\cos(x)\geq \pi^2/4-1>0$.\\
Comme $x$ est quelconque, on a bien montré la propriété pour tout $x\in\R$.\end{red}

\paragraph{Cas particulier : démonstrations par récurrence}
Dans le cas particulier où le quantificateur universel porte sur l'ensemble $\N$, on peut utiliser une méthode de preuve spécifique, la récurrence. Cette méthode de démonstration s'appuie sur le fait que toute partie non vide de $\N$ admet un plus petit élément (ce qui est faux pour la plupart des autres ensembles classiques). Il suffit alors de montrer d'une part que $A(0)$ est vraie, ce qui est généralement facile, puis de montrer que pour tout $n \in\N$, on a $A(n) \Rightarrow A(n+1)$. La première étape est cruciale et le raisonnement est faux si on l'omet.


\paragraph{Démonstrations de propositions avec quantificateur existentiel}$ $\\

Pour démontrer \og$\exists x\in E\slash A(x)$, il faut soit construire un élément $x$ tel que $A(x)$ soit vrai, soit utiliser un théorème qui affirme l'existence d'un tel objet, ou qui affirme l'existence d'un objet à partir duquel on peut obtenir l'existence de $x$.\\
\\
Exemple 1 : soit $f$ une fonction croissante de $[0,1]$ dans $\R$. Montrer que $f$ est majorée, autrement dit montrer que $(\exists M\in\R\slash (\forall x\in[0,1],\;f(x)\leq M))$.\\
\begin{red}
Posons $M = f(1)$. On a bien $\forall x\in[0,1],\;f(x)\leq f(1)= M$, car $f$ est croissante.\end{red}\\
\\
Exemple 2 : Montrer qu'il existe deux irrationnels $a$ et $b$ tels que $a^b$ soit rationnel.\\
\begin{red} Considérons le nombre réel $\sqrt{2}^{\sqrt{2}}$. Il est soit rationnel, soit irrationnel. Dans le premier cas, il suffit de poser $a=b=\sqrt{2}$ (irrationnels, voir exemple plus haut) et la preuve est terminée. Dans le second cas, il suffit de poser $a=\sqrt{2}^{\sqrt{2}}$ (qui est supposé irrationnel) et $b=\sqrt{2}$ On a alors $a^b = \left(\sqrt{2}^{\sqrt{2}}\right)^{\sqrt{2}} = \sqrt{2}^2 = 2 \in \Q$.\end{red}\\
\\
Ce deuxième exemple montre que parfois, on n'a pas besoin de construire explicitement l'objet, seulement de montrer que ça existe, soit par l'analyse de cas de figure complémentaires, soit en utiisant un théorème qui affirme l'existence d'un certain objet sans forcément l'expliciter. Cela dit, la plupart du temps, il faut construire l'objet.


\section{Résolution des équations}

Soit $A(x)$ une proposition portant sur $x\in E$. Résoudre $A(x)$, c'est déterminer exactement l'ensemble des $x$ tels que $A(x)$ soit vrai. Cet ensemble est un sous-ensemble de $E$, on l'appelle l'ensemble des solutions. Il peut parfois être vide (aucune solution) ou égal à $E$ (équation triviale).


\paragraph{Méthode par équivalence}$ $\\
$A(x) \Leftrightarrow B(x) \Leftrightarrow ... \Leftrightarrow C(x)$ et on sait facilement résoudre $C(x)$. Cette méthode ne  s'applique que rarement, essentiellement qu'aux (systèmes d') équations linéaires.

\begin{exemple}Résoudre $2x+3=5$, d'inconnue $x\in \R$.\\
Exemple de rédaction:\\
Soit $x\in \R$. On a 
\begin{align*}
2x+3=8
&\iff 2x=5\\
&\iff x=5/2.
\end{align*}
\end{exemple}

\paragraph{Méthode par conditions nécessaires et suffisantes}$ $\\
Lorsque $A(x) \Rightarrow B(x)$, on dit que $B(x)$ est une condition nécessaire à $A(x)$, et $A(x)$ est une condition suffisante pour $B(x)$.\\
Dans la pratique, on écrit $A(x)\Rightarrow B(x) \Rightarrow ... x\in\Omega$. Ensuite, parmi les éléments de $\Omega$, on détermine ceux qui sont solution.\\
\\
Exemple : résoudre $|x-1|=2x+3$, d'inconnue $x\in \R$.\\
\begin{red} Soit $x\in\R$. On a la chaîne d'implications $|x-1|=2x+3 \Rightarrow |x-1|^2=(2x+3)^2 \Leftrightarrow x^2-2x+1=4x^2+12x+9 \Leftrightarrow 3x^2+14x+8=0 \Leftrightarrow (x\in \{-4;-2/3\})$. Réciproquement, on vérifie que $-4$ n'est pas solution mais que $-2/3$ est solution. Finalement, l'équation a une unique solution, $-2/3$.\end{red}



\chapter{Ensembles}
\minitoc
\hyperlink{toc}{\retourTOC}

\section{Définitions (ou pas)}

En mathématiques, le sens du mot \emph{ensemble} est plus précis que celui donné par la langue française. Définir rigoureusement ce qu'est un ensemble (au sens mathématique du terme) est assez complexe et dans ce cours, on utilisera la définition intuitive suivante.

\begin{definition}(Ensemble, définition intuitive)

\begin{enumerate}
\item Un \emph{ensemble} $E$ est une collection d'objets.
\item Les objets dont est constitué la collection définissant $E$  sont les \emph{éléments} de $E$.
\item On dit que $x$ appartient à $E$ et on note $x\in E$ si $x$ est un élément de $E$. On note $x\not\in E$ dans le cas contraire.
\item Deux ensembles $E$ et $F$ sont dits \emph{égaux} s'ils ont les mêmes éléments. Dans ce cas on note $E=F$ (et $E\neq F$ dans le cas contraire).
\end{enumerate}
\end{definition}

(La définition donnée est insuffisante car en réalité, toutes les collections ne sont pas autorisées (pour éviter certains paradoxes). Mais la plupart de celles auxquelles on peut penser forment bien des ensembles au sens mathématique du terme). 

\begin{definition}(Manières de définir un ensemble)

\index{définition!par énumération}\index{définition!par compréhension}
\begin{enumerate}
\item Une définition \emph{par énumération (ou extension)} d'un ensemble $E$ est la donnée explicite de tous les éléments de l'ensemble, sous forme de liste entre accolades. Par exemple : $E = \left\{1,3,\pi,5,\sqrt2\right\}$.
\item Une définition \emph{par compréhension} d'un ensemble $E$ est la donnée d'une propriété qui caractérise les éléments de $E$ parmi un ensemble plus gros $F$. Par exemple : $E = \{x\in \R\:\mid\: x^2+x\leq 2\}$, qui se lit \og$E$ est l'ensemble des réels $x$ tels que $x^2+x\leq 2$\fg.
\item Un \emph{paramétrage} d'un ensemble $E$ est la donnée de $E$ comme l'ensemble des éléments d'une certaine forme, dépendant d'un paramètre, lorsque le paramètre varie dans un autre ensemble. Autrement dit, c'est une écriture de la forme $E = \{f(a)\:\mid\: a\in A\}$, ce qui se lit \og $E$ est l'ensemble des $f(a)$, lorsque le paramètre $a$ décrit l'ensemble $A$\fg. Par exemple, l'ensemble $\{2k+1\:\mid\: k\in \Z\}$ est l'ensemble des entiers de la forme $2k+1$, lorsque le paramètre $k$ décrit $\Z$. (C'est l'ensemble des nombres impairs.)
\end{enumerate}
\end{definition}



Attention, dans une définition par énumération, il n'y a pas de notion d'ordre, ni de multiplicité (un élément ne peut pas appartenir \og plusieurs fois\fg{} à un ensemble). Donc  $\left\{1,3,\pi,5,\sqrt2\right\} = \left\{\sqrt 2, 3,5,1,\pi\right\}=\left\{\sqrt 2,2, 2, 3,  3,5,1,1,\pi\right\}$.

\begin{definition}(Ensemble vide, singleton, paire)
\index{ensemble vide}\index{singleton}
\begin{enumerate}
\item L'\emph{ensemble vide} est l'unique ensemble ne contenant aucun élément. On le note $\varnothing$ (la notation $\{\:\}$ est également correcte mais n'est pas utilisée). Une assertion du type \og $\forall x\in \varnothing, \: A(x)$\fg{} est toujours vraie par définition. L'ensemble vide est inclus dans tout ensemble, puisque l'assertion \og $\forall x\in \varnothing, \: x\in E$\fg{} est toujours vraie.
\item Un \emph{singleton} est un ensemble contenant un unique élément. C'est donc un ensemble de la forme $\{x\}$.
\item Une \emph{paire} est un ensemble de la forme $\{a,b\}$. Si $a=b$, alors il s'agit d'un singleton, mais la plupart des cas, les éléments sont différents et $\{a,b\}$ est donc un ensemble contenant deux éléments distincts.
\end{enumerate}
\end{definition}

\begin{exemple} (Paramétrage d'un segment) 
Soient $a$ et $b$ deux réels. L'intervalle fermé de $\R$ délimité par $a$ et $b$ peut être paramétré par:
\[ \{a+t(b-a)\:\mid\: t\in [0,1]\}\]
C'est paramétrage standard de cet intervalle. Lorsque $t=0$, le réel $a+t(b-a)$ vaut $a$, lorsque $t=1$, le réel $a+t(b-a)$ vaut $b$, et lorsque $t$ varie entre $0$ et $1$, le réel $a+t(b-a)v$, varie entre $a$ et $b$.

On voit souvent $(1-t)a+tb$ au lieu de $a+t(b-a)$ : les deux écritures doivent être reconnues instantanément. Noter qu'on pourrait paramétrer l'intervalle de nombreuses autres manières, par exemple en \og partant de $b$\fg{} c'est-à-dire en échangeant $a$ et $b$ dans le paramétrage.
\end{exemple}

\begin{exemple} (Nombres complexes de module un) \index{cercle!unité de $\C$}
 On appelle \emph{ensemble des nombres complexes de module un} (ou encore \emph{cercle unité de $\C$}) et on note $\U$ l'ensemble 
\[ \U = \ensemble{ z\in \C}{|z|=1}.\]
 On montrera dans la suite que cet ensemble peut s'écrire sous forme paramétrée de la façon suivante :
\[ \U = \ensemble{ e^{it}}{t \in \R}.\]
\end{exemple}

% mettre après les inclusions pour pouvoir justifier les égalités par double inclusion ?
\begin{exemple}
Il est crucial de savoir jongler entre définitions paramétrées et définitions en compréhension. Ainsi, on a par exemple les égalités suivantes :
% en anticipant sur les ensembles produits
\begin{align*}
\left\{2k+1 \:\mid\: k\in \N\right\}  &= \left\{ n\in \N\:\mid\: 2\not|\: n\right\} \\
\left\{2k \:\mid\: k\in \N\right\} &=  \left\{ n\in \N\:\mid\: 2| n\right\}\\
\R_+ = \{u^2\:\mid\: u\in \R\} & = \{x\in \R\:\mid\: x\geq 0\}
\end{align*}
Noter que plusieurs valeurs du paramètre peuvent correspondre à un même élément de l'ensemble. L'important est que le paramétrage \og couvre\fg{} l'ensemble souhaité, il peut y avoir de la redondance.
\end{exemple}



\begin{remarque}
En anticipant sur le chapitre suivant, paramétrer un ensemble revient à l'écrire comme \emph{image d'une application} (voir la section\ref{subsec-retour-parametrage-familles}).
\end{remarque}




\section{Parties d'un ensemble}

\begin{definition}[Sous-ensemble / partie]
\index{ensemble!sous-ensemble}\index{ensemble!partie d'un ensemble}\index{partie!d'un ensemble}
Soient $E$ et $F$ des ensembles. On dit que $E$ est inclus dans $F$, ou que $E$ est un sous-ensemble, ou une partie de $F$ si tous les éléments de $E$ sont des éléments de $F$, autrement dit si 
\[
\forall x\in E, \: x\in F.
\]
Dans ce cas on note $E\subset F$ (notation la plus répandue) ou $E\subseteq F$ (dans ce cours, les deux notations sont synonymes et on privilégie la seconde). On note $E\not\subset F$ ou $E\nsubseteq F$ si $E$ n'est pas un sous-ensemble de $F$, et $E\subsetneq F$  si c'est un sous-ensemble \emph{strict} de $F$, c'est-à-dire $E\subseteq F$ et $E\neq F$.
\end{definition}

\begin{remarque}[Principe de double-inclusion] Si $E$ et $F$ sont des ensembles, alors 
$
E=F \iff \left(E\subseteq F \text{ et } F \subseteq E\right).
$
\end{remarque}


\begin{remarque}
Attention, les objets $x$ et $\{x\}$ sont différents !  Par exemple, $\varnothing$ et $\{\varnothing\}$ sont deux ensembles différents:  le premier est l'ensemble vide, alors que $\{\varnothing\}$ est un ensemble non vide : c'est un ensemble contenant un élément (l'ensemble vide).
\end{remarque}

\begin{axiomedef}[Ensemble des parties]
Soit $E$ un ensemble. La collection de toutes les parties de $E$ est un ensemble (au sens mathématique). On le note $\mathcal P(E)$. 

Ainsi, si $F$ est un ensemble, alors on a $F\in \mathcal P(E) \iff F\subseteq E$.
\end{axiomedef}

Remarque : cet ensemble n'est jamais vide car il contient toujours au moins $\varnothing$, qui est une partie de  tout ensemble $E$.

\begin{exemple}
Un singleton $\{a\}$ contient deux parties : la partie vide $\varnothing$ et la partie $\{a\}$. 

L'ensemble $\{1,2\}$ a pour ensemble de parties :  $\mathcal P(\{1,2\}) = \{\varnothing,\{1\},\{2\},\{1,2\}\}$.
\end{exemple}


\section{Union, intersection, complémentaire, produit}

\begin{definition}(Union, intersection, complémentaire)

Soient $A$ et $B$ des ensembles. Leur \emph{union}, notée $A\cup B$, est la collection formée par les éléments de $A$ et de $B$. Leur \emph{intersection}, notée $A\cap B$, est l'ensemble des éléments de $A$ qui sont également des éléments de $B$ (ou encore : l'ensemble des éléments de $B$ qui sont aussi des éléments de $A$). 

L'ensemble $A\setminus B$ est l'ensemble des éléments de $A$ qui n'appartiennent pas à $B$.

Si $A$ est un sous-ensemble d'un ensemble $E$, le complémentaire de $A$ dans $E$ est $E\setminus A$, on le note aussi $\complement A$ s'il n'y a pas d'ambiguïté sur l'ensemble $E$ dans lequel on prend le complémentaire de $A$.
\end{definition}


\begin{proposition}
Si $A$ et $B$ sont deux parties d'un ensemble $E$, on a :
\[
\complement(A\cup B) = \complement A \cap \complement B; \quad
\complement(A\cap B) = \complement A \cup \complement B.
\]
\end{proposition}
\begin{proof}
Soit $x\in E$. Alors:
\begin{align*}
x\in \complement(A\cup B) 
&\iff \text{non}(x \in A\cup B)\\
&\iff \text{non}(x \in A \text{ ou } x\in B)\\
&\iff \text{non}(x \in A) \text{ et } \text{non}(x\in B)\\
&\iff \left(x\in\complement A\right) \text{ et } \left(x\in\complement B\right)\\
&\iff x\in \complement A \cap \complement B.
\end{align*}
\end{proof}

\begin{definition}[Ensembles disjoints, unions disjointes]
Deux ensembles $A$ et $B$ sont \emph{disjoints} si leur intersection est vide : $A\cap B = \varnothing$. 
\end{definition}

\begin{attention}
Ne pas confondre des ensembles \emph{disjoints} et des ensembles \emph{distincts}. Les ensembles $\{1,2\}$ et $\{1,3\}$ sont distincts mais non disjoints.
\end{attention}


\begin{definition}[Produit cartésien]
\index{produit cartésien d'ensembles}
Soient $E$ et $F$ deux ensembles. Le \emph{produit cartésien}, ou simplement \emph{produit}, noté $E\times F$, est la collection de tous les couples  de la forme $(x,y)$, avec $x\in E$ et $y\in F$. Si $E=F$, on note $E^2$ au lieu de $E\times E$. 

On peut définir de même les produits finis du type $E_1\times E_2 \times ... E_n$ : leurs éléments sont les $n$-uplets de la forme $(x_1, x_2, ... x_n)$, avec $x_1\in E_1$, $x_2\in E_2$ etc. 
\end{definition}

\begin{remarque}
Le lecteur est sans doute déjà familier avec l'ensemble $\R^2$ (l'ensemble des couples de réels), qui est par exemple utilisé en géométrie plane lorsque l'on travaille en coordonnées (relatives à un repère fixé).
\end{remarque}

\begin{remarque}
$E\times F = \varnothing \iff \left( E=\varnothing\text{ ou }F=\varnothing\right)$.
\end{remarque}

\begin{definition}[Diagonale]
\index{diagonale!d'un produit de deux ensembles}
Soit $E$ un ensemble. La \emph{diagonale} de $E\times E$ est l'ensemble des couples d'éléments identiques, c'est-à-dire l'ensemble
\[
\Delta_E = \{(x,x)\:|\: x\in E\}
\]
\end{definition}



Pour finir, on donne quelques exemples de sous-ensembles de $\R^2$ et $\R^3$, sous forme paramétrée ou définis par des équations.

\begin{exemple}
\begin{enumerate}
\item L'ensemble $\mathcal D = \{(x,y)\in \R^2\:\mid\: x+6y-20=0\}$ est la droite de $\R^2$ d'équation cartésienne $x+6y-20=0$.

On peut aussi l'écrire sous forme paramétrée $\mathcal D = \{ (2,3)+t(6,-1)\:\mid\: t\in \R\}$ (droite passant par le point $(2,3)$ et dirigée par le vecteur $(6,-1)$).
\item \index{cercle!unité de $\R^2$} On appelle \emph{cercle unité de $\R^2$} et on note $\S^1$ le cercle centré sur l'origine et de rayon un, c'est-à-dire :
\[ \S^1=\ensemble{(x,y)\in \R^2}{x^2+y^2=1}.\]
Une de ses écritures paramétrées les plus courantes est  $\S^1=\ensemble{(\cos t,\sin t)}{t\in \R}$, mais on peut également écrire $\S^1=\ensemble{(\cos t,\sin t)}{t\in [-\pi,\pi]}$ ou même $\S^1=\ensemble{(\cos t,\sin t)}{t\in ]-\pi,\pi]}$.
\item (Un cylindre) \index{cylindre} L'ensemble $\ensemble{(x,y,z)\in \R^3}{x^2+y^2=1}$ peut être visualisé comme un  cylindre (\og creux, vertical et de longueur infinie\fg). On peut aussi l'écrire sous forme de produit $\S^1 \times \R$ (cercle par droite), qui est bien un sous-ensemble de $\R^2\times \R = \R^3$. Enfin, on peut aussi l'écrire sous la forme paramétrée $\ensemble{(\cos t,\sin t,z)}{(t,z)\in \R^2}$.
\item \index{sphère!unité de $\R^3$} On appelle \emph{sphère unité de $\R^3$} et on note\footnote{Le \og $2$\fg{} dans la notation $\S^2$ désigne la \og dimension\fg{}: la sphère est une surface, donc de \og dimension\fg{} deux, même si elle est plongée dans un espace ambiant de dimension trois.} $\S^2$ l'ensemble
\[ \S^2 = \ensemble{(x,y,z)\in \R^3}{x^2+y^2+z^2=1}\]
C'est l'ensemble des points de l'espace à distance $1$ de l'origine. Cet ensemble est un peu plus difficile à paramétrer. On utilise en général le paramétrage donné par les \emph{coordonnées sphériques}\index{coordonnées!sphériques}:
\[ \S^2 = \ensemble{(\cos\theta\cos\phi, \sin\theta\cos\phi, \sin\phi)}{\theta \in [-\pi,\pi], \: \phi\in[-\frac{\pi}{2},\frac{\pi}{2}]}\]
On démontrera dans l'exercice \ref{exo-coord-spheriques-sphere} que ceci est bien un paramétrage de la sphère. (Les notations sont ici celles des mathématiciens et géographes, et non des physiciens : $\theta$ est la longitude, et $\phi$ est la latitude usuelle. Les physiciens utilisent la colatitude et permutent les symboles $\theta$ et $\phi$.) On peut également paramétrer l'espace $\R^3$ par des \og coordonnées\fg{} sphériques, voir l'exercice \ref{exo-coord-spheriques-espace} qui explique également les guillemets.
\end{enumerate}
\end{exemple}



\section{Familles indexées}

\begin{definition}
Soient $E$ et $I$ des ensembles. Une famille d'éléments de $E$ indexée ou paramétrée par $I$ est un objet de la forme $(x_i)_{i\in I}$, c'est-à-dire la donnée, pour tout élément $i\in I$, d'un élément de $E$ noté $x_i$.
\end{definition}

\begin{exemple}
Une suite réelle $(u_n)_{n\in \N}$ est une famille de réels indexée par $\N$.
\end{exemple}

\begin{attention}
Ne pas confondre \emph{famille paramétrée} et \emph{ensemble paramétré}. Par exemple la famille $\left( (-1)^n \right)_{n\in \N}$ est bien différente de l'ensemble $\left\{(-1)^n\:\mid\:n\in \N\right\}$ : en effet, la famille est infinie, c'est la suite $(1,-1, 1,-1, \dots)$, alors que l'ensemble, lui est fini, il ne contient que deux éléments : $1$ et $-1$.
\end{attention}

L'ensemble $I$ qui sert à indexer la famille peut être fini ou infini, et s'il est infini, il peut être plus gros que $\N$ : il n'est pas nécessaire de pouvoir numéroter les éléments de la famille par des nombres : une famille peut être indexée par $\R$. Par exemple, si $a\in \R$, on peut définir la fonction $f_a : \R\to \R; x\mapsto e^{ax}$. Les fonctions $f_a$ forment la famille $(f_a)_{a\in \R}$.


\begin{definition}[Unions et intersections indexées par un ensemble]
Soit $E$ un ensemble, et $(E_i)_{i\in I}$ une famille de parties de $E$ indexée par un ensemble $I$.

Leur \emph{union}, notée $\bigcup_{i\in I} E_i$, est l'ensemble $\{x\in E\:\mid\: \exists i\in I, x\in E_i\}$.

Leur \emph{intersection}, notée $\bigcap_{i\in I} E_i$, est l'ensemble $\{x\in E\:\mid\: \forall i\in I, x\in E_i\}$.
\end{definition}


%\begin{definition}[Cylindre sur un ensemble]\index{cercle}\index{cylindre}
%Soit $E$ un ensemble. Le \emph{cylindre sur $E$} est l'ensemble 
%\[ \operatorname{Cyl}(E) = E\times[0,1].\]
%Dans le cas particulier où l'ensemble $E$ est un cercle, son cylindre peut effectivement être vu comme un cylindre (de hauteur un). Le cas général est une abstraction de cet exemple. Dans le cas général, le cylindre peut être vu comme \og l'épaississement de $E$ dans une direction\fg.
%\end{definition}

\section{Exercices d'approfondissement}

\begin{exercice}[Topologies]
Soit $E$ un ensemble et  $\mathcal O$ une partie de $\mathcal P(E)$. On dit que $\mathcal O$ est une \emph{topologie sur $E$} si les conditions suivantes sont vérifiées
\begin{itemize}
\item $\mathcal O$ est stable par intersection finie, autrement dit : pour tout $n\in \N^*$ et toute famille $U_1, \cdots U_n$ d'éléments de $\mathcal O$, on a $\bigcap_{i=1}^n U_i\in \mathcal O$.
\item $\mathcal O$ est stable par union quelconque, autrement dit : pour tout ensemble $I$ et toute famille $(U_i)_{i\in I}$ d'éléments de $\mathcal O$, $\bigcup_{i\in I}U_i \in \mathcal O$.
\item Les parties $\emptyset$ et $E$ sont des éléments de $\mathcal O$.
\end{itemize}

\begin{enumerate}
\item Montrer que $\mathcal O_1=\{\emptyset, E\}$ et $\mathcal O_2=\mathcal P(E)$ sont des topologies sur $E$.
\item Montrer que 
\[ \mathcal O_3 = \ensemble{U\in \mathcal P(E)}{U=\emptyset \text{ ou }{}^cU\text{ est fini}}
\]
est une topologie sur $E$.
\item Combien de topologies différentes y a-t-il si $E$ est l'ensemble vide ? S'il n'a qu'un seul élément ? Deux éléments ? Trois éléments ?
% les ensembles finis n'ont pas encore été vus en détail mais c'est faisable
\end{enumerate}
\end{exercice}

\begin{exercice}[Bornologies]
Dans l'ensemble $\R$, il existe une notion de \emph{partie bornée} : c'est une partie qui est incluse dans un segment du type $[-M,M]$, pour un certain $M$. Cet exercice montre comment généraliser cette notion de \emph{partie bornée} à un ensemble quelconque.

Soit $E$ un ensemble et  $\mathcal B$ une partie de $\mathcal P(E)$. On dit que $\mathcal B$ est une \emph{bornologie sur $E$} si les conditions suivantes sont vérifiées
\begin{itemize}
\item Si $A\in \mathcal B$ et $B\subseteq A$, alors $B\in \mathcal B$.
\item Si $A\in \mathcal B$ et $B \in \mathcal B$, alors $A\cup B\in \mathcal B$.
\item Pour tout $x\in E$, on a  $\{x\} \in \mathcal B$.
\end{itemize}
Les éléments de $\mathcal B$ sont dits \emph{$\mathcal B$-bornés}, ou simplement \emph{bornés} s'il n'y a pas d'ambiguïté sur la bornologie utilisée.

Dans la suite, on fixe un ensemble $E$.
\begin{enumerate}
\item Montrer que $\mathcal B_1=\{\emptyset, E\}$ est une bornologie de $E$. On l'appelle la \emph{bornologie triviale (ou : grossière)}.
\item Montrer que l'ensemble $\mathcal B_2$ des parties finies de $E$ est une bornologie de $E$. On l'appelle la \emph{bornologie discrète}.
\item Combien de bornologies différentes y a-t-il si $E$ est vide ? S'il contient (exactement) un élément ? Deux ? Trois ?
\item On suppose maintenant que $E=\R$. Soit $\mathcal B_3$ l'ensemble des parties $A\subseteq \R$ bornées au sens classique, autrement dit 
\[ A\in \mathcal B_3 \iff \exists M\in \R, \forall a\in A, |a|\leq M\]
Montrer que $\mathcal B_3$ est une bornologie. On l'appelle la \emph{bornologie usuelle sur $\R$}, et lorsqu'on parle de bornés de $\R$, il est implicite qu'on se réfère à cette bornologie (et non aux deux premières par exemple).
\item 
\end{enumerate}
\end{exercice}

\begin{exercice}[Algèbre de parties]
Soit $E$ un ensemble et  $\mathcal A$ une partie de $\mathcal P(E)$. On dit que $\mathcal A$ est une \emph{algèbre de parties $E$} si les conditions suivantes sont vérifiées:
\begin{itemize}
\item $\mathcal A$ n'est pas vide.
\item Si $X\in \mathcal A$, alors $E\setminus X$ aussi.
\item $\mathcal A$ est stable par union finie, autrement dit : pour tout $n\in \N^*$ et toute famille $U_1, \cdots U_n$ d'éléments de $\mathcal A$, on a $\bigcup_{i=1}^n U_i\in \mathcal A$.
\end{itemize}
\begin{enumerate}
\item Montrer que $\mathcal P(E)$ est une algèbre de parties de $E$.
\item Montrer  qu'une algèbre de parties de $E$ est stable par intersection finie.
\item Si $E$ a un, deux, ou trois éléments, combien d'algèbres de parties y a-t-il ?
\end{enumerate}
\end{exercice}


\chapter{Applications}


%Mettre plutôt trop de vocabulaire que pas assez : sections, rétractions, fibres etc.

\section{Applications, graphes}

\begin{definition}[Graphe d'application]
Soient $E$ et $F$ deux ensembles, et $\Gamma \subseteq E\times F$. On dit que $\Gamma$ est un \emph{graphe d'application de $E$ dans $F$} si la condition suivante est vérifiée:
\[\forall x\in E, \: \exists! y\in F, \: (x,y) \in \Gamma.\]
\end{definition}

\begin{exemple}[Application directe de la définition]
\begin{enumerate}
\item Si $E = \{1,2,3\}$ et $F = \{1,4\}$, alors l'ensemble $\Gamma = \{(1,4),(2,1),(3,1)\} \subseteq E\times F$ est un graphe d'application.

L'ensemble $\Gamma' = \{(1,1),(2,4)\} \subseteq E\times F$ n'est pas un graphe d'application.

L'ensemble $\Gamma'' = \{(1,1),(2,1),(2,4),(3,4)\} \subseteq E\times F$ non plus.
\item Si $E = F = \R$, l'ensemble $\Gamma = \{x,y)\in \R^2 \:\mid\: y=x^2\}$ est un graphe d'application, mais pas l'ensemble $\Gamma' = \{x,y)\in \R^2 :\mid\: x=y^2\}$. Par contre, si $E=F=(\R_+)^2$, l'ensemble  $\Gamma'' = \{x,y)\in (\R_+)^2 :\mid\: x=y^2\}$ est un graphe d'application de $E$ dans $F$.
\item Pour un sous-ensemble $\Gamma\subseteq E\times F$, être un graphe d'application de $E$ dans $F$ ne dépend pas que de l'ensemble $\Gamma$ lui-même mais aussi de $E$ et de $F$. Par exemple, si $E=\R_+$ et $F=\R_+$, alors $\Gamma = \{(x,y)\in \R_+\times \R_+ \:\mid\: x=y^2\}$ est un graphe d'application de $E$ dans $F$. Par contre, si $E=\R$ et $F=\R_+$, l'ensemble $\{(x,y)\in \R\times \R_+ \:\mid\: x=y^2\}$ (c'est le même que le précédent : les éléments sont les mêmes) n'est \emph{pas} un graphe d'application de $E$ dans $F$.
\end{enumerate}
\end{exemple}

\begin{definition}[Applications/fonction entre ensembles]
\index{application}\index{fonction}\index{domaine}\index{codomaine}\index{graphe}\index{image!d'un élément}
Une \emph{application} ou \emph{fonction} (dans ce cours, les deux mots sont synonymes) $f$ est la donnée de trois objets:
\begin{enumerate}
\item un ensemble $E$, appelé le \emph{domaine} de $f$;
\item un ensemble $F$, appelé le \emph{codomaine} de $f$;
\item une partie $\Gamma_f \subseteq E\times F$, appelée le \emph{graphe de $f$} qui est un \emph{graphe d'application} au sens de la définition précédente. 
\end{enumerate}
Ceci revient à donner $E$, $F$, et pour tout élément $x \in E$, un élément (unique) $y\in F$, appelé l'\emph{image de $x$ par $f$}. Cet élément est noté $f(x)$.
\end{definition}

Deux fonctions sont égales si elles ont même domaine et codomaine, et si les images des éléments sont les mêmes. (Et il n'est pas suffisant de demander que les images soient les mêmes.)

\begin{definition}[Image d'une application]\index{image!d'une application}
Soit $f : E\to F$. L'\emph{image} de $f$ est l'ensemble de toutes les images des éléments de $E$ par $f$:
\[
\{f(x)\:\mid\: x\in E\}
\]
\end{definition}

\begin{remarque}
\begin{enumerate}
\item Si $f$ est une application de $\R$ dans $\R$, ce que l'on appelle souvent une \og représentation graphique de $f$\fg{} est en fait une représentation graphique de son graphe. La représentation graphique n'est pas unique (l'échelle peut varier, on ne représente en général pas le domaine ni le codomaine en entier mais seulement une partie, etc) mais le graphe, lui, est un objet mathématique abstrait et unique.
\item Une fonction ne peut pas être uniquement définie par son graphe : la donnée du domaine et du codomaine sont nécessaires.
\item \index{application vide}\index{zérologie} (Zérologie : application vide) Soit $E = \varnothing$ et $F$ un ensemble. Il existe une (unique) application de $E$ dans $F$, appelée \emph{application vide}, celle dont le graphe $\Gamma$ est la partie vide de $E\times F = \varnothing$. (Si $E=\varnothing$,  l'assertion \og $\forall x\in E, \exists! y\in F,\: (x,y)\in \Gamma$\fg{} est effectivement vraie même si $\Gamma$ est vide et donc $\Gamma$ est bien un graphe d'application.)
\end{enumerate}
\end{remarque}

Pour définir une fonction de $E$ dans $F$, on écrit \og Soit $f : E\to F$ une fonction\fg. Pour définir une fonction particuière, plutôt que donner son graphe comme le demanderait la définition, on utilise le symbole \og$\mapsto$\fg{} qui se lit \og est envoyé sur / s'envoie sur / est associé à \fg{} comme dans l'exemple suivant:
\[
\text{ Soit } f :\Z \to \R,\: n\mapsto \sqrt{n^2+n+1}.
\]
Ceci se lit par exemple \og Soit $f$ l'application de $\Z$ dans $\R$ qui à (un entier relatif) $n$ associe (le réel) $\sqrt{n^2+n+1}$\fg.


(Dans cet exemple, on devrait auparavant justifier que l'expression sous le radical désigne bien un réel positif, c'est bien le cas : exercice.)

On rencontre également la mise en forme du type suivant:
\[
\text{ Soit } f :\begin{cases}\Z \to \R,\\ n\mapsto \sqrt{n^2+n+1}.\end{cases}
\]
\begin{definition}
Soient $E$ et $F$ des ensembles. L'ensemble des fonctions de $E$ dans $F$ est noté $\mathcal F(E,F)$ ou bien $F^E$ (attention à l'ordre dans la seconde notation).
\end{definition}

\begin{remarque} Un graphe de fonction n'est pas forcément défini par une formule simple du type $y=\sin(x)$, ou $y=x^2+e^x$. Par exemple, on peut utiliser plusieurs formules suivant l'endroit du domaine où se trouve la variable :
\[ f: 
\R \to \R, 
x\mapsto \begin{cases}\sqrt{x}\text{ si }x\geq 0\\ x^2+x+e^x\text{ sinon.}\end{cases}\]
%Il existe des fonctions pouvant paraître encore plus inhabituelles, par exemple:
%\[ f: 
%\R \to \R, 
%x\mapsto \begin{cases}e^x\text{ si } x\not\in\Q}\\ \text{si $x \in \Q$, le dénominateur (positif) $q$ de la fraction irréductible $\frac{p}{q}$ représentant $x$}\end{cases}\]
%Une fonction n'a pas de raison d'être continue, dérivable etc.
\end{remarque}

\begin{definition}[Antécédents d'un élément]
Soit $f : E\to F$ une fonction, et $y\in F$. On dit qu'un élément $x\in E$ est un \emph{antécédent} de $y$ si $f(x)=y$. En reformulant, l'ensemble des antécédents de $y$ est donc l'ensemble des solutions de l'équation $f(x)=y$, d'inconnue $x\in E$.
\end{definition}

\begin{exemple} Un élément du codomaine peut ne pas avoir d'antécédents, ou en avoir plusieurs. Par exemple, si $f : \R\to \R, x\mapsto x^2$, alors l'élément $-1$ n'a aucun antécédent (il n'existe pas de $x\in \R$ tel que $f(x)=x^2=-1$). L'élément $0$ a exactement un antécédent ($0$), et l'élément $4$ a deux antécédents : $2$ et $-2$.
\end{exemple}

\begin{definition}[Fonction caractéristique]
Soit $E$ un ensemble et $A\in \mathcal P(E)$ une partie de $E$. La \emph{fonction caractéristique} de $A$ (sous-entendu, dans $E$) est la fonction 
\[
\operatorname{1}_A :\begin{cases}E \to \{0,1\},\\ x\mapsto \begin{cases}1&\text{ si } x\in A\\0&\text{ si } x\not\in A\end{cases}\end{cases}
\]
\end{definition}

%----------------------------------
\section{Composition}

\begin{definition}[Composition]
\index{composition de deux applications}
Soit $f : X\to Y$ et $g : Y\to Z$ deux fonctions. La composée de $g$ et de $f$ est la fonction $g\circ f$ (se lit \og $g$ rond $f$\fg) de $X$ dans $Z$ qui à $x\in X$ associe $g(f(x)) \in Z$. Autrement dit, par définition, $(g\circ f)(x) = g(f(x))$.

Une composition d'applications se visualise à l'aide du diagramme\index{diagramme} suivant (attention à l'ordre : appliquer la fonction $g\circ f$ consiste à appliquer $f$ \emph{suivie} de $g$):
\[
\xymatrix{
X \ar[r]_{f} \ar@/^1pc/[rr]^{g\circ f}& Y \ar[r]_{g}& Z
}
\]
\end{definition}

Plus généralement, pour pouvoir composer deux fonctions il est suffisant que le codomaine de la première fonction (dans l'ordre de la composition) soit inclus dans le domaine de la seconde. (Cette condition est bien sûr également nécessaire, autrement l'écriture $g(f(x))$ n'a pas de sens. Deux fonctions quelconques ne sont donc en général pas composables.)

\begin{remarque}[Factorisation d'une application comme composée d'autres applications]
\'Ecrire une application $f$ sous la forme d'une composition $f = g\circ h$, c'est la \emph{factoriser}. Plusieurs factorisations sont possibles. Par exemple, si $f : \R\to \R, x\mapsto x^2+1$, on peut écrire $f : g\circ h$, avec
\[
h : \R\to \R, x\mapsto x^2 
\text{ et }
g : \R\to \R, x\mapsto x+1.
\]
\end{remarque}

\begin{remarque} Attention, même si les domaines et codomaines permettent de composer deux fonctions dans les deux sens, les fonctions $g\circ f$ et $g\circ f$ obtenues sont en général disctinctes. Par exemple, avec $f : \R\to R, x\mapsto x^2
$ et $g : \R\to \R, x\mapsto x+1$,  on peut composer dans les deux sens mais on a :
\begin{align*}
g\circ f : \R\to \R, & x\mapsto g(f(x)) = g(x^2) = x^2+1,\\
f\circ g : \R\to \R, & x\mapsto f(g(x)) = f(x+1) = (x+1)^2 = x^2+2x+1.
\end{align*}
\end{remarque}


\begin{proposition}[\og La composition est associative\fg]
\index{associative (composition)}
Soient $f : X\to Y$, $g : Y\to Z$, $h  : Z\to T$ des fonctions. Alors $h\circ (g\circ f) = (h\circ g)\circ f$. Cette fonction est notée $h\circ g\circ f$.

Ce résultat se visualise à l'aide du diagramme\index{diagramme}:
\[
\xymatrix{
X \ar[r]_{f} \ar@/^1pc/[rr]^{g\circ f} \ar[r]_{f} \ar@/^3pc/[rrr]^{h\circ (g\circ f)} \ar@/_3pc/[rrr]_{(h\circ g) \circ f}& Y \ar[r]_{g} \ar@/_1.2pc/[rr]_{h\circ g} & Z \ar[r]^{h} & T
}
\]

\end{proposition}
\begin{proof}
Les domaines et codomaines sont les mêmes ($X$ et $T$), et si $x\in X$, on a 
\[ \left(h\circ (g\circ f)\right) (x) = h((g\circ f)(x)) = h(g(f(x)) \text{ et } \]
\[ \left( (h\circ g)\circ f \right) (x) = (h \circ g)(f(x)) = h(g(f(x))\quad \]
d'où l'égalité des deux fonctions.
\end{proof}

\begin{definition}[Fonction identité]
Soit $E$ un ensemble. La fonction identité sur $E$ est la fonction $\Id_E : E\to E, x\mapsto x$.
\end{definition}

\begin{remarque}
\begin{enumerate}
\item Ne pas confondre la fonction identité avec une fonction constante.
\item Si $\phi = E\to F$, alors $\phi = \phi\circ \Id_E = \Id_F\circ \phi$.
\end{enumerate}
\end{remarque}

\begin{definition}[diagramme]
\index{diagramme}
Lorsque l'on est en présence de plusieurs ensembles et de plusieurs applications entre ces ensembles, il est classique de visualiser la situation à l'aide d'un diagramme. Un diagramme (d'applications entre ensembles) est un graphe dont chaque sommet représente un ensemble et chaque arête (orientée) représente une application.

Par exemple, étant donnés des ensembles $A$, $B$, $C$ et $D$ et des applications $f : A\to B$, $g : A\to C$, $h : D\to A$, $\phi : D\to C$, $\psi : B\to A$, on peut représenter la situation à l'aide du diagramme suivant:
\[
\xymatrix{
A \ar@/^1pc/[r]^{f} \ar[d]_{g} & B \ar@/^/[l]_{\psi}\\
C & D \ar[ul]^{h} \ar[l]^{\phi}
}
\]
\end{definition}



%---------------------------------
\section{Applications réciproques, sections et rétractions}



\begin{definition}[Fonction réciproque]
\index{réciproque (fonction)}
Soient $f : E\to F$ et $g = F\to E$ deux fonctions. On dit qu'elles sont réciproques l'une de l'autre (ou que $g$ est une réciproque de $f$, ou que $f$ est une réciproque de $g$) si $g\circ f = \Id_E$ \underline{et} $f\circ g = \Id_F$. 
\[
\xymatrix{
 E \ar@(dl,ul)^{\Id_E} \ar@/^2pc/[rrr]^{f} \ar@(ur,dr)^{g\circ f} 
& & & 
\ar@/^2pc/[lll]^{g} \ar@(dl,ul)^{f\circ g} F \ar@(ur,dr)^{\Id_F}
}
\]
\end{definition}



Attention, une fonction $f$ n'a pas toujours de fonction réciproque.

\begin{proposition} Soit $f : E\to F$ une fonction. Si elle admet une (fonction) réciproque, alors celle-ci est unique. Elle est notée généralement $f^{-1}$.
\end{proposition}

\textbf{Attention}, on ne doit pas utiliser la notation $f^{-1}$ avant d'avoir démontré que la fonction admet effectivement une réciproque.

\begin{proof}
Soient en effet $g = F\to E$ et $h : F \to E$ deux réciproques de $f$. Alors
\[
g\circ  f \circ h = (g\circ f) \circ h = \Id_E \circ h = h, \text{ et}
\]
\[
g\circ  f \circ h = g\circ (f \circ h) = g \circ \Id_F = g
\]
d'où $g=h$.
\end{proof}

\begin{exemple}
Les fonctions $f = \R\to \R_+^*, x\mapsto e^x$ et $g : \R_+^*\to R, x\mapsto \ln(x)$ sont réciproques l'une de l'autre.
\end{exemple}

\begin{remarque}
Dans la définition de réciproque, les conditions $g\circ f = \Id_E$ et  $f\circ g = \Id_F$ sont toutes les deux nécessaires : il est en effet possible que l'une soit vérifiée et pas l'autre. Par exemple, les fonctions 
\[
f : \R_+\to \R, x\mapsto \sqrt x
\quad \text{ et }\quad
g : \R\to \R_+, x\mapsto x^2
\]
vérifient $g\circ f=\Id_{\R_+}$, mais $f\circ g \neq \Id_{\R}$ : en effet, on a $(f\circ g)(x)=\sqrt{x^2}=|x|\neq x$.

Dans ce type de cas, on n'est pas en présence de fonctions réciproques mais la situation porte tout de même un nom. C'est l'objet de la définition suivante.
\end{remarque}

\begin{definition}[Rétraction/inverse à gauche, section/inverse à droite]
Soit $f : E\to F$ une fonction.
\begin{enumerate}
\item Une \emph{rétraction} (ou \emph{inverse à gauche}) de $f$, est une fonction $r:F\to E$ telle que $r\circ f = \Id_E$.
\item Une \emph{section} (ou \emph{inverse à droite}) de $f$, est une fonction $s:F\to E$ telle que $f \circ s = \Id_F$.
\end{enumerate}
\end{definition}

\begin{exemple}
Soit $f = \R\to \R_+, x\mapsto x^2$. Alors les fonctions $s_1 : \R_+\to \R, x\mapsto \sqrt x$ et $s_2 : \R_+\to \R, x\mapsto -\sqrt x$ sont deux sections (inverses à droite) distinctes de $f$ (on a bien $f\circ s_1 = \Id_{\R^+}$ et $f\circ s_2 = \Id_{\R^+}$). 
Par ailleurs, $f$ est une rétraction de $s_1$ et de $s_2$. (Voir remarque ci-dessous.)
%Soit $f : \{1,2\} \to \{3,4,5\}$ définie par $f(1)=3$ et $f(2)=5$. La fonction $g : \{3,4,5\}\to\{1,2\}$ telle que $g(3)=1$, $g(4)=2$ et $g(5)=2$ est une rétraction de $f$.  Par ailleurs, la fonction $f$ est une section de $g$ (voir remarque plus bas).

\end{exemple}

\begin{remarque}
\begin{enumerate}
\item Une fonction $g$ est une rétraction de $f$ si et seulement si $f$ est une section de $g$ puisque les deux assertions signifient $g\circ f = \Id_E$ : les deux notions sont \og duales\fg.
\item Les sections et rétractions, lorsqu'elles existent, ne sont en général pas uniques (contrairement à la fonction réciproque qui est unique si elle existe).
\item Une fonction réciproque est à la fois une rétraction et une section (ou : à la fois un inverse à gauche et un inverse à droite).
\item De même que toutes les fonctions n'ont pas forcément de réciproque, toutes les fonctions n'admettent pas forcément une section ou une rétraction. Par exemple, $f : \R\to \R, x\mapsto x^2$ n'admet ni section ni rétraction. 
\end{enumerate}
\end{remarque}

% autre exemple avec inverse à gauche / pas à droite ou l'inverse



%---------------------------------
\section{Restriction, prolongement, corestriction}
% corestriction plus tard, lorsqu'on aura vu l'image

\begin{definition}[Restriction]
\index{restriction d'une application}
Soit $f : E\to F$ et $A \in \mathcal P(E)$ une partie de $E$. La \emph{restriction} de $f$ à $A$, notée $f|_A$, est l'application de $A$ dans $F$ suivante:
\[
f|_A : A\to F, \: x\mapsto f(x)
\]
Attention, les fonctions $f|_A$ et $f$ doivent être considérées comme distinctes car leurs domaines sont distincts ($A$ au lieu de $E$).
\end{definition}

\begin{definition}[Prolongement]
Soient $E$ et $F$ des ensembles, $A\in \mathcal P(E)$ une partie de $E$ et $f : A\to F$ une fonction. On dit qu'une application $g : E\to F$ est un \emph{prolongement} de $f$ si $g|_A = f$.
\end{definition}

Attention, il existe en général plusieurs prolongements possibles d'une même fonction et même si la fonction $f$ est donnée par une formule, un prolongement n'a aucune raison d'être défini par la même formule hors du domaine originel de $f$. Par exemple, si $f : \R_+^* \to \R, x\mapsto e^x$, alors les fonctions suivantes  sont des prolongements de $f$ (à divers domaines):
\[g : \R^* \to \R, x\mapsto \begin{cases}e^x\text{ si } x>0\\ \sin(x) \text{ si } x<0\end{cases},\]
\[h : \R_+ \to \R, x\mapsto \begin{cases}e^x\text{ si } x>0\\ 10 \text{ si }x=0\end{cases}.\]
(Un prolongement ne doit pas non plus être forcément continu ni dérivable, etc.)


\begin{definition}[Corestriction]\index{corestriction}
Soit $f : E\to F$, et $B$ une partie de $F$ contenant toutes les images des éléments de $E$ (autrement dit, $\forall x\in E, f(x)\in B$).
La \emph{corestriction} de $f$ à $B$ est l'application de domaine $E$, codomaine $B$ et de même graphe que $f$, autrement dit c'est l'application 
\[ g : E\to B, x\mapsto f(x).\]
\end{definition}

\begin{exemple}
Soit $f : \R\to \R, x\mapsto x^2$. On peut la corestrindre à $[-3,+\infty[$ car cette partie de $\R$ contient toutes les images de $f$. La corestriction de $f$ à $[-3,+\infty[$ est $g : \R\to [-3,+\infty[, x\mapsto x^2$.
\end{exemple}




%---------------------------------
\section{Fonctions injectives et surjectives}



\begin{definition}[Fonction injective]
\index{injection}
Soient $A$ et $B$ deux ensembles, et $f : A \to B$ une application. On dit que $f$ est \emph{injective} (ou que c'est une \emph{injection}) si
\[\forall (x,y) \in A^2,\quad f(x)=f(y) \Rightarrow  x=y,\]
autrement dit si (contraposée) 
\[\forall (x,y) \in A^2,\quad x\neq y \Rightarrow  f(x)\neq f(y),\]
autrement dit si deux éléments distincts ont toujours des images distinctes. On dit aussi que $f$ \og sépare les points\fg.
\end{definition}

\begin{definition}[Fonction surjective]
\index{surjection}
On dit que $f$ est \emph{surjective} (ou que c'est une \emph{surjection}) si
\[\forall b \in B,\quad \exists a\in A / f(a)=b,\]
autrement dit tout élément $b\in B$ a (au moins) un antécédent par $f$.
\end{definition}

\begin{definition}[Fonction bijective]
\index{bijection}
On dit que $f$ est bijective si elle est injective et surjective.
\end{definition}

\begin{exemple}
\begin{enumerate}
\item La fonction identité (d'un ensemble $E$ dans lui-même) est bijective.
\item La fonction $f : \R \to \R, x\mapsto x^2$ n'est ni injective, ni surjective. Elle n'est pas injective car bien que $1$ soit différent de $-1$, ils ont la même image. Elle n'est pas surjective car $-2$ n'a pas d'antécédent dans $\R$ : on ne peut pas trouver de réel $x$ tel que $x^2 = -2$.
\item La corestriction de $f$ à $\R_+$ est $g : \R \to \R_+, x\mapsto x^2$. Elle n'est pas injective pour les mêmes raisons que $f$, mais elle est surjective : le codomaine est cette fois $\R_+$, et tout nombre réel positif $y\geq 0$ a au moins un antécédent, par exemple $-\sqrt{y}$.
\item La restriction de $g$ à $\R_+$ est $h : \R_+ \to \R_+, x\mapsto x^2$. Elle est injective et surjective, donc bijective. Elle est injective car si $x$ et $y$ sont des réels positifs ayant même carré, ils sont forcément égaux (ils sont positifs donc il n'y a pas l'ambiguité de signe). Sa surjectivité ne découle pas de celle de $g$ car le domaine a été restreint : elle est surjective car tout nombre réel positif $y\geq 0$ a au moins un antécédent \emph{dans $\R_+$}, à savoir $\sqrt{y}$.
\item (Zérologie)\index{zérologie} L'unique fonction de $\varnothing$ dans $\varnothing$ est bijective. La fonction vide de $\varnothing$ dans n'importe quel ensemble est toujours injective.
\end{enumerate}
\end{exemple}

En général, la surjectivité est plus difficile à montrer que l'injectivité, car il faut résoudre une équation à paramètre : l'équation $f(x)=y$, de paramètre $y$, et d'inconnue $x$, et ce pour tous les paramètres $y$. La non surjectivité est en revanche souvent plus facile à montrer, il suffit de trouver un élément qui n'a pas d'antécédent, en général cela se voit (éventuellement après un petit calcul / majoration / développement d'expression).

\begin{remarque}
Si $f : A\to B$ est injective, alors on peut parfois \og identifier\fg{} $A$ à un sous-ensemble de $B$ grâce à $f$ : un élément $a \in A$ est identifié à $f(a) \in B$. \textbf{Attention}, cette façon d'identifier $A$ à une partie de $B$ dépend de $f$ et il existe en général plusieurs injections de $A$ dans $B$, donc le choix de l'injection n'est pas anodin, ni canonique en général.
\end{remarque}



\begin{proposition}
\label{bijective_admet_reciproque}
Soit $f : E\to F$.
\begin{enumerate}
\item Si elle est surjective, elle admet une section.
\item Si elle est injective, elle admet une rétraction.
\item Si elle est bijective, elle admet une réciproque.
\end{enumerate}
\end{proposition}
\begin{proof}\index{section}
\begin{enumerate}
\item  Pour tout $y\in F$, on choisit un antécédent de $y$ par $f$, que l'on note $x_y$. On définit alors une fonction $g : F\to E$ par $g(y)=x_y$. Par construction, on a $f\circ g = \Id_F$ donc $g$ est une section de $f$.
\item À tout $y\in F$ on associe soit son unique antécédent s'il en existe un, soit un élément de $E$ arbitraire dans le cas contraire. Ceci définit une fonction $g : F\to E$ et par construction on a $g\circ f=\Id_E$.
\item D'une part, comme $f$ est surjective, elle admet (d'après le premier point) une section $s$, qui vérifie donc $f\circ s = \Id_F$. Montrons  que $s\circ f = \Id_E$. Soit $x\in E$ et soit $a = (s\circ f)(x)$. Alors $f(a) = (f\circ s \circ f) (x) = ((f\circ s)\circ f)(x) = f(x)$ et comme $f$ est injective, $a=x$ c'est-à-dire $(s\circ f)(x) = x$. Donc $s\circ f=\Id_E$ et donc $s$ est la réciproque de $f$.
\item \emph{Preuve alternative du dernier point, en suivant le cheminement inverse.} D'une part, comme $f$ est injective, elle admet (d'après le second point) une rétraction $r$, qui vérifie donc $r\circ f=\Id_E$. Montrons  que $f\circ r = \Id_F$. Soit $y\in F$. Comme $f$ est surjective, considérons $x$ un antécédent de $y$. Alors, $(f\circ r)(y) = (f\circ r)(f(x)) =(f\circ (r \circ f)(x) = (f\circ \Id_E)(x) = f(x) = y$. D'où $f\circ r = \Id_F$ et donc $r$ est la réciproque de $f$.
\end{enumerate}
\end{proof}

\begin{proposition}[Stabilité à la composition de l'injectivité et de la surjectivité]
Soient $f : E\to F$ et $g : F\to G$ deux fonctions.
\begin{enumerate}
\item Si $f$ et $g$ sont injectives, alors $g\circ f$ l'est également.
\item Si $f$ et $g$ sont surjectives, alors $g\circ f$ l'est également.
\item Si $f$ et $g$ sont bijectives, alors $g\circ f$ l'est également.
\end{enumerate}
\end{proposition}
\begin{proof}
\begin{enumerate}
\item Soient $x, y \in E$ tels que $(g\circ f)(x) = (g\circ f)(y) $, c'est-à-dire tels que $g(f(x))=g(f(y))$. Comme $g$ est injective, on a $f(x)=f(y)$. Comme $f$ est injective, on a alors $x=y$, ce qu'il fallait démontrer.
\item Soit $z\in G$. Comme $g$ est surjective, $z$ possède un antécédent par $g$ c'est-à-dire qu'il existe $y\in F$ tel que $g(y)=z$. Ensuite, comme $f$ est surjective, $y$ possède un antécédent par $f$, c'est-à-dire qu'il existe $x\in E$ tel que $f(x)=y$. On a alors $g(f(x)) = g(y)=z$, donc $x$ est un antécédent de $z$ par $g\circ f$. Ceci montre que tout élément de $G$ possède un antécédent par $g\circ f$, donc que $g\circ f$ est surjective.
\item Il suffit d'appliquer les deux premiers points.
\end{enumerate}
\end{proof}

\begin{proposition}[réciproques partielles]
Soient $f : E\to F$ et $g : F\to G$ deux fonctions.
\begin{enumerate}
\item Si $g\circ f$ est injective, alors $f$ l'est également.
\item Si $g\circ f$ est surjective, alors $g$ l'est également.
\end{enumerate}
\end{proposition}
\begin{proof}
\begin{enumerate}
\item Soient $x, y\in E$ tels que $f(x)=f(y)$. En appliquant $g$, il vient $g(f(x))=g(f(y))$. Comme $g\circ f$ est injective, $x=y$.
\item Soit $z\in G$. Comme $g\circ f$ est surjective, il existe $x\in E$ tel que $g(f(x))=z$. Posons $y = f(x)$. On a $g(y)=z$ donc $y$ est un antécédent de $z$ par $g$.
\end{enumerate}
\end{proof}

\begin{corollaire}
Soit $f : E\to F$ une application.
\begin{enumerate}
\item Elle admet une rétraction (inverse à gauche) ssi elle est injective.
\item Elle admet une section (inverse à droite) ssi elle est surjective.
\item Elle admet une fonction réciproque ssi elle est bijective.
\end{enumerate}
\end{corollaire}
\begin{proof}
Le sens \og si\fg{} a déjà été prouvé plus haut (prop. \ref{bijective_admet_reciproque}). Montrons le sens \og seulement si\fg.
\begin{enumerate}
\item Soit $r$ une rétraction de $f$. La composée $r\circ f = \Id_E$ est injective donc par la proposition précédente, $f$ est injective.
\item Soit $s$ une section de $f$. La composée $f \circ s= \Id_F$ est surjective donc par la proposition précédente, $f$ est surjective.
\item Une réciproque étant à la fois une section et une rétraction, on applique les deux points précédents.
\end{enumerate}
\end{proof}

\begin{remarque}
\textbf{Attention}, on peut avoir $g\circ f$ injective et $g$ non injective, et on peut aussi avoir $g\circ f$ surjective et $f$ non surjective. Considérons par exemple:
\[
f : \N\to \N, n\mapsto 2n
\quad \text{ et }\quad
g : \N\to \N, n\mapsto \lfloor n/2\rfloor.
\]
Alors $g\circ f = \Id_\N$ donc est bijective, mais $f$ n'est pas surjective et $g$ n'est pas injective.

On peut également considérer les fonctions
\[
f : \R_+\to \R, x\mapsto \sqrt x
\quad \text{ et }\quad
g : \R\to \R_+, x\mapsto x^2
\]
qui vérifient également $g\circ f=\Id_{\R_+}$ sans que $f$ soit surjective ni $g$ injective.
\end{remarque}



%--------------------------------------
\section{Images directes et réciproques de parties}

\begin{definition}[Image directe d'une partie]
\index{image!directe d'une partie}
Soit $f : E\to F$ et $A\subseteq E$. On appelle image directe de $A$ l'ensemble des images des éléments de $A$ :
\[f(A) = \{f(x)\:\mid\: x\in A\} = \{y\in F \:\mid\: \exists x\in A, y=f(x)\}.\]
Autrement dit, pour un élément $y\in F$, on a $y\in f(A) \iff \left(\exists x\in A, y=f(x)\right)$.
\end{definition}
\begin{definition}[Image réciproque d'une partie]
\index{image!réciproque d'une partie}
Soit $f : E\to F$ et $B\subseteq F$. On appelle image réciproque de $B$ l'ensemble de tous les antécédents d'éléments de $B$ :
\[f^{<-1>}(B) = \{x\in E\:\mid\: f(x)\in B \}.\]
Autrement dit, pour un élément $x\in E$, on a $x\in f^{<-1>}(B) \iff f(x)\in B$.

\textbf{Attention}, on voit très souvent la notation $f^{-1}(B)$ au lieu de $f^{<-1>}(B)$ mais cela peut prêter à confusion, la fonction $f$ n'ayant pas forcément de réciproque. Dans ce chapitre, on utilise la notation $f^{<-1>}(B)$, puis on utilisera progressivement la première.
\end{definition}

\begin{proposition}
Soit $f : E\to F$.
\begin{enumerate}
\item $f(\varnothing)=\varnothing$.
\item $f(E)\subseteq F$ avec égalité si et seulement si $f$ est surjective. En général, $f(E)\neq F$.
\item Si $(A_i)_{i\in I}$ est une famille de parties de $E$, alors $f\left(\bigcup_{i\in I}A_i\right) = \bigcup_{i\in I} f(A_i)$.
\item Par contre, on a en général $f\left(\bigcap_{i\in I}A_i\right) \subseteq \bigcap_{i\in I} f\left(A_i\right)$ mais pas forcément égalité.
\end{enumerate}
\end{proposition}
\begin{proof}
\begin{enumerate}
\item Clair.
\item On a $f(E)=F \iff (\forall y\in F, y\in f(E))$ ce qui signifie par définition que tout élément $y\in F$ admet au moins un antécédent, et donc que $f$ est surjective.  
\item Soit $y\in F$. On a 
\begin{align*}
y\in f\left( \bigcup_{i\in I} A_i \right)
&\iff \exists x\in  \bigcup_{i\in I} A_i,   y=f(x) \\
&\iff \exists x\in E, \left( x\in \bigcup_{i\in I} A_i\text{ et }  y=f(x)\right)\\
&\iff \exists x\in E, \exists i\in I,  (x\in A_i\text{ et } y=f(x)) \\
&\boxed{\iff \exists i\in I, \exists x\in E}\:, (x\in A_i\text{ et } y=f(x))\\
&\iff \exists i\in I, \exists x\in A_i, y=f(x)\\
&\iff \exists i\in I, y\in f(A_i)\\
&\iff y\in \bigcup_{i\in I} f(A_i)
\end{align*}
L'interversion de quantificateur signalée est licite car ce sont deux quantificateurs existentiels.
\item On a:
\begin{align*}
y\in f\left( \bigcap_{i\in I} A_i \right)
&\iff \exists x\in  \bigcap_{i\in I} A_i,   y=f(x) \\
&\iff \exists x\in E, \left( x\in \bigcap_{i\in I} A_i\text{ et }  y=f(x)\right)\\
&\iff \exists x\in E, \forall i\in I,  (x\in A_i\text{ et } y=f(x)) \\
&\boxed{\implies \forall i\in I, \exists x\in E}\:, (x\in A_i\text{ et } y=f(x)) \quad(*)\\
&\implies \forall i\in I, \exists x\in A_i, y=f(x)\\
&\implies \forall i\in I, y\in f(A_i)\\
&\implies y\in \bigcap_{i\in I} f(A_i)
\end{align*}
L'interversion des quantificateurs  est licite \textbf{dans ce sens-là seulement : $\exists x \forall i ... \implies \forall i \exists x ...$}, et l'équivalence devient une implication. Ceci prouve l'inclusion. Pour montrer qu'il n'y a pas forcément égalité, il suffit de donner un contre-exemple, par exemple $\sin(\R_-^* \cap \R_+^*) = \sin(\varnothing) = \varnothing \subsetneq \sin(\R_+^*) \cap \sin(\R_-^*)=[-1,1]$.
\end{enumerate}
\end{proof}

\begin{proposition}
Soit $f : E\to F$.
\begin{enumerate}
\item $f^{<-1>}(\varnothing)=\varnothing$.
\item $f^{<-1>}(F)=E$.
\item Si $(B_i)_{i\in I}$ est une famille de parties de $F$, alors $f^{<-1>}\left(\bigcup_{i\in I}B_i\right) = \bigcup_{i\in I} f^{<-1>}(B_i)$, ainsi que $f^{<-1>}\left(\bigcap_{i\in I}B_i\right) = \bigcap_{i\in I} f^{<-1>}(B_i)$.
\end{enumerate}
\end{proposition}
\begin{proof} Exercice.
\end{proof}

En conclusion, l'image réciproque se comporte un peu mieux que l'image directe.

\begin{exercice}[Factorisation en surjection puis injection]
Montrer que toute application se factorise en une surjection suivie d'une injection. Autrement dit, montrer que pour toute fonction $f : E\to F$, il existe un ensemble $G$, et des fonctions $\phi : E\to G$ et $\psi : G\to F$ telles que $f = \psi\circ \phi$, avec $\phi$ surjective et $\psi$ injective.
\end{exercice}

\begin{exercice}[propriété du relèvement]
Soient $e : A\to B$ et $m : C\to D$ deux applications. On dit que $e$ est orthogonale (à gauche) à $m$ (ou que $m$ est orthogonale à droite à $e$), et on note $e\perp m$, si pour tout diagramme commutatif
\[
\xymatrix{
A \ar[r] \ar[d]_{e}& C\ar[d]^{m}\\
B \ar[r]& D 
},
\]
il existe un relèvement c'est-à-dire une application $r : B\to C$ faisant commuter le diagramme :
\[
\xymatrix{
A \ar[r] \ar[d]_{e}& C\ar[d]^{m}\\
B \ar[r]\ar[ur]^{r}& D 
}
\]
Montrer que si $e$ est surjective et $m$ est injective, alors $e\perp m$.
\end{exercice}




























Chapitre vide.

\section{L'ensemble $\N$ et la récurrence}

\section{Ensembles finis et cardinal}

\section{Coefficients binomiaux et combinatoire}



Attention, la présentation qui suit diffère sans doute beaucoup de celle vue en terminale : il faut faire l'effort de l'étudier en détail même si l'ordre dans lequel les notions sont introduites semble \og mauvais\fg : en fait, c'est le \og bon\fg{}  ordre.

Le cours d'arithmétique des polynômes suivra le même canevas (définitions semblables, mêmes lemmes aux mêmes endroits, mêmes preuves), de même que le cours d'algèbre générale sur les anneaux par la suite.

\section{Préliminaires}

\subsection{Division euclidienne}
\begin{proposition}[Division euclidienne]
Soit $a\in \N$ et $b\in \N*$. Il existe un unique couple $(b,r) \in \N^2$ vérifiant les deux propriétés suivantes:
\begin{enumerate}
\item $a=bq+r$;
\item $r < b$.
\end{enumerate}
L'entier $b$ est le \emph{quotient} de la division euclidienne de $a$ par $b$, et $r$ est le \emph{reste}.
Effectuer la division euclidienne de $a$ par $b$, c'est écrire $a = bq+r$ avec $b$ et $q$ comme plus haut.
\end{proposition}

Exemple : $17=5\times 3 + 2$ est la division euclidienne de $17$ par $5$. Le quotient est $3$ et il reste $2$. Par contre, l'écriture $17=5\times 2+7$ bien que correcte  n'est pas une division euclidienne, car le reste \emph{doit} être strictement inférieur à $5$, dans une division euclidienne.

\subsection{Idéaux de $\Z$}

Soit $I \subseteq \Z$ une partie de $\Z$. On dit que $I$ est un \emph{idéal} de $\Z$ si
\begin{enumerate}
\item C'est un \emph{sous-groupe} de $\Z$, c'est-à-dire $I$ contient $0$ et est stable par addition et opposé : 
\[ \forall x, y\in I, \: x+y \in I \text{et} -x \in I.\]
\item Il est \emph{absorbant pour la multiplication} c'est-à-dire:
\[  \text{Si } x \in I \text{ et }n\in \Z, \text{alors } n\cdot x \in I.\]
\end{enumerate}

\begin{proposition}
Tout idéal de $\Z$ est de la forme $\{k\alpha\:\mid \: k \in \Z\}$, avec $\alpha /in \N$. (Un tel ensemble est noté $\alpha\Z$.)
\end{proposition}

\begin{proof}
On va montrer que les sous-groupes de $\Z$ sont de cette forme, et que ce sont des idéaux.

Soit $G \subseteq \Z$ un sous-groupe. Soit $G^*_+ = G \cap \N^*$. Il y a deux cas:
\begin{enumerate}
\item Si $G^*_+$ est vide, cela signifie que $G$ ne possède aucun élément strictement positif. Comme $G$ est stable par opposé, il ne peut pas non plus contenir d'éléments strictement négatifs. Cela signifie que $G=\{0\}$.
\item Sinon, c'est une partie non vide de $\N$, qui possède donc un plus petit élément, notons-le $\alpha$.
Par définition, $G$ est stable par somme et opposé, donc $2\alpha\in G$ et $-\alpha \in G$ et plus généralement, pour tout $k\in \Z$, on a $k\alpha \in G$.
Donc $\alpha \subseteq G$. Montrons l'inclusion inverse.
Soit $x\in G$, positif. \'Ecrivons la division euclidienne de $x$ par $\alpha$. On a $x = \alpha q + r$, avec $r<\alpha$. Comme $G$ est stable par somme et différence et que $\alpha q \in G$, on en déduit que $r = x-\alpha q$ est également dans $G$. Or, $r<\alpha$, donc par minimalité de $\alpha$, $r=0$, ce qui montre que $x = \alpha q$, donc que $x\in \alpha\Z$.
Si $x$ est négatif, ce qui précède montre que $-x\in \alpha\Z$, donc que $x\in \alpha\Z$.
\end{enumerate}

On vérifie ensuite sans peine que tous les sous-groupes de $\Z$ sont en fait des idéaux de $\Z$.
\end{proof}


L'entier $\alpha$ dans la définition est appelé le générateur principal de $G$.


\section{Pgcd}

\begin{proposition}
Soient $a$ et $b$ deux entiers. L'ensemble $\{ak+bl\:\mid\: k, l\in \Z\}$ noté par définition $a\Z+b\Z$ ou $\langle a,b\rangle$, est un idéal de $\Z$. 
\end{proposition}
\begin{proof}
Appliquer la définition de sous-groupe, ce qui prouve que c'est un idéal par la section précédente.
\end{proof}

\begin{definition}
Soient $a$ et $b$ des entiers. Le générateur principal de $a\Z+b\Z$ est appelé le pgcd de $a$ et $b$, il est noté $\pgcd(a,b)$ ou bien $a\wedge b$.
\end{definition}

\begin{proposition}
Soient $a$ et $b$ des entiers, et $d=\pgcd(a,b)$.
On a les propriétés suivantes
\begin{enumerate}
\item L'entier $a$ est dans $a\Z+b\Z$, donc $d$ divise $a$. De même, $d$ divise $b$.
\item L'entier $df$ est dans $d\Z$, donc il existe $k$ et $l$ dans $\Z$, tels que $d = ak+bl$. On dit que $(k,l)$ est un couple (ou paire, par abus de langage) de Bézout pour $a$ et $b$. L'égalité $d=ak+bl$ est appelée \emph{relation de Bézout}.
\item Au sens de la divisibilité, $d$ est le plus grand diviseur commun  de $a$ et $b$. Ceci explique le nom (\emph{plus grand commun diviseur} de $d$. Cette propriété est précisée dans la proposition suivante.
\item Si $d=0$, alors $a=b=0$.
\item On a $\pgcd(a,b)=\pgcd(a,b) = \pgcd(a,-b)$, car $a\Z+b\Z = b\Z+a\Z=a\Z+(-b)\Z$.
\item $\pgcd(a,0)=|a|$.
\item $\pgcd(a,1)=1$.
\end{enumerate}
\end{proposition}

\begin{proposition}
Si $x>0$, $x|a$ et $x|b$, et $\forall m, m|a \text{ et } m|b \implies m|x$, alors $x=d$.
\end{proposition}
\begin{proof}
Si $x|a$ et $x|b$, alors $x|d$. D'autre part, $d|a$ et $d|a$, donc $d|x$. Donc finalement, $d=x$.
Attention, la condition $x>0$ est indispensable pour ce raisonnement. Deux entiers relatifs peuvent se diviser l'un l'autre, comme $1$ et $-1$, sans être égaux.
\end{proof}

\begin{proposition}
Soit $k>0$. On a $\pgcd(ka,kb)=k\pgcd(a,b)$.
\end{proposition}
\begin{proof}
Notons provisoirement $d_1 = \pgcd(a,b)$ et $d_2 = \pgcd(ka,kb)$.

Comme $d_1|a$ et $d_1|b$, on a $kd_1|ka$ et $kd_1|kb$ donc finalement $kd_1|d_2$. En  particulier, $k|d_2$ donc $\frac{d_2}{k}$ est un entier.

D'autre part, $d_2|ka$ et $d_2|kb$, donc  en divisant par $k$ et en utilisant la remarque précédente, on a $\frac{d_2}{k} | a$ et $\frac{d_2}{k} | b$ donc $\frac{d_2}{k} | d_1$, d'où $d_2 | kd_1$. 

Comme $kd_1$ et $d_2$ sont positifs, on en déduit $d_2=kd_1$.
\end{proof}

\subsection{Algorithme d'Euclide}

\begin{lemme}[d'Euclide]
Soient $a$, $b$ et $k$ des entiers relatifs. Alors:
\[ \pgcd(a,b) = \pgcd(a+kb,b).\]
\end{lemme}

\begin{proof}

\end{proof}

\section{Nombres premiers}

\subsection{Définition}

\subsection{Factorisation en produit d'irréductibles}

\section{Ppcm}

\section{Compléments}

\subsection{Petit théorème de Fermat}
\subsection{Théorème chinois}
\subsection{Indicatrice d'Euler}



\begin{definition} Soit $E$ un ensemble. Une \emph{relation binaire} $R$ sur $E$ est une application de $E\times E$ dans $\{\text{vrai, faux}\}$.
\end{definition}

On notera $xRy$ au lieu de $R(x,y)=\text{vrai}$.

\section{Relations d'ordre}

\begin{definition}
Soit $E$ un ensemble. Une relation binaire $R$ sur $E$ est
\begin{enumerate}
\item réflexive ssi $\forall x\in E, xRx$;
\item transitive ssi $\forall x, y, z\in E, xRy \text{ et } yRz \implies xRz$;
\item antisymétrique ssi $\forall x, y \in E, xRy\text{ et } yRx \implies x=y$.
\end{enumerate}

Une relation est une \emph{relation d'ordre} ssi elle est réflexive, transitive et antisymétrique.
\end{definition}

\begin{exemples}
$\leq$ est une relation d'ordre sur $\N$, ou sur $\Z$, $\Q$, $\R$. (Mais pas sur $\C$ : la relation $\leq$ n'est même pas \emph{définie} sur $\C$.)
$\subset$ est une relation d'ordre sur $\mathcal P(E)$.
$|$ (\og divise\fg) est une relation d'ordre sur $\N^*$
\end{exemples}

Attention : $<$ n'est pas une relation d'ordre, et $|$ n'est pas une relation d'ordre sur $\Z^*$. Pourquoi ?
% pas réflexive, pas antisymétrique (1 et -1)

\begin{definition}
Si $R$ est une relation d'ordre sur $E$, on peut lui associer une relation \emph{d'ordre strict}, définie par \og$ xRy\text{ et }x\neq y$\fg. (Remarque : une relation d'ordre strict n'est pas une relation d'ordre puisqu'elle n'est pas réflexive.)
\end{definition}

Un ensemble $E$ muni d'une relation d'ordre $R$ est appelé ensemble ordonné. Par exemple, $(\R, \leq)$ est un ensemble ordonné. S'il n'y a pas de confusion possible sur la relation d'ordre, on peut simplement dire que $E$ est ordonné. (Cependant, il y a en général plusieurs relations d'ordre sur un ensemble.)

Dans ce cours, on notera souvent $\leq_E$ au lieu de $R$ une relation d'ordre sur $E$, même si la relation n'a rien à voir avec $\leq$ sur $\R$.

% exos de base de vérification de définitions.

\paragraph{Vocabulaire sur les ensembles ordonnés}

\begin{definition}
Une relation d'ordre $\leq_E$ sur un ensemble $E$ est \emph{totale} si:
\[ \forall x, y\in E, x\leq_Ey\text{ ou } y\leq x.\]
\end{definition}

\begin{exemples}
La relation d'ordre $\leq$ sur $\R$ (ou $\N$, $\$Q$, $\Z$) est totale. Par contre, $\subset$ et $|$ ne sont pas totales. Par exemple, dans $\mathcal P(\R)$, les parties $\R_+$ et $]-3,6]$ ne sont pas comparables pour l'inclusion. Dans $\N^*$, les éléments $2$ et $3$ ne sont pas comparables pour la divisibilité.
\end{exemples}

\begin{definition}
Soient $(E,\leq_E)$ et $(F,\leq_F)$ des ensembles ordonnés, et $f : E\to F$. On dit que $f$ est \emph{croissante} si :
\[ \forall x, y\in E, x\leq_E y \implies f(x) \leq_F f(y),\]
et \emph{décroissante} si :
\[\forall x, y\in E, x\leq_E y \implies f(y) \leq_F f(x).\]
\end{definition}

(Remarque : dans cette situation, il est crucial de distinguer les relations d'ordre sur $E$ et sur $F$.)

\begin{exemple}
\begin{enumerate}
\item L'application $f : \R\to \R, x\mapsto x+e^x$ est croissante pour l'ordre usuel $\leq $ sur $\R$.
\item Si $E$ est fini, l'application $f : \mathcal P(E) \to \N, \: A\mapsto \operatorname{Card}(A)$ est croissante entre les ensembles ordonnés $(\mathcal P(E), \subset)$ et $(\N, \leq)$.
\item L'application $f : \mathcal P(E) \to \mathcal P(E), \: A\mapsto A^c$ est décroissante pour l'inclusion, car $A\subset B \implies B^c\subset A^c$.
\end{enumerate}
\end{exemple}

Comme d'habitude, après les définitions viennent les propositions et théorèmes.

\begin{proposition}
\begin{enumerate}
\item La composée de deux applications croissantes est croissante.
\item La composée de deux applications décroissantes est décroissante.
\item La composée d'une application décroissante et d'une décroissante est décroissante.
\end{enumerate}
\end{proposition}

\begin{proof}
Application directe de la définition.
\end{proof}

\section{Éléments remarquables dans un ensemble ordonné}

Soit $(E,\leq_E)$ un ensemble ordonné et $A\subset E$ une partie non vide. Un élément $m\in E$ est un \emph{majorant} de $A$ si $\forall a\in A, a\leq_E m$.

La partie $A$ est majorée si elle possède des majorants.

La partie $A$ possède un plus grand élément s'il existe un élément $m\in A$ qui majore $A$.

\begin{exemple}
La partie $[0,1]$ est majorée dans $\R$ car $1$, $2$, $\pi$ sont des majorants. Elle possède un plus grand élément : $1$.

La partie $[0,1[$ est majorée dans $\R$ (pour les mêmes raisons). Par contre, elle n'a pas de plus grand élément. Elle possède par contre un plus petit majorant réel, à savoir $1$, mais il n'appartient pas à  $A$.
\end{exemple}

\begin{proposition}
Si $A\subset E$ possède un plus grand élément, il est unique.
\end{proposition}
\begin{proof}
Soient $m$ et $m'$ deux plus grands éléments de $A$. Comme $m$ est un plus grand élément, on a par définition $\forall x\in A, x\leq_E m$ et donc en particulier $m'\leq_E m$. De même, comme $m'$ est un plus grand élément, on a $m\leq_E m'$. Par antisymétrie de la relation d'ordre, on a $m=m'$.
\end{proof}

Si $A$ possède un plus grand élément (unique par ce qui précède), on le note $\max(A)$.


On définit de même  la notion de minorant, de plus petit élément, et on montre que s'il existe un plus petit élément d'une partie $A$, il est unique. On le note alors $\min(A)$.

\begin{definition} La partie $A\subset E$ admet une borne supérieure $s\in E$ ssi:
\begin{enumerate}
\item $s$ est un majorant de $A$;
\item tout majorant de $A$ majore $s$.
\end{enumerate}
(En d'autres termes, $s$ est le plus petit des majorants de $A$, ou encore : l'ensemble de tous les majorants de $A$ possède un plus petit élément $s$.)
\end{definition}

Attention, contrairement à un plus grand élément, la borne supérieure de $A$, si elle existe, n'appartient pas forcément à $A$. 
\begin{exemple}
La partie $A=[0,2[ \subset \R$ possède une borne supérieure à savoir $2$.
\end{exemple}
\begin{proof}
D'une part, il est clair que $2$ est un majorant de $[0,2[$, c'est-à-dire que $\forall x\in [0,2[, \: x\leq 2$.

Vérifions la seconde partie de la définition.  Soit $m$ un majorant de $[0,2[$ et supposons par l'absurde que $m < 2$. On doit forcément avoir $0\leq m$ puisque $0\in [0,2[$. Donc $1+\frac{m}{2} \in [0,2[$. Comme $m$ est un majorant, on doit avoir $1+\frac{m}{2}\leq m$, donc $2+m\leq 2m$ donc $m\geq 2$, absurde.
\end{proof}

\begin{proposition}
Si $A$ possède une borne supérieure, elle est unique et on la note $\sup(A)$.
\end{proposition}

\printindex
\end{document}